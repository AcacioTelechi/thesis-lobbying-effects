\documentclass[12pt, a4paper, twoside, chapter=TITLE, subsection=TITLE, section=TITLE, subsubsection=TITLE, subsubsubsection=TITLE, english, german, brazil]{abntex2}
\usepackage{glossaries}
\usepackage[utf8]{inputenc}
\usepackage[brazilian,hyperpageref]{backref}
\usepackage{graphicx} % Required for inserting images
\usepackage[alf]{abntex2cite}	% Citações padrão ABNT,
\usepackage{pacotes}
\usepackage{pdflscape}
\usepackage{lscape}
\usepackage{tabularray}
\usepackage{csvsimple}
\usepackage{csvsimple}

% envs
\providecommand{\varitem}{} % to keep LaTeX quiet
\makeatletter
\newenvironment{statements}[1]
 {\renewcommand\varitem[1]{\item[\textbf{#1\arabic{enumi}\rlap{$##1$}.}]%
    \edef\@currentlabel{#1\arabic{enumi}{$##1$}}}%
  \enumerate[label=\textbf{#1\arabic*:}, ref=#1\arabic*]}
 {\endenumerate}
\makeatother

\newenvironment{hypotheses}{\statements{H}}{\endstatements}
\newenvironment{objectives}{\statements{O}}{\endstatements}
 
\titulo{Mensurando os Efeitos do Lobby no Comportamento Parlamentar: uma abordagem utilizando Tripla Diferença no Parlamento Europeu}
\tituloestrangeiro{--texto do título estrangeiro--}
\autor{Acácio Vasconcelos Telechi}
\orientador{Prof. Dr. Alexsandro Eugênio Pereira}
% \coorientador{--Monitor--}
\data{2024}
\instituicao{
    Universidade Federal do Paraná (UFPR) \par
    Departamento de Ciência Política \par
    Programa de Pós-Graduação em Ciência Política
}
\tipotrabalho{Tese}
\local{Curitiba, PR}
\preambulo{Texto para qualificação de tese de doutoramento em Ciência Política}

\setlength{\absparsep}{18pt} % ajusta o espaçamento dos parágrafos do resumo


\newacronym{ue}{UE}{União Europeia}
\newacronym{pe}{PE}{Parlamento Europeu}
\newacronym{ce}{CE}{Comissão Europeia}
\newacronym {mpe}{MPE}{Membro do Parlamento Europeu}
\newacronym{tfue}{TFUE}{Tratado sobre o Funcionamento da União Europeia}
\newacronym{airt}{AIRTO}{Acordo Interinstitucional entre o Parlamento Europeu, o Conselho da União Europeia e a Comissão Europeia sobre um Registo de Transparência Obrigatório}
\newacronym{hloga}{HLOGA}{\textit{Honest Leadership and Open Government Act}}
\newacronym{dg}{DG}{Direção-Geral}
\newacronym{ceca}{CECA}{Comunidade Europeia do Carvão e do Aço}
\newacronym{cee}{CEE}{Comunidade Econômica Europeia}
\newacronym{euratom}{Euratom}{Comunidade Europeia da Energia Atômica}
\newacronym{ual}{UAL}{Unidade de Assuntos Legislativos}
\newacronym{plo}{PLO}{Processo Legislativo Ordinário}
\newacronym{bce}{BCE}{Banco Central Europeu}
\newacronym{tce}{TCE}{Tribunal de Contas Europeu}
\newacronym{tj}{TJ}{Tribunal de Justiça}
\newacronym{al}{AL}{Atividade Legislativa}
\newacronym{coue}{CoUE}{Conselho da União Europeia}
\newacronym{coe}{CoE}{Conselho Europeu}
\newacronym{did}{DiD}{Diferenças em Diferenças}
\newacronym{ddd}{DDD}{Tripla Diferença}
\newacronym{ppml}{PPML}{Poisson Pseudo Maximum Likelihood}
\newacronym{glm}{GLM}{Generalized Linear Model}
\newacronym{mqo}{MQO}{Mínimos Quadrados Ordinários}
\newacronym{ong}{ONG}{Organzação Não-Governamental}
\makeglossaries

\begin{document}
    % \pretextual
    % \imprimircapa
    % \imprimirfolhaderosto[]* % talvez dê merda na numeração das páginas
    % \makeatother\cleardoublepage
    
    
    % \begin{resumo}[Resumo] 
    %     A influência de grupos de interesse, incluindo lobistas, na tomada de decisões
    %     políticas tem sido objeto de intenso debate e escrutínio nos últimos anos. A \acrfull{ue} não é exceção, pois se tornou um alvo altamente atrativo para esforços de lobby devido a sua natureza supranacional única e processos complexos de tomada de decisão, especialmente o \acrfull{pe}, que, após o Tratado de Lisboa, vem crescendo em relevância para a tomada de decisão da \acrshort{ue}. Esta tese buscará investigar os efeitos do lobby no comportamento parlamentar. Para isso, proponho a utilização do método da \acrfull{ddd} que combina a análise da atividade legislativa com dados sobre as interações dos parlamentares com lobistas controladas pelos domínios temáticos. A pesquisa se destaca por sua contribuição à literatura sobre lobby, oferecendo uma nova perspectiva sobre a influência desses grupos em um contexto institucional complexo como o \acrshort{pe}. Além disso, proponho, também, um \textit{framework} de análise do comportamento parlamentar, incorporando o papel do lobby como um fator determinante e enriquecendo a teoria sobre a atuação dos representantes eleitos. \\
    %     \vspace{\onelineskip}
    %     \noindent
    %     \textbf{Palavras-chave}: lobby, comportamento parlamentar, tripla-diferença, Parlamento Europeu, União Europeia.
    % \end{resumo}
    
    % \begin{resumo}[Abstract]
    %     \begin{otherlanguage*}{english}
    %         The influence of interest groups, including lobbyists, on political decision-making has been the subject of intense debate and scrutiny in recent years. The EU is no exception, as it has become a highly attractive target for lobbying efforts due to its unique supranational nature and complex decision-making processes, especially the EP, which, after the Treaty of Lisbon, has been growing in relevance for EU decision-making. This thesis will seek to investigate the effects of lobbying on parliamentary behavior. To do so, I propose the use of the Triple-Difference method, which combines the analysis of legislative activity with data on the interactions of parliamentarians with lobbyists, controlled by thematic domains. The research stands out for its contribution to the literature on lobbying, offering a new perspective on the influence of these groups in a complex institutional context such as the EP. In addition, I also propose a framework for analyzing parliamentary behavior, incorporating the role of lobbying as a determining factor and enriching the theory on the performance of elected representatives. \\
    %         \vspace{\onelineskip}
    %         \noindent
    %         \textbf{Keywords}: lobbying, parliamentary behavior, triple-difference, European Parliament, European Union.
    %     \end{otherlanguage*}
    % \end{resumo}
    
    % \pdfbookmark[0]{\contentsname}{toc}

    
    \tableofcontents*
    % \cleardoublepage
    
    \printglossaries
    % \listoffigures
    
    \textual % a partir daqui são os elementos textuais.
    \setlength{\parskip}{6pt}
        % \section{Poderes Legislativos}
\label{section:podereslegislativos}

O \acrshort{pe} possui um papel crucial no processo legislativo da \acrshort{ue}, atuando em conjunto com o \acrshort{coue} na aprovação da maioria das leis europeias, de acordo com o processo legislativo ordinário (artigo 289 do \acrshort{tfue}). O trabalho legislativo se inicia com a apresentação de um "texto legislativo" pela \acrshort{ce}, que detém o monopólio da iniciativa normativa. Em seguida, a proposta é analisada por um parlamentar no âmbito de uma comissão parlamentar, que elabora um relatório, sob a liderança de um relator designado. A escolha da comissão, normalmente, ocorre por decisão dos coordenadores de comissão. 

Cada proposta legislativa é confiada a um grupo político, o qual designa um "relator" para elaborar o relatório em nome da comissão. Demais grupos políticos podem designar "relatores-sombra" para coordenar as respectivas posições sobre o assunto em debate.

Após votação e possíveis alterações no âmbito da comissão, o relatório é submetido ao plenário do Parlamento para aprovação. O presidente da comissão preside as suas reuniões e dos seus coordenadores. Além disso, o presidente possui ingerência no processo de votação, bem como nas regras relativas à admissibilidade das alterações.

Os coordenadores da comissão são designados pelos grupos políticos. Os coordenadores muitas vezes se reúnem a portas fechadas (\textit{in camera}) às margens das reuniões das comissões. A comissão pode delegar aos coordenadores o poder de alocação dos relatórios e opiniões aos grupos, de decisão sobre audiências públicas na comissão, de pedidos de estudos, entre outras atividades relativas à organização dos trabalhos da comissão.

As comissões parlamentares e seus membros possuem o apoio administrativo pelos secretários de comissões. Esses funcionários organizam as reuniões da comissão, planejam e dão suporte e assessoria a respeito dos assuntos da comissão). Além dos secretários, há os assessores dos grupos políticos, os quais dão suporte e assessoria tanto para o coordenador do seu grupo quanto para os membros individualmente). Outros membros e órgãos de apoios são: os assessores dos membros da comissão, a \acrfull{ual}, o Serviço Jurídico, o Diretório para Atos Legislativos, os Departamentos de Políticas, o Serviço de Pesquisa do Parlamento Europeu, a assessoria de imprensa do \acrshort{pe} e os Diretórios-Gerais para Tradução e Interpretação. Vale mencionar que na terceira leitura, a \acrshort{ual} coordena a assistência administrativa para as delegações do Parlamento no comitê de conciliação.

Aprovado o texto na comissão, ele é encaminhado para o Plenário. Caso obtenha aprovação em sessão plenária, a posição do \acrshort{pe} terá sido adotada. Esse processo pode se repetir uma ou mais vezes a depender do tipo de procedimento e do acordo, ou não, alcançado com o Conselho.

No que diz respeito à adoção de atos legislativos, há dois procedimentos: o \acrfull{plo}, ou também chamado de codecisão; e os processos legislativos especiais, aplicados a casos específicos nos quais o Parlamento possui apenas papel consultivo.

Sobre os últimos, o Conselho atua como único legislador. Ele ocorre em questões definidas nos tratados da \acrshort{ue}, como por exemplo, em temas fiscais. Em tais casos, o \acrshort{pe} emite apenas um parecer consultivo. Em casos específicos, porém, quando assim determinado, o parecer consultivo é obrigatório. A matéria só poderá ter força de lei quando o \acrshort{pe} tiver emitido seu parecer (art. 311º, \acrshort{tfue}).

\begin{figure}
    \caption{Fluxograma resumido do Processo Legislativo Ordinário}
    \includegraphics[width=\textwidth]{imgs/Processo legislativo ordinário.drawio.png}
    \label{fig:plo}
    \centering
    \caption*{Fonte: o autor (2025)}
\end{figure}


No \acrshort{plo}, o Parlamento e o Conselho atuam como colegisladores. Há uma gama vasta de temas em que o processo de codecisão é demandado: governança econômica, imigração, energia, transportes, meio ambiente, proteção dos consumidores, entre outros. Atualmente, a maior parte das leis europeias são adotadas por esse processo. 

O Tratado de Maastricht (1992) introduziu o mecanismo de codecisão, porém ainda num escopo limitado de temas. O Tratado de Amsterdã (1999) ampliou e reforçou a eficácia do mecanismo. Em 2009, com a entrada em vigor do Tratado de Lisboa, a codecisão passou a ser denominada de \acrshort{plo}, tornando-se, assim, o principal processo legislativo adotado pela \acrshort{ue}. A figura \ref{fig:plo} resume o principal rito legislativo da \acrshort{ue}.

No \acrshort{plo}, a posição do Parlamento é encaminhada ao Conselho. Se o texto for aprovado sem alterações, a proposta é adotada. Caso, entretanto, haja alterações por parte do Conselho, o texto é reencaminhado para o \acrshort{pe} para segunda leitura. O Parlamento examina as alterações e pode aprová-las, rejeitá-las, ou propor novas alterações. No primeiro caso, a proposta legislativa é adotada após aprovação em sessão plenária. Se, contudo, as alterações do Conselho forem rejeitadas, a proposta é arquivada. No terceiro caso, o texto alterado é reencaminhado para o Conselho para segunda leitura.

Na segunda leitura do conselho, se o Conselho aprovar todas as alterações do \acrshort{pe}, o texto é, então aprovado. Se, todavia, não aprová-las, convoca-se o Comitê de Conciliação. Tal comitê é composto por igual número de deputados e representantes do Conselho. Seus membros buscam chegar a um texto aceito por ambos. Se a tentativa for fracassada, o texto é arquivado. Se houver sucesso, o texto é reencaminhado para o Parlamento e para o Conselho para terceira leitura. Em terceira leitura, nem o Conselho, nem o Parlamento podem propor alterações. Devem, portanto, ou rejeitar, ou aprovar o texto como foi enviado pelo Comitê de Conciliação.


O \acrshort{pe} também possui participação em atos não-legislativos, chamado de "aprovação", introduzido pelo Ato Único Europeu (1986). O mecanismo aplica-se em acordos de associação e de adesão à \acrshort{ue}, em alguns acordos comerciais, em casos de violações graves aos direitos fundamentais (art. 7º, \acrshort{tfue}). A aprovação também pode ser utilizada como ato legislativo em casos de legislação sobre combate à discriminação.

Há outros procedimentos cuja participação do \acrshort{pe} é destacada: parecer nos termos do artigo 140.º do \acrshort{tfue} (União Monetária), procedimentos relativos ao diálogo social (art. 154º, \acrshort{tfue}), à apreciação de acordos voluntários (art. 48º, \acrshort{tfue})), à codificação (art. 46º do Regimento), à medidas de execução e disposições delegadas.

No âmbito da União Monetária (art. 140.º do \acrshort{tfue}), o Parlamento emite um parecer sobre os progressos dos Estados-Membros em relação à adoção da moeda única, embora o Conselho tome a decisão final.

Nos procedimentos relativos ao diálogo social (artigos 154.º e 155.º do \acrshort{tfue}), o Parlamento atua recepcionando a consulta realizada pela Comissão aos parceiros sociais e avaliando acordos e convenções. 

Adicionalmente, o Parlamento é informado sobre acordos voluntários propostos pela Comissão como alternativa a medidas legislativas (artigo 48.º do Regimento) e pode apresentar uma resolução recomendando sua aprovação ou rejeição.

Quanto à codificação, que visa consolidar atos legislativos em um único texto para maior clareza (artigo 46.º do Regimento), a Comissão de Assuntos Jurídicos do Parlamento fica a cargo dessa consolidação.

O Parlamento também exerce controle sobre medidas de execução e disposições delegadas adotadas pela Comissão, podendo se opor a medidas que não estejam em conformidade com a legislação vigente ou que violem os princípios da subsidiariedade e proporcionalidade.

Embora a iniciativa legislativa na \acrshort{ue} seja de competência da \acrshort{ce}, o Parlamento também possui um direito de iniciativa, garantido pelo Tratado de Maastricht e consolidado no Tratado de Lisboa.

Esse direito permite ao Parlamento solicitar à Comissão a apresentação de propostas legislativas em áreas específicas, mediante aprovação da maioria de seus membros e com base em um relatório elaborado pela comissão parlamentar competente (artigo 225º do \acrshort{tfue}). O Parlamento pode, inclusive, estabelecer um prazo para a apresentação da proposta, mas a Comissão tem a prerrogativa de aceitar ou recusar o pedido.

Além disso, deputados individuais também podem apresentar propostas de atos da União com base nesse direito de iniciativa. A proposta é encaminhada ao Presidente do Parlamento, que a transmite à comissão competente para análise e possível apresentação à sessão plenária.

O Parlamento Europeu também exerce sua iniciativa através da elaboração de relatórios de iniciativa pelas comissões parlamentares, que podem apresentar propostas de resolução sobre assuntos de sua competência, desde que autorizados pela Conferência dos Presidentes (artigos 37, 46 e 52 do Regimento e artigo 17(1) do \acrshort{tfue}).

A participação do Parlamento na programação anual e plurianual da \acrshort{ue} também é relevante. O Parlamento coopera com a Comissão na elaboração do programa de trabalho, que define as prioridades da \acrshort{ue} para o período. Após a adoção do programa pela Comissão, um diálogo tripartido (trílogo) entre o Parlamento, o Conselho e a Comissão busca um acordo sobre a programação, conforme detalhado no anexo XIV do Regimento.

Em resumo, o direito de iniciativa do \acrshort{pe}, seja através de solicitações à Comissão, propostas de deputados individuais ou relatórios de iniciativa, fortalece seu papel no processo legislativo da \acrshort{ue}, complementando a iniciativa da Comissão.

\section{Poderes Orçamentários}

Com a entrada em vigor do Tratado de Lisboa, o \acrshort{pe} adquiriu um papel crucial na elaboração e aprovação do orçamento anual global da \acrshort{ue}, não só em colaboração com o Conselho, mas também com capacidade decisória na matéria.


O processo orçamentário anual inicia-se com a elaboração das previsões de receitas e despesas por todas as instituições da \acrshort{ue}. A \acrshort{ce}, com base nessas previsões, consolida e apresenta um projeto de orçamento ao Parlamento e ao Conselho. O Conselho, por sua vez, adota uma posição sobre o projeto e a encaminha ao Parlamento, que dispõe de um prazo para aprová-la ou emendá-la.

Durante as discussões no \acrshort{pe}, as comissões parlamentares debatem o projeto de orçamento e apresentam os seus pareceres à Comissão dos Orçamentos, responsável pela preparação da posição do Parlamento. A decisão do parlamento é tomada por maioria absoluta dos membros, podendo aceitar o parecer da \acrshort{ce}, propor alterações, ou aceitar tacitamente - caso não decida dentro de 42 dias.


Caso o Parlamento opte por alterar o projeto, um Comitê de Conciliação, composto por representantes de ambas as instituições, é convocado para buscar um consenso. Se um acordo for alcançado, o projeto comum é submetido à aprovação do Parlamento e do Conselho. Na ausência de consenso, a Comissão apresenta um novo projeto de orçamento, reiniciando o ciclo.

Vale mencionar que as decisões do Parlamento e do Conselho no que diz respeito às receitas e às despesas devem respeitar os limites das despesas da \acrshort{ue}, definidas no Quadro Financeiro Plurianual, negociado uma vez a cada sete anos.

% despesas anuais fixadas na programação financeira da \acrshort{ue}

Além da co-decisão na elaboração do orçamento anual, o \acrshort{pe} exerce um papel importante no controle da execução orçamentária. A \acrshort{ce}, responsável pela gestão do orçamento, está sujeita ao escrutínio do Parlamento, que avalia a conformidade das ações com as diretrizes estabelecidas e a eficácia na utilização dos recursos.

O Parlamento detém a prerrogativa de conceder ou recusar a quitação à Comissão, ou seja, a aprovação final das contas do exercício. Essa decisão é tomada após uma análise minuciosa dos relatórios financeiros e das atividades da Comissão, incluindo a avaliação do Tribunal de Contas Europeu.

Em suma, o \acrshort{pe} desempenha um papel ativo e influente no processo orçamentário da \acrshort{ue}, compartilhando com o Conselho a responsabilidade pela definição das prioridades de gastos e exercendo um controle rigoroso sobre a execução do orçamento, garantindo a transparência e a \textit{accountability} na gestão dos recursos públicos.

\section{Poderes Fiscalizatórios}

O \acrshort{pe} possui um amplo espectro de poderes fiscalizatórios sobre as principais instituições da \acrshort{ue}, buscando assegurar a \textit{accountability} e a transparência no funcionamento do bloco.

No âmbito do \acrshort{coue}, o \acrshort{pe} assegura sua influência através da participação do seu presidente nas reuniões do Conselho, onde este apresenta a posição do Parlamento sobre os temas em discussão. Além disso, o presidente do \acrshort{coue} reporta os resultados das reuniões ao \acrshort{pe}, permitindo um acompanhamento contínuo das decisões tomadas pelos chefes de Estado e de governo.

A interação com o Conselho da \acrshort{ue} também é fundamental para o exercício do controle parlamentar. O Parlamento debate o programa de cada presidência semestral do Conselho, apresenta perguntas e solicita informações sobre políticas específicas. O Alto Representante para os Negócios Estrangeiros e a Política de Segurança também presta contas ao Parlamento, apresentando relatórios periódicos sobre as ações da UE em matéria de política externa e de segurança.
 
O \acrshort{pe} exerce um controle crucial sobre a \acrshort{ce}, com a prerrogativa de aprovar ou destituir seus membros. Os comissários designados passam por audiências no Parlamento, onde são questionados sobre suas qualificações e planos de ação. A Comissão também apresenta relatórios regulares ao Parlamento, incluindo um relatório anual sobre as atividades da \acrshort{ue} e a execução do orçamento, garantindo a transparência e a prestação de contas. Além disso, o \acrshort{pe} pode apresentar monções de censura à Comissão, com a possibilidade de destituí-la em última instância. 

O Parlamento também exerce controle sobre outras instituições, como o \acrfull{tj}, o \acrfull{bce} e o \acrfull{tce}. O Parlamento pode solicitar ao \acrshort{tj} que tome medidas contra a Comissão ou o Conselho em caso de violação da legislação da \acrshort{ue}. No caso do \acrshort{bce}, o Parlamento aprova a nomeação dos seus principais dirigentes e recebe relatórios periódicos sobre a política monetária da zona euro. O Parlamento também utiliza os relatórios do \acrshort{tce} para avaliar a execução do orçamento da \acrshort{ue} e decidir sobre a concessão da quitação à Comissão.

Além disso, o \acrshort{pe} recebe petições de cidadãos da \acrshort{ue} e pode estabelecer comissões de inquérito para investigar alegações de violações ou má administração na aplicação do direito da \acrshort{ue}. Essas ferramentas permitem que o Parlamento atue como um canal de comunicação entre os cidadãos e as instituições europeias, garantindo que suas preocupações sejam ouvidas e consideradas.

Os poderes fiscalizatórios do \acrshort{pe}, portanto,  são essenciais para o funcionamento democrático e transparente da \acrshort{ue}. Através do exercício desses poderes, o Parlamento busca assegurar que as instituições da \acrshort{ue} atuem em conformidade com os princípios do Estado de Direito e que os interesses dos cidadãos europeus sejam protegidos e promovidos.


        % \section{Poderes Legislativos}
\label{section:podereslegislativos}

O \acrshort{pe} possui um papel crucial no processo legislativo da \acrshort{ue}, atuando em conjunto com o \acrshort{coue} na aprovação da maioria das leis europeias, de acordo com o processo legislativo ordinário (artigo 289 do \acrshort{tfue}). O trabalho legislativo se inicia com a apresentação de um "texto legislativo" pela \acrshort{ce}, que detém o monopólio da iniciativa normativa. Em seguida, a proposta é analisada por um parlamentar no âmbito de uma comissão parlamentar, que elabora um relatório, sob a liderança de um relator designado. A escolha da comissão, normalmente, ocorre por decisão dos coordenadores de comissão. 

Cada proposta legislativa é confiada a um grupo político, o qual designa um "relator" para elaborar o relatório em nome da comissão. Demais grupos políticos podem designar "relatores-sombra" para coordenar as respectivas posições sobre o assunto em debate.

Após votação e possíveis alterações no âmbito da comissão, o relatório é submetido ao plenário do Parlamento para aprovação. O presidente da comissão preside as suas reuniões e dos seus coordenadores. Além disso, o presidente possui ingerência no processo de votação, bem como nas regras relativas à admissibilidade das alterações.

Os coordenadores da comissão são designados pelos grupos políticos. Os coordenadores muitas vezes se reúnem a portas fechadas (\textit{in camera}) às margens das reuniões das comissões. A comissão pode delegar aos coordenadores o poder de alocação dos relatórios e opiniões aos grupos, de decisão sobre audiências públicas na comissão, de pedidos de estudos, entre outras atividades relativas à organização dos trabalhos da comissão.

As comissões parlamentares e seus membros possuem o apoio administrativo pelos secretários de comissões. Esses funcionários organizam as reuniões da comissão, planejam e dão suporte e assessoria a respeito dos assuntos da comissão). Além dos secretários, há os assessores dos grupos políticos, os quais dão suporte e assessoria tanto para o coordenador do seu grupo quanto para os membros individualmente). Outros membros e órgãos de apoios são: os assessores dos membros da comissão, a \acrfull{ual}, o Serviço Jurídico, o Diretório para Atos Legislativos, os Departamentos de Políticas, o Serviço de Pesquisa do Parlamento Europeu, a assessoria de imprensa do \acrshort{pe} e os Diretórios-Gerais para Tradução e Interpretação. Vale mencionar que na terceira leitura, a \acrshort{ual} coordena a assistência administrativa para as delegações do Parlamento no comitê de conciliação.

Aprovado o texto na comissão, ele é encaminhado para o Plenário. Caso obtenha aprovação em sessão plenária, a posição do \acrshort{pe} terá sido adotada. Esse processo pode se repetir uma ou mais vezes a depender do tipo de procedimento e do acordo, ou não, alcançado com o Conselho.

No que diz respeito à adoção de atos legislativos, há dois procedimentos: o \acrfull{plo}, ou também chamado de codecisão; e os processos legislativos especiais, aplicados a casos específicos nos quais o Parlamento possui apenas papel consultivo.

Sobre os últimos, o Conselho atua como único legislador. Ele ocorre em questões definidas nos tratados da \acrshort{ue}, como por exemplo, em temas fiscais. Em tais casos, o \acrshort{pe} emite apenas um parecer consultivo. Em casos específicos, porém, quando assim determinado, o parecer consultivo é obrigatório. A matéria só poderá ter força de lei quando o \acrshort{pe} tiver emitido seu parecer (art. 311º, \acrshort{tfue}).

\begin{figure}
    \caption{Fluxograma resumido do Processo Legislativo Ordinário}
    \includegraphics[width=\textwidth]{imgs/Processo legislativo ordinário.drawio.png}
    \label{fig:plo}
    \centering
    \caption*{Fonte: o autor (2025)}
\end{figure}


No \acrshort{plo}, o Parlamento e o Conselho atuam como colegisladores. Há uma gama vasta de temas em que o processo de codecisão é demandado: governança econômica, imigração, energia, transportes, meio ambiente, proteção dos consumidores, entre outros. Atualmente, a maior parte das leis europeias são adotadas por esse processo. 

O Tratado de Maastricht (1992) introduziu o mecanismo de codecisão, porém ainda num escopo limitado de temas. O Tratado de Amsterdã (1999) ampliou e reforçou a eficácia do mecanismo. Em 2009, com a entrada em vigor do Tratado de Lisboa, a codecisão passou a ser denominada de \acrshort{plo}, tornando-se, assim, o principal processo legislativo adotado pela \acrshort{ue}. A figura \ref{fig:plo} resume o principal rito legislativo da \acrshort{ue}.

No \acrshort{plo}, a posição do Parlamento é encaminhada ao Conselho. Se o texto for aprovado sem alterações, a proposta é adotada. Caso, entretanto, haja alterações por parte do Conselho, o texto é reencaminhado para o \acrshort{pe} para segunda leitura. O Parlamento examina as alterações e pode aprová-las, rejeitá-las, ou propor novas alterações. No primeiro caso, a proposta legislativa é adotada após aprovação em sessão plenária. Se, contudo, as alterações do Conselho forem rejeitadas, a proposta é arquivada. No terceiro caso, o texto alterado é reencaminhado para o Conselho para segunda leitura.

Na segunda leitura do conselho, se o Conselho aprovar todas as alterações do \acrshort{pe}, o texto é, então aprovado. Se, todavia, não aprová-las, convoca-se o Comitê de Conciliação. Tal comitê é composto por igual número de deputados e representantes do Conselho. Seus membros buscam chegar a um texto aceito por ambos. Se a tentativa for fracassada, o texto é arquivado. Se houver sucesso, o texto é reencaminhado para o Parlamento e para o Conselho para terceira leitura. Em terceira leitura, nem o Conselho, nem o Parlamento podem propor alterações. Devem, portanto, ou rejeitar, ou aprovar o texto como foi enviado pelo Comitê de Conciliação.


O \acrshort{pe} também possui participação em atos não-legislativos, chamado de "aprovação", introduzido pelo Ato Único Europeu (1986). O mecanismo aplica-se em acordos de associação e de adesão à \acrshort{ue}, em alguns acordos comerciais, em casos de violações graves aos direitos fundamentais (art. 7º, \acrshort{tfue}). A aprovação também pode ser utilizada como ato legislativo em casos de legislação sobre combate à discriminação.

Há outros procedimentos cuja participação do \acrshort{pe} é destacada: parecer nos termos do artigo 140.º do \acrshort{tfue} (União Monetária), procedimentos relativos ao diálogo social (art. 154º, \acrshort{tfue}), à apreciação de acordos voluntários (art. 48º, \acrshort{tfue})), à codificação (art. 46º do Regimento), à medidas de execução e disposições delegadas.

No âmbito da União Monetária (art. 140.º do \acrshort{tfue}), o Parlamento emite um parecer sobre os progressos dos Estados-Membros em relação à adoção da moeda única, embora o Conselho tome a decisão final.

Nos procedimentos relativos ao diálogo social (artigos 154.º e 155.º do \acrshort{tfue}), o Parlamento atua recepcionando a consulta realizada pela Comissão aos parceiros sociais e avaliando acordos e convenções. 

Adicionalmente, o Parlamento é informado sobre acordos voluntários propostos pela Comissão como alternativa a medidas legislativas (artigo 48.º do Regimento) e pode apresentar uma resolução recomendando sua aprovação ou rejeição.

Quanto à codificação, que visa consolidar atos legislativos em um único texto para maior clareza (artigo 46.º do Regimento), a Comissão de Assuntos Jurídicos do Parlamento fica a cargo dessa consolidação.

O Parlamento também exerce controle sobre medidas de execução e disposições delegadas adotadas pela Comissão, podendo se opor a medidas que não estejam em conformidade com a legislação vigente ou que violem os princípios da subsidiariedade e proporcionalidade.

Embora a iniciativa legislativa na \acrshort{ue} seja de competência da \acrshort{ce}, o Parlamento também possui um direito de iniciativa, garantido pelo Tratado de Maastricht e consolidado no Tratado de Lisboa.

Esse direito permite ao Parlamento solicitar à Comissão a apresentação de propostas legislativas em áreas específicas, mediante aprovação da maioria de seus membros e com base em um relatório elaborado pela comissão parlamentar competente (artigo 225º do \acrshort{tfue}). O Parlamento pode, inclusive, estabelecer um prazo para a apresentação da proposta, mas a Comissão tem a prerrogativa de aceitar ou recusar o pedido.

Além disso, deputados individuais também podem apresentar propostas de atos da União com base nesse direito de iniciativa. A proposta é encaminhada ao Presidente do Parlamento, que a transmite à comissão competente para análise e possível apresentação à sessão plenária.

O Parlamento Europeu também exerce sua iniciativa através da elaboração de relatórios de iniciativa pelas comissões parlamentares, que podem apresentar propostas de resolução sobre assuntos de sua competência, desde que autorizados pela Conferência dos Presidentes (artigos 37, 46 e 52 do Regimento e artigo 17(1) do \acrshort{tfue}).

A participação do Parlamento na programação anual e plurianual da \acrshort{ue} também é relevante. O Parlamento coopera com a Comissão na elaboração do programa de trabalho, que define as prioridades da \acrshort{ue} para o período. Após a adoção do programa pela Comissão, um diálogo tripartido (trílogo) entre o Parlamento, o Conselho e a Comissão busca um acordo sobre a programação, conforme detalhado no anexo XIV do Regimento.

Em resumo, o direito de iniciativa do \acrshort{pe}, seja através de solicitações à Comissão, propostas de deputados individuais ou relatórios de iniciativa, fortalece seu papel no processo legislativo da \acrshort{ue}, complementando a iniciativa da Comissão.

\section{Poderes Orçamentários}

Com a entrada em vigor do Tratado de Lisboa, o \acrshort{pe} adquiriu um papel crucial na elaboração e aprovação do orçamento anual global da \acrshort{ue}, não só em colaboração com o Conselho, mas também com capacidade decisória na matéria.


O processo orçamentário anual inicia-se com a elaboração das previsões de receitas e despesas por todas as instituições da \acrshort{ue}. A \acrshort{ce}, com base nessas previsões, consolida e apresenta um projeto de orçamento ao Parlamento e ao Conselho. O Conselho, por sua vez, adota uma posição sobre o projeto e a encaminha ao Parlamento, que dispõe de um prazo para aprová-la ou emendá-la.

Durante as discussões no \acrshort{pe}, as comissões parlamentares debatem o projeto de orçamento e apresentam os seus pareceres à Comissão dos Orçamentos, responsável pela preparação da posição do Parlamento. A decisão do parlamento é tomada por maioria absoluta dos membros, podendo aceitar o parecer da \acrshort{ce}, propor alterações, ou aceitar tacitamente - caso não decida dentro de 42 dias.


Caso o Parlamento opte por alterar o projeto, um Comitê de Conciliação, composto por representantes de ambas as instituições, é convocado para buscar um consenso. Se um acordo for alcançado, o projeto comum é submetido à aprovação do Parlamento e do Conselho. Na ausência de consenso, a Comissão apresenta um novo projeto de orçamento, reiniciando o ciclo.

Vale mencionar que as decisões do Parlamento e do Conselho no que diz respeito às receitas e às despesas devem respeitar os limites das despesas da \acrshort{ue}, definidas no Quadro Financeiro Plurianual, negociado uma vez a cada sete anos.

% despesas anuais fixadas na programação financeira da \acrshort{ue}

Além da co-decisão na elaboração do orçamento anual, o \acrshort{pe} exerce um papel importante no controle da execução orçamentária. A \acrshort{ce}, responsável pela gestão do orçamento, está sujeita ao escrutínio do Parlamento, que avalia a conformidade das ações com as diretrizes estabelecidas e a eficácia na utilização dos recursos.

O Parlamento detém a prerrogativa de conceder ou recusar a quitação à Comissão, ou seja, a aprovação final das contas do exercício. Essa decisão é tomada após uma análise minuciosa dos relatórios financeiros e das atividades da Comissão, incluindo a avaliação do Tribunal de Contas Europeu.

Em suma, o \acrshort{pe} desempenha um papel ativo e influente no processo orçamentário da \acrshort{ue}, compartilhando com o Conselho a responsabilidade pela definição das prioridades de gastos e exercendo um controle rigoroso sobre a execução do orçamento, garantindo a transparência e a \textit{accountability} na gestão dos recursos públicos.

\section{Poderes Fiscalizatórios}

O \acrshort{pe} possui um amplo espectro de poderes fiscalizatórios sobre as principais instituições da \acrshort{ue}, buscando assegurar a \textit{accountability} e a transparência no funcionamento do bloco.

No âmbito do \acrshort{coue}, o \acrshort{pe} assegura sua influência através da participação do seu presidente nas reuniões do Conselho, onde este apresenta a posição do Parlamento sobre os temas em discussão. Além disso, o presidente do \acrshort{coue} reporta os resultados das reuniões ao \acrshort{pe}, permitindo um acompanhamento contínuo das decisões tomadas pelos chefes de Estado e de governo.

A interação com o Conselho da \acrshort{ue} também é fundamental para o exercício do controle parlamentar. O Parlamento debate o programa de cada presidência semestral do Conselho, apresenta perguntas e solicita informações sobre políticas específicas. O Alto Representante para os Negócios Estrangeiros e a Política de Segurança também presta contas ao Parlamento, apresentando relatórios periódicos sobre as ações da UE em matéria de política externa e de segurança.
 
O \acrshort{pe} exerce um controle crucial sobre a \acrshort{ce}, com a prerrogativa de aprovar ou destituir seus membros. Os comissários designados passam por audiências no Parlamento, onde são questionados sobre suas qualificações e planos de ação. A Comissão também apresenta relatórios regulares ao Parlamento, incluindo um relatório anual sobre as atividades da \acrshort{ue} e a execução do orçamento, garantindo a transparência e a prestação de contas. Além disso, o \acrshort{pe} pode apresentar monções de censura à Comissão, com a possibilidade de destituí-la em última instância. 

O Parlamento também exerce controle sobre outras instituições, como o \acrfull{tj}, o \acrfull{bce} e o \acrfull{tce}. O Parlamento pode solicitar ao \acrshort{tj} que tome medidas contra a Comissão ou o Conselho em caso de violação da legislação da \acrshort{ue}. No caso do \acrshort{bce}, o Parlamento aprova a nomeação dos seus principais dirigentes e recebe relatórios periódicos sobre a política monetária da zona euro. O Parlamento também utiliza os relatórios do \acrshort{tce} para avaliar a execução do orçamento da \acrshort{ue} e decidir sobre a concessão da quitação à Comissão.

Além disso, o \acrshort{pe} recebe petições de cidadãos da \acrshort{ue} e pode estabelecer comissões de inquérito para investigar alegações de violações ou má administração na aplicação do direito da \acrshort{ue}. Essas ferramentas permitem que o Parlamento atue como um canal de comunicação entre os cidadãos e as instituições europeias, garantindo que suas preocupações sejam ouvidas e consideradas.

Os poderes fiscalizatórios do \acrshort{pe}, portanto,  são essenciais para o funcionamento democrático e transparente da \acrshort{ue}. Através do exercício desses poderes, o Parlamento busca assegurar que as instituições da \acrshort{ue} atuem em conformidade com os princípios do Estado de Direito e que os interesses dos cidadãos europeus sejam protegidos e promovidos.


        % \section{Poderes Legislativos}
\label{section:podereslegislativos}

O \acrshort{pe} possui um papel crucial no processo legislativo da \acrshort{ue}, atuando em conjunto com o \acrshort{coue} na aprovação da maioria das leis europeias, de acordo com o processo legislativo ordinário (artigo 289 do \acrshort{tfue}). O trabalho legislativo se inicia com a apresentação de um "texto legislativo" pela \acrshort{ce}, que detém o monopólio da iniciativa normativa. Em seguida, a proposta é analisada por um parlamentar no âmbito de uma comissão parlamentar, que elabora um relatório, sob a liderança de um relator designado. A escolha da comissão, normalmente, ocorre por decisão dos coordenadores de comissão. 

Cada proposta legislativa é confiada a um grupo político, o qual designa um "relator" para elaborar o relatório em nome da comissão. Demais grupos políticos podem designar "relatores-sombra" para coordenar as respectivas posições sobre o assunto em debate.

Após votação e possíveis alterações no âmbito da comissão, o relatório é submetido ao plenário do Parlamento para aprovação. O presidente da comissão preside as suas reuniões e dos seus coordenadores. Além disso, o presidente possui ingerência no processo de votação, bem como nas regras relativas à admissibilidade das alterações.

Os coordenadores da comissão são designados pelos grupos políticos. Os coordenadores muitas vezes se reúnem a portas fechadas (\textit{in camera}) às margens das reuniões das comissões. A comissão pode delegar aos coordenadores o poder de alocação dos relatórios e opiniões aos grupos, de decisão sobre audiências públicas na comissão, de pedidos de estudos, entre outras atividades relativas à organização dos trabalhos da comissão.

As comissões parlamentares e seus membros possuem o apoio administrativo pelos secretários de comissões. Esses funcionários organizam as reuniões da comissão, planejam e dão suporte e assessoria a respeito dos assuntos da comissão). Além dos secretários, há os assessores dos grupos políticos, os quais dão suporte e assessoria tanto para o coordenador do seu grupo quanto para os membros individualmente). Outros membros e órgãos de apoios são: os assessores dos membros da comissão, a \acrfull{ual}, o Serviço Jurídico, o Diretório para Atos Legislativos, os Departamentos de Políticas, o Serviço de Pesquisa do Parlamento Europeu, a assessoria de imprensa do \acrshort{pe} e os Diretórios-Gerais para Tradução e Interpretação. Vale mencionar que na terceira leitura, a \acrshort{ual} coordena a assistência administrativa para as delegações do Parlamento no comitê de conciliação.

Aprovado o texto na comissão, ele é encaminhado para o Plenário. Caso obtenha aprovação em sessão plenária, a posição do \acrshort{pe} terá sido adotada. Esse processo pode se repetir uma ou mais vezes a depender do tipo de procedimento e do acordo, ou não, alcançado com o Conselho.

No que diz respeito à adoção de atos legislativos, há dois procedimentos: o \acrfull{plo}, ou também chamado de codecisão; e os processos legislativos especiais, aplicados a casos específicos nos quais o Parlamento possui apenas papel consultivo.

Sobre os últimos, o Conselho atua como único legislador. Ele ocorre em questões definidas nos tratados da \acrshort{ue}, como por exemplo, em temas fiscais. Em tais casos, o \acrshort{pe} emite apenas um parecer consultivo. Em casos específicos, porém, quando assim determinado, o parecer consultivo é obrigatório. A matéria só poderá ter força de lei quando o \acrshort{pe} tiver emitido seu parecer (art. 311º, \acrshort{tfue}).

\begin{figure}
    \caption{Fluxograma resumido do Processo Legislativo Ordinário}
    \includegraphics[width=\textwidth]{imgs/Processo legislativo ordinário.drawio.png}
    \label{fig:plo}
    \centering
    \caption*{Fonte: o autor (2025)}
\end{figure}


No \acrshort{plo}, o Parlamento e o Conselho atuam como colegisladores. Há uma gama vasta de temas em que o processo de codecisão é demandado: governança econômica, imigração, energia, transportes, meio ambiente, proteção dos consumidores, entre outros. Atualmente, a maior parte das leis europeias são adotadas por esse processo. 

O Tratado de Maastricht (1992) introduziu o mecanismo de codecisão, porém ainda num escopo limitado de temas. O Tratado de Amsterdã (1999) ampliou e reforçou a eficácia do mecanismo. Em 2009, com a entrada em vigor do Tratado de Lisboa, a codecisão passou a ser denominada de \acrshort{plo}, tornando-se, assim, o principal processo legislativo adotado pela \acrshort{ue}. A figura \ref{fig:plo} resume o principal rito legislativo da \acrshort{ue}.

No \acrshort{plo}, a posição do Parlamento é encaminhada ao Conselho. Se o texto for aprovado sem alterações, a proposta é adotada. Caso, entretanto, haja alterações por parte do Conselho, o texto é reencaminhado para o \acrshort{pe} para segunda leitura. O Parlamento examina as alterações e pode aprová-las, rejeitá-las, ou propor novas alterações. No primeiro caso, a proposta legislativa é adotada após aprovação em sessão plenária. Se, contudo, as alterações do Conselho forem rejeitadas, a proposta é arquivada. No terceiro caso, o texto alterado é reencaminhado para o Conselho para segunda leitura.

Na segunda leitura do conselho, se o Conselho aprovar todas as alterações do \acrshort{pe}, o texto é, então aprovado. Se, todavia, não aprová-las, convoca-se o Comitê de Conciliação. Tal comitê é composto por igual número de deputados e representantes do Conselho. Seus membros buscam chegar a um texto aceito por ambos. Se a tentativa for fracassada, o texto é arquivado. Se houver sucesso, o texto é reencaminhado para o Parlamento e para o Conselho para terceira leitura. Em terceira leitura, nem o Conselho, nem o Parlamento podem propor alterações. Devem, portanto, ou rejeitar, ou aprovar o texto como foi enviado pelo Comitê de Conciliação.


O \acrshort{pe} também possui participação em atos não-legislativos, chamado de "aprovação", introduzido pelo Ato Único Europeu (1986). O mecanismo aplica-se em acordos de associação e de adesão à \acrshort{ue}, em alguns acordos comerciais, em casos de violações graves aos direitos fundamentais (art. 7º, \acrshort{tfue}). A aprovação também pode ser utilizada como ato legislativo em casos de legislação sobre combate à discriminação.

Há outros procedimentos cuja participação do \acrshort{pe} é destacada: parecer nos termos do artigo 140.º do \acrshort{tfue} (União Monetária), procedimentos relativos ao diálogo social (art. 154º, \acrshort{tfue}), à apreciação de acordos voluntários (art. 48º, \acrshort{tfue})), à codificação (art. 46º do Regimento), à medidas de execução e disposições delegadas.

No âmbito da União Monetária (art. 140.º do \acrshort{tfue}), o Parlamento emite um parecer sobre os progressos dos Estados-Membros em relação à adoção da moeda única, embora o Conselho tome a decisão final.

Nos procedimentos relativos ao diálogo social (artigos 154.º e 155.º do \acrshort{tfue}), o Parlamento atua recepcionando a consulta realizada pela Comissão aos parceiros sociais e avaliando acordos e convenções. 

Adicionalmente, o Parlamento é informado sobre acordos voluntários propostos pela Comissão como alternativa a medidas legislativas (artigo 48.º do Regimento) e pode apresentar uma resolução recomendando sua aprovação ou rejeição.

Quanto à codificação, que visa consolidar atos legislativos em um único texto para maior clareza (artigo 46.º do Regimento), a Comissão de Assuntos Jurídicos do Parlamento fica a cargo dessa consolidação.

O Parlamento também exerce controle sobre medidas de execução e disposições delegadas adotadas pela Comissão, podendo se opor a medidas que não estejam em conformidade com a legislação vigente ou que violem os princípios da subsidiariedade e proporcionalidade.

Embora a iniciativa legislativa na \acrshort{ue} seja de competência da \acrshort{ce}, o Parlamento também possui um direito de iniciativa, garantido pelo Tratado de Maastricht e consolidado no Tratado de Lisboa.

Esse direito permite ao Parlamento solicitar à Comissão a apresentação de propostas legislativas em áreas específicas, mediante aprovação da maioria de seus membros e com base em um relatório elaborado pela comissão parlamentar competente (artigo 225º do \acrshort{tfue}). O Parlamento pode, inclusive, estabelecer um prazo para a apresentação da proposta, mas a Comissão tem a prerrogativa de aceitar ou recusar o pedido.

Além disso, deputados individuais também podem apresentar propostas de atos da União com base nesse direito de iniciativa. A proposta é encaminhada ao Presidente do Parlamento, que a transmite à comissão competente para análise e possível apresentação à sessão plenária.

O Parlamento Europeu também exerce sua iniciativa através da elaboração de relatórios de iniciativa pelas comissões parlamentares, que podem apresentar propostas de resolução sobre assuntos de sua competência, desde que autorizados pela Conferência dos Presidentes (artigos 37, 46 e 52 do Regimento e artigo 17(1) do \acrshort{tfue}).

A participação do Parlamento na programação anual e plurianual da \acrshort{ue} também é relevante. O Parlamento coopera com a Comissão na elaboração do programa de trabalho, que define as prioridades da \acrshort{ue} para o período. Após a adoção do programa pela Comissão, um diálogo tripartido (trílogo) entre o Parlamento, o Conselho e a Comissão busca um acordo sobre a programação, conforme detalhado no anexo XIV do Regimento.

Em resumo, o direito de iniciativa do \acrshort{pe}, seja através de solicitações à Comissão, propostas de deputados individuais ou relatórios de iniciativa, fortalece seu papel no processo legislativo da \acrshort{ue}, complementando a iniciativa da Comissão.

\section{Poderes Orçamentários}

Com a entrada em vigor do Tratado de Lisboa, o \acrshort{pe} adquiriu um papel crucial na elaboração e aprovação do orçamento anual global da \acrshort{ue}, não só em colaboração com o Conselho, mas também com capacidade decisória na matéria.


O processo orçamentário anual inicia-se com a elaboração das previsões de receitas e despesas por todas as instituições da \acrshort{ue}. A \acrshort{ce}, com base nessas previsões, consolida e apresenta um projeto de orçamento ao Parlamento e ao Conselho. O Conselho, por sua vez, adota uma posição sobre o projeto e a encaminha ao Parlamento, que dispõe de um prazo para aprová-la ou emendá-la.

Durante as discussões no \acrshort{pe}, as comissões parlamentares debatem o projeto de orçamento e apresentam os seus pareceres à Comissão dos Orçamentos, responsável pela preparação da posição do Parlamento. A decisão do parlamento é tomada por maioria absoluta dos membros, podendo aceitar o parecer da \acrshort{ce}, propor alterações, ou aceitar tacitamente - caso não decida dentro de 42 dias.


Caso o Parlamento opte por alterar o projeto, um Comitê de Conciliação, composto por representantes de ambas as instituições, é convocado para buscar um consenso. Se um acordo for alcançado, o projeto comum é submetido à aprovação do Parlamento e do Conselho. Na ausência de consenso, a Comissão apresenta um novo projeto de orçamento, reiniciando o ciclo.

Vale mencionar que as decisões do Parlamento e do Conselho no que diz respeito às receitas e às despesas devem respeitar os limites das despesas da \acrshort{ue}, definidas no Quadro Financeiro Plurianual, negociado uma vez a cada sete anos.

% despesas anuais fixadas na programação financeira da \acrshort{ue}

Além da co-decisão na elaboração do orçamento anual, o \acrshort{pe} exerce um papel importante no controle da execução orçamentária. A \acrshort{ce}, responsável pela gestão do orçamento, está sujeita ao escrutínio do Parlamento, que avalia a conformidade das ações com as diretrizes estabelecidas e a eficácia na utilização dos recursos.

O Parlamento detém a prerrogativa de conceder ou recusar a quitação à Comissão, ou seja, a aprovação final das contas do exercício. Essa decisão é tomada após uma análise minuciosa dos relatórios financeiros e das atividades da Comissão, incluindo a avaliação do Tribunal de Contas Europeu.

Em suma, o \acrshort{pe} desempenha um papel ativo e influente no processo orçamentário da \acrshort{ue}, compartilhando com o Conselho a responsabilidade pela definição das prioridades de gastos e exercendo um controle rigoroso sobre a execução do orçamento, garantindo a transparência e a \textit{accountability} na gestão dos recursos públicos.

\section{Poderes Fiscalizatórios}

O \acrshort{pe} possui um amplo espectro de poderes fiscalizatórios sobre as principais instituições da \acrshort{ue}, buscando assegurar a \textit{accountability} e a transparência no funcionamento do bloco.

No âmbito do \acrshort{coue}, o \acrshort{pe} assegura sua influência através da participação do seu presidente nas reuniões do Conselho, onde este apresenta a posição do Parlamento sobre os temas em discussão. Além disso, o presidente do \acrshort{coue} reporta os resultados das reuniões ao \acrshort{pe}, permitindo um acompanhamento contínuo das decisões tomadas pelos chefes de Estado e de governo.

A interação com o Conselho da \acrshort{ue} também é fundamental para o exercício do controle parlamentar. O Parlamento debate o programa de cada presidência semestral do Conselho, apresenta perguntas e solicita informações sobre políticas específicas. O Alto Representante para os Negócios Estrangeiros e a Política de Segurança também presta contas ao Parlamento, apresentando relatórios periódicos sobre as ações da UE em matéria de política externa e de segurança.
 
O \acrshort{pe} exerce um controle crucial sobre a \acrshort{ce}, com a prerrogativa de aprovar ou destituir seus membros. Os comissários designados passam por audiências no Parlamento, onde são questionados sobre suas qualificações e planos de ação. A Comissão também apresenta relatórios regulares ao Parlamento, incluindo um relatório anual sobre as atividades da \acrshort{ue} e a execução do orçamento, garantindo a transparência e a prestação de contas. Além disso, o \acrshort{pe} pode apresentar monções de censura à Comissão, com a possibilidade de destituí-la em última instância. 

O Parlamento também exerce controle sobre outras instituições, como o \acrfull{tj}, o \acrfull{bce} e o \acrfull{tce}. O Parlamento pode solicitar ao \acrshort{tj} que tome medidas contra a Comissão ou o Conselho em caso de violação da legislação da \acrshort{ue}. No caso do \acrshort{bce}, o Parlamento aprova a nomeação dos seus principais dirigentes e recebe relatórios periódicos sobre a política monetária da zona euro. O Parlamento também utiliza os relatórios do \acrshort{tce} para avaliar a execução do orçamento da \acrshort{ue} e decidir sobre a concessão da quitação à Comissão.

Além disso, o \acrshort{pe} recebe petições de cidadãos da \acrshort{ue} e pode estabelecer comissões de inquérito para investigar alegações de violações ou má administração na aplicação do direito da \acrshort{ue}. Essas ferramentas permitem que o Parlamento atue como um canal de comunicação entre os cidadãos e as instituições europeias, garantindo que suas preocupações sejam ouvidas e consideradas.

Os poderes fiscalizatórios do \acrshort{pe}, portanto,  são essenciais para o funcionamento democrático e transparente da \acrshort{ue}. Através do exercício desses poderes, o Parlamento busca assegurar que as instituições da \acrshort{ue} atuem em conformidade com os princípios do Estado de Direito e que os interesses dos cidadãos europeus sejam protegidos e promovidos.


        \section{Poderes Legislativos}
\label{section:podereslegislativos}

O \acrshort{pe} possui um papel crucial no processo legislativo da \acrshort{ue}, atuando em conjunto com o \acrshort{coue} na aprovação da maioria das leis europeias, de acordo com o processo legislativo ordinário (artigo 289 do \acrshort{tfue}). O trabalho legislativo se inicia com a apresentação de um "texto legislativo" pela \acrshort{ce}, que detém o monopólio da iniciativa normativa. Em seguida, a proposta é analisada por um parlamentar no âmbito de uma comissão parlamentar, que elabora um relatório, sob a liderança de um relator designado. A escolha da comissão, normalmente, ocorre por decisão dos coordenadores de comissão. 

Cada proposta legislativa é confiada a um grupo político, o qual designa um "relator" para elaborar o relatório em nome da comissão. Demais grupos políticos podem designar "relatores-sombra" para coordenar as respectivas posições sobre o assunto em debate.

Após votação e possíveis alterações no âmbito da comissão, o relatório é submetido ao plenário do Parlamento para aprovação. O presidente da comissão preside as suas reuniões e dos seus coordenadores. Além disso, o presidente possui ingerência no processo de votação, bem como nas regras relativas à admissibilidade das alterações.

Os coordenadores da comissão são designados pelos grupos políticos. Os coordenadores muitas vezes se reúnem a portas fechadas (\textit{in camera}) às margens das reuniões das comissões. A comissão pode delegar aos coordenadores o poder de alocação dos relatórios e opiniões aos grupos, de decisão sobre audiências públicas na comissão, de pedidos de estudos, entre outras atividades relativas à organização dos trabalhos da comissão.

As comissões parlamentares e seus membros possuem o apoio administrativo pelos secretários de comissões. Esses funcionários organizam as reuniões da comissão, planejam e dão suporte e assessoria a respeito dos assuntos da comissão). Além dos secretários, há os assessores dos grupos políticos, os quais dão suporte e assessoria tanto para o coordenador do seu grupo quanto para os membros individualmente). Outros membros e órgãos de apoios são: os assessores dos membros da comissão, a \acrfull{ual}, o Serviço Jurídico, o Diretório para Atos Legislativos, os Departamentos de Políticas, o Serviço de Pesquisa do Parlamento Europeu, a assessoria de imprensa do \acrshort{pe} e os Diretórios-Gerais para Tradução e Interpretação. Vale mencionar que na terceira leitura, a \acrshort{ual} coordena a assistência administrativa para as delegações do Parlamento no comitê de conciliação.

Aprovado o texto na comissão, ele é encaminhado para o Plenário. Caso obtenha aprovação em sessão plenária, a posição do \acrshort{pe} terá sido adotada. Esse processo pode se repetir uma ou mais vezes a depender do tipo de procedimento e do acordo, ou não, alcançado com o Conselho.

No que diz respeito à adoção de atos legislativos, há dois procedimentos: o \acrfull{plo}, ou também chamado de codecisão; e os processos legislativos especiais, aplicados a casos específicos nos quais o Parlamento possui apenas papel consultivo.

Sobre os últimos, o Conselho atua como único legislador. Ele ocorre em questões definidas nos tratados da \acrshort{ue}, como por exemplo, em temas fiscais. Em tais casos, o \acrshort{pe} emite apenas um parecer consultivo. Em casos específicos, porém, quando assim determinado, o parecer consultivo é obrigatório. A matéria só poderá ter força de lei quando o \acrshort{pe} tiver emitido seu parecer (art. 311º, \acrshort{tfue}).

\begin{figure}
    \caption{Fluxograma resumido do Processo Legislativo Ordinário}
    \includegraphics[width=\textwidth]{imgs/Processo legislativo ordinário.drawio.png}
    \label{fig:plo}
    \centering
    \caption*{Fonte: o autor (2025)}
\end{figure}


No \acrshort{plo}, o Parlamento e o Conselho atuam como colegisladores. Há uma gama vasta de temas em que o processo de codecisão é demandado: governança econômica, imigração, energia, transportes, meio ambiente, proteção dos consumidores, entre outros. Atualmente, a maior parte das leis europeias são adotadas por esse processo. 

O Tratado de Maastricht (1992) introduziu o mecanismo de codecisão, porém ainda num escopo limitado de temas. O Tratado de Amsterdã (1999) ampliou e reforçou a eficácia do mecanismo. Em 2009, com a entrada em vigor do Tratado de Lisboa, a codecisão passou a ser denominada de \acrshort{plo}, tornando-se, assim, o principal processo legislativo adotado pela \acrshort{ue}. A figura \ref{fig:plo} resume o principal rito legislativo da \acrshort{ue}.

No \acrshort{plo}, a posição do Parlamento é encaminhada ao Conselho. Se o texto for aprovado sem alterações, a proposta é adotada. Caso, entretanto, haja alterações por parte do Conselho, o texto é reencaminhado para o \acrshort{pe} para segunda leitura. O Parlamento examina as alterações e pode aprová-las, rejeitá-las, ou propor novas alterações. No primeiro caso, a proposta legislativa é adotada após aprovação em sessão plenária. Se, contudo, as alterações do Conselho forem rejeitadas, a proposta é arquivada. No terceiro caso, o texto alterado é reencaminhado para o Conselho para segunda leitura.

Na segunda leitura do conselho, se o Conselho aprovar todas as alterações do \acrshort{pe}, o texto é, então aprovado. Se, todavia, não aprová-las, convoca-se o Comitê de Conciliação. Tal comitê é composto por igual número de deputados e representantes do Conselho. Seus membros buscam chegar a um texto aceito por ambos. Se a tentativa for fracassada, o texto é arquivado. Se houver sucesso, o texto é reencaminhado para o Parlamento e para o Conselho para terceira leitura. Em terceira leitura, nem o Conselho, nem o Parlamento podem propor alterações. Devem, portanto, ou rejeitar, ou aprovar o texto como foi enviado pelo Comitê de Conciliação.


O \acrshort{pe} também possui participação em atos não-legislativos, chamado de "aprovação", introduzido pelo Ato Único Europeu (1986). O mecanismo aplica-se em acordos de associação e de adesão à \acrshort{ue}, em alguns acordos comerciais, em casos de violações graves aos direitos fundamentais (art. 7º, \acrshort{tfue}). A aprovação também pode ser utilizada como ato legislativo em casos de legislação sobre combate à discriminação.

Há outros procedimentos cuja participação do \acrshort{pe} é destacada: parecer nos termos do artigo 140.º do \acrshort{tfue} (União Monetária), procedimentos relativos ao diálogo social (art. 154º, \acrshort{tfue}), à apreciação de acordos voluntários (art. 48º, \acrshort{tfue})), à codificação (art. 46º do Regimento), à medidas de execução e disposições delegadas.

No âmbito da União Monetária (art. 140.º do \acrshort{tfue}), o Parlamento emite um parecer sobre os progressos dos Estados-Membros em relação à adoção da moeda única, embora o Conselho tome a decisão final.

Nos procedimentos relativos ao diálogo social (artigos 154.º e 155.º do \acrshort{tfue}), o Parlamento atua recepcionando a consulta realizada pela Comissão aos parceiros sociais e avaliando acordos e convenções. 

Adicionalmente, o Parlamento é informado sobre acordos voluntários propostos pela Comissão como alternativa a medidas legislativas (artigo 48.º do Regimento) e pode apresentar uma resolução recomendando sua aprovação ou rejeição.

Quanto à codificação, que visa consolidar atos legislativos em um único texto para maior clareza (artigo 46.º do Regimento), a Comissão de Assuntos Jurídicos do Parlamento fica a cargo dessa consolidação.

O Parlamento também exerce controle sobre medidas de execução e disposições delegadas adotadas pela Comissão, podendo se opor a medidas que não estejam em conformidade com a legislação vigente ou que violem os princípios da subsidiariedade e proporcionalidade.

Embora a iniciativa legislativa na \acrshort{ue} seja de competência da \acrshort{ce}, o Parlamento também possui um direito de iniciativa, garantido pelo Tratado de Maastricht e consolidado no Tratado de Lisboa.

Esse direito permite ao Parlamento solicitar à Comissão a apresentação de propostas legislativas em áreas específicas, mediante aprovação da maioria de seus membros e com base em um relatório elaborado pela comissão parlamentar competente (artigo 225º do \acrshort{tfue}). O Parlamento pode, inclusive, estabelecer um prazo para a apresentação da proposta, mas a Comissão tem a prerrogativa de aceitar ou recusar o pedido.

Além disso, deputados individuais também podem apresentar propostas de atos da União com base nesse direito de iniciativa. A proposta é encaminhada ao Presidente do Parlamento, que a transmite à comissão competente para análise e possível apresentação à sessão plenária.

O Parlamento Europeu também exerce sua iniciativa através da elaboração de relatórios de iniciativa pelas comissões parlamentares, que podem apresentar propostas de resolução sobre assuntos de sua competência, desde que autorizados pela Conferência dos Presidentes (artigos 37, 46 e 52 do Regimento e artigo 17(1) do \acrshort{tfue}).

A participação do Parlamento na programação anual e plurianual da \acrshort{ue} também é relevante. O Parlamento coopera com a Comissão na elaboração do programa de trabalho, que define as prioridades da \acrshort{ue} para o período. Após a adoção do programa pela Comissão, um diálogo tripartido (trílogo) entre o Parlamento, o Conselho e a Comissão busca um acordo sobre a programação, conforme detalhado no anexo XIV do Regimento.

Em resumo, o direito de iniciativa do \acrshort{pe}, seja através de solicitações à Comissão, propostas de deputados individuais ou relatórios de iniciativa, fortalece seu papel no processo legislativo da \acrshort{ue}, complementando a iniciativa da Comissão.

\section{Poderes Orçamentários}

Com a entrada em vigor do Tratado de Lisboa, o \acrshort{pe} adquiriu um papel crucial na elaboração e aprovação do orçamento anual global da \acrshort{ue}, não só em colaboração com o Conselho, mas também com capacidade decisória na matéria.


O processo orçamentário anual inicia-se com a elaboração das previsões de receitas e despesas por todas as instituições da \acrshort{ue}. A \acrshort{ce}, com base nessas previsões, consolida e apresenta um projeto de orçamento ao Parlamento e ao Conselho. O Conselho, por sua vez, adota uma posição sobre o projeto e a encaminha ao Parlamento, que dispõe de um prazo para aprová-la ou emendá-la.

Durante as discussões no \acrshort{pe}, as comissões parlamentares debatem o projeto de orçamento e apresentam os seus pareceres à Comissão dos Orçamentos, responsável pela preparação da posição do Parlamento. A decisão do parlamento é tomada por maioria absoluta dos membros, podendo aceitar o parecer da \acrshort{ce}, propor alterações, ou aceitar tacitamente - caso não decida dentro de 42 dias.


Caso o Parlamento opte por alterar o projeto, um Comitê de Conciliação, composto por representantes de ambas as instituições, é convocado para buscar um consenso. Se um acordo for alcançado, o projeto comum é submetido à aprovação do Parlamento e do Conselho. Na ausência de consenso, a Comissão apresenta um novo projeto de orçamento, reiniciando o ciclo.

Vale mencionar que as decisões do Parlamento e do Conselho no que diz respeito às receitas e às despesas devem respeitar os limites das despesas da \acrshort{ue}, definidas no Quadro Financeiro Plurianual, negociado uma vez a cada sete anos.

% despesas anuais fixadas na programação financeira da \acrshort{ue}

Além da co-decisão na elaboração do orçamento anual, o \acrshort{pe} exerce um papel importante no controle da execução orçamentária. A \acrshort{ce}, responsável pela gestão do orçamento, está sujeita ao escrutínio do Parlamento, que avalia a conformidade das ações com as diretrizes estabelecidas e a eficácia na utilização dos recursos.

O Parlamento detém a prerrogativa de conceder ou recusar a quitação à Comissão, ou seja, a aprovação final das contas do exercício. Essa decisão é tomada após uma análise minuciosa dos relatórios financeiros e das atividades da Comissão, incluindo a avaliação do Tribunal de Contas Europeu.

Em suma, o \acrshort{pe} desempenha um papel ativo e influente no processo orçamentário da \acrshort{ue}, compartilhando com o Conselho a responsabilidade pela definição das prioridades de gastos e exercendo um controle rigoroso sobre a execução do orçamento, garantindo a transparência e a \textit{accountability} na gestão dos recursos públicos.

\section{Poderes Fiscalizatórios}

O \acrshort{pe} possui um amplo espectro de poderes fiscalizatórios sobre as principais instituições da \acrshort{ue}, buscando assegurar a \textit{accountability} e a transparência no funcionamento do bloco.

No âmbito do \acrshort{coue}, o \acrshort{pe} assegura sua influência através da participação do seu presidente nas reuniões do Conselho, onde este apresenta a posição do Parlamento sobre os temas em discussão. Além disso, o presidente do \acrshort{coue} reporta os resultados das reuniões ao \acrshort{pe}, permitindo um acompanhamento contínuo das decisões tomadas pelos chefes de Estado e de governo.

A interação com o Conselho da \acrshort{ue} também é fundamental para o exercício do controle parlamentar. O Parlamento debate o programa de cada presidência semestral do Conselho, apresenta perguntas e solicita informações sobre políticas específicas. O Alto Representante para os Negócios Estrangeiros e a Política de Segurança também presta contas ao Parlamento, apresentando relatórios periódicos sobre as ações da UE em matéria de política externa e de segurança.
 
O \acrshort{pe} exerce um controle crucial sobre a \acrshort{ce}, com a prerrogativa de aprovar ou destituir seus membros. Os comissários designados passam por audiências no Parlamento, onde são questionados sobre suas qualificações e planos de ação. A Comissão também apresenta relatórios regulares ao Parlamento, incluindo um relatório anual sobre as atividades da \acrshort{ue} e a execução do orçamento, garantindo a transparência e a prestação de contas. Além disso, o \acrshort{pe} pode apresentar monções de censura à Comissão, com a possibilidade de destituí-la em última instância. 

O Parlamento também exerce controle sobre outras instituições, como o \acrfull{tj}, o \acrfull{bce} e o \acrfull{tce}. O Parlamento pode solicitar ao \acrshort{tj} que tome medidas contra a Comissão ou o Conselho em caso de violação da legislação da \acrshort{ue}. No caso do \acrshort{bce}, o Parlamento aprova a nomeação dos seus principais dirigentes e recebe relatórios periódicos sobre a política monetária da zona euro. O Parlamento também utiliza os relatórios do \acrshort{tce} para avaliar a execução do orçamento da \acrshort{ue} e decidir sobre a concessão da quitação à Comissão.

Além disso, o \acrshort{pe} recebe petições de cidadãos da \acrshort{ue} e pode estabelecer comissões de inquérito para investigar alegações de violações ou má administração na aplicação do direito da \acrshort{ue}. Essas ferramentas permitem que o Parlamento atue como um canal de comunicação entre os cidadãos e as instituições europeias, garantindo que suas preocupações sejam ouvidas e consideradas.

Os poderes fiscalizatórios do \acrshort{pe}, portanto,  são essenciais para o funcionamento democrático e transparente da \acrshort{ue}. Através do exercício desses poderes, o Parlamento busca assegurar que as instituições da \acrshort{ue} atuem em conformidade com os princípios do Estado de Direito e que os interesses dos cidadãos europeus sejam protegidos e promovidos.


        \section{Poderes Legislativos}
\label{section:podereslegislativos}

O \acrshort{pe} possui um papel crucial no processo legislativo da \acrshort{ue}, atuando em conjunto com o \acrshort{coue} na aprovação da maioria das leis europeias, de acordo com o processo legislativo ordinário (artigo 289 do \acrshort{tfue}). O trabalho legislativo se inicia com a apresentação de um "texto legislativo" pela \acrshort{ce}, que detém o monopólio da iniciativa normativa. Em seguida, a proposta é analisada por um parlamentar no âmbito de uma comissão parlamentar, que elabora um relatório, sob a liderança de um relator designado. A escolha da comissão, normalmente, ocorre por decisão dos coordenadores de comissão. 

Cada proposta legislativa é confiada a um grupo político, o qual designa um "relator" para elaborar o relatório em nome da comissão. Demais grupos políticos podem designar "relatores-sombra" para coordenar as respectivas posições sobre o assunto em debate.

Após votação e possíveis alterações no âmbito da comissão, o relatório é submetido ao plenário do Parlamento para aprovação. O presidente da comissão preside as suas reuniões e dos seus coordenadores. Além disso, o presidente possui ingerência no processo de votação, bem como nas regras relativas à admissibilidade das alterações.

Os coordenadores da comissão são designados pelos grupos políticos. Os coordenadores muitas vezes se reúnem a portas fechadas (\textit{in camera}) às margens das reuniões das comissões. A comissão pode delegar aos coordenadores o poder de alocação dos relatórios e opiniões aos grupos, de decisão sobre audiências públicas na comissão, de pedidos de estudos, entre outras atividades relativas à organização dos trabalhos da comissão.

As comissões parlamentares e seus membros possuem o apoio administrativo pelos secretários de comissões. Esses funcionários organizam as reuniões da comissão, planejam e dão suporte e assessoria a respeito dos assuntos da comissão). Além dos secretários, há os assessores dos grupos políticos, os quais dão suporte e assessoria tanto para o coordenador do seu grupo quanto para os membros individualmente). Outros membros e órgãos de apoios são: os assessores dos membros da comissão, a \acrfull{ual}, o Serviço Jurídico, o Diretório para Atos Legislativos, os Departamentos de Políticas, o Serviço de Pesquisa do Parlamento Europeu, a assessoria de imprensa do \acrshort{pe} e os Diretórios-Gerais para Tradução e Interpretação. Vale mencionar que na terceira leitura, a \acrshort{ual} coordena a assistência administrativa para as delegações do Parlamento no comitê de conciliação.

Aprovado o texto na comissão, ele é encaminhado para o Plenário. Caso obtenha aprovação em sessão plenária, a posição do \acrshort{pe} terá sido adotada. Esse processo pode se repetir uma ou mais vezes a depender do tipo de procedimento e do acordo, ou não, alcançado com o Conselho.

No que diz respeito à adoção de atos legislativos, há dois procedimentos: o \acrfull{plo}, ou também chamado de codecisão; e os processos legislativos especiais, aplicados a casos específicos nos quais o Parlamento possui apenas papel consultivo.

Sobre os últimos, o Conselho atua como único legislador. Ele ocorre em questões definidas nos tratados da \acrshort{ue}, como por exemplo, em temas fiscais. Em tais casos, o \acrshort{pe} emite apenas um parecer consultivo. Em casos específicos, porém, quando assim determinado, o parecer consultivo é obrigatório. A matéria só poderá ter força de lei quando o \acrshort{pe} tiver emitido seu parecer (art. 311º, \acrshort{tfue}).

\begin{figure}
    \caption{Fluxograma resumido do Processo Legislativo Ordinário}
    \includegraphics[width=\textwidth]{imgs/Processo legislativo ordinário.drawio.png}
    \label{fig:plo}
    \centering
    \caption*{Fonte: o autor (2025)}
\end{figure}


No \acrshort{plo}, o Parlamento e o Conselho atuam como colegisladores. Há uma gama vasta de temas em que o processo de codecisão é demandado: governança econômica, imigração, energia, transportes, meio ambiente, proteção dos consumidores, entre outros. Atualmente, a maior parte das leis europeias são adotadas por esse processo. 

O Tratado de Maastricht (1992) introduziu o mecanismo de codecisão, porém ainda num escopo limitado de temas. O Tratado de Amsterdã (1999) ampliou e reforçou a eficácia do mecanismo. Em 2009, com a entrada em vigor do Tratado de Lisboa, a codecisão passou a ser denominada de \acrshort{plo}, tornando-se, assim, o principal processo legislativo adotado pela \acrshort{ue}. A figura \ref{fig:plo} resume o principal rito legislativo da \acrshort{ue}.

No \acrshort{plo}, a posição do Parlamento é encaminhada ao Conselho. Se o texto for aprovado sem alterações, a proposta é adotada. Caso, entretanto, haja alterações por parte do Conselho, o texto é reencaminhado para o \acrshort{pe} para segunda leitura. O Parlamento examina as alterações e pode aprová-las, rejeitá-las, ou propor novas alterações. No primeiro caso, a proposta legislativa é adotada após aprovação em sessão plenária. Se, contudo, as alterações do Conselho forem rejeitadas, a proposta é arquivada. No terceiro caso, o texto alterado é reencaminhado para o Conselho para segunda leitura.

Na segunda leitura do conselho, se o Conselho aprovar todas as alterações do \acrshort{pe}, o texto é, então aprovado. Se, todavia, não aprová-las, convoca-se o Comitê de Conciliação. Tal comitê é composto por igual número de deputados e representantes do Conselho. Seus membros buscam chegar a um texto aceito por ambos. Se a tentativa for fracassada, o texto é arquivado. Se houver sucesso, o texto é reencaminhado para o Parlamento e para o Conselho para terceira leitura. Em terceira leitura, nem o Conselho, nem o Parlamento podem propor alterações. Devem, portanto, ou rejeitar, ou aprovar o texto como foi enviado pelo Comitê de Conciliação.


O \acrshort{pe} também possui participação em atos não-legislativos, chamado de "aprovação", introduzido pelo Ato Único Europeu (1986). O mecanismo aplica-se em acordos de associação e de adesão à \acrshort{ue}, em alguns acordos comerciais, em casos de violações graves aos direitos fundamentais (art. 7º, \acrshort{tfue}). A aprovação também pode ser utilizada como ato legislativo em casos de legislação sobre combate à discriminação.

Há outros procedimentos cuja participação do \acrshort{pe} é destacada: parecer nos termos do artigo 140.º do \acrshort{tfue} (União Monetária), procedimentos relativos ao diálogo social (art. 154º, \acrshort{tfue}), à apreciação de acordos voluntários (art. 48º, \acrshort{tfue})), à codificação (art. 46º do Regimento), à medidas de execução e disposições delegadas.

No âmbito da União Monetária (art. 140.º do \acrshort{tfue}), o Parlamento emite um parecer sobre os progressos dos Estados-Membros em relação à adoção da moeda única, embora o Conselho tome a decisão final.

Nos procedimentos relativos ao diálogo social (artigos 154.º e 155.º do \acrshort{tfue}), o Parlamento atua recepcionando a consulta realizada pela Comissão aos parceiros sociais e avaliando acordos e convenções. 

Adicionalmente, o Parlamento é informado sobre acordos voluntários propostos pela Comissão como alternativa a medidas legislativas (artigo 48.º do Regimento) e pode apresentar uma resolução recomendando sua aprovação ou rejeição.

Quanto à codificação, que visa consolidar atos legislativos em um único texto para maior clareza (artigo 46.º do Regimento), a Comissão de Assuntos Jurídicos do Parlamento fica a cargo dessa consolidação.

O Parlamento também exerce controle sobre medidas de execução e disposições delegadas adotadas pela Comissão, podendo se opor a medidas que não estejam em conformidade com a legislação vigente ou que violem os princípios da subsidiariedade e proporcionalidade.

Embora a iniciativa legislativa na \acrshort{ue} seja de competência da \acrshort{ce}, o Parlamento também possui um direito de iniciativa, garantido pelo Tratado de Maastricht e consolidado no Tratado de Lisboa.

Esse direito permite ao Parlamento solicitar à Comissão a apresentação de propostas legislativas em áreas específicas, mediante aprovação da maioria de seus membros e com base em um relatório elaborado pela comissão parlamentar competente (artigo 225º do \acrshort{tfue}). O Parlamento pode, inclusive, estabelecer um prazo para a apresentação da proposta, mas a Comissão tem a prerrogativa de aceitar ou recusar o pedido.

Além disso, deputados individuais também podem apresentar propostas de atos da União com base nesse direito de iniciativa. A proposta é encaminhada ao Presidente do Parlamento, que a transmite à comissão competente para análise e possível apresentação à sessão plenária.

O Parlamento Europeu também exerce sua iniciativa através da elaboração de relatórios de iniciativa pelas comissões parlamentares, que podem apresentar propostas de resolução sobre assuntos de sua competência, desde que autorizados pela Conferência dos Presidentes (artigos 37, 46 e 52 do Regimento e artigo 17(1) do \acrshort{tfue}).

A participação do Parlamento na programação anual e plurianual da \acrshort{ue} também é relevante. O Parlamento coopera com a Comissão na elaboração do programa de trabalho, que define as prioridades da \acrshort{ue} para o período. Após a adoção do programa pela Comissão, um diálogo tripartido (trílogo) entre o Parlamento, o Conselho e a Comissão busca um acordo sobre a programação, conforme detalhado no anexo XIV do Regimento.

Em resumo, o direito de iniciativa do \acrshort{pe}, seja através de solicitações à Comissão, propostas de deputados individuais ou relatórios de iniciativa, fortalece seu papel no processo legislativo da \acrshort{ue}, complementando a iniciativa da Comissão.

\section{Poderes Orçamentários}

Com a entrada em vigor do Tratado de Lisboa, o \acrshort{pe} adquiriu um papel crucial na elaboração e aprovação do orçamento anual global da \acrshort{ue}, não só em colaboração com o Conselho, mas também com capacidade decisória na matéria.


O processo orçamentário anual inicia-se com a elaboração das previsões de receitas e despesas por todas as instituições da \acrshort{ue}. A \acrshort{ce}, com base nessas previsões, consolida e apresenta um projeto de orçamento ao Parlamento e ao Conselho. O Conselho, por sua vez, adota uma posição sobre o projeto e a encaminha ao Parlamento, que dispõe de um prazo para aprová-la ou emendá-la.

Durante as discussões no \acrshort{pe}, as comissões parlamentares debatem o projeto de orçamento e apresentam os seus pareceres à Comissão dos Orçamentos, responsável pela preparação da posição do Parlamento. A decisão do parlamento é tomada por maioria absoluta dos membros, podendo aceitar o parecer da \acrshort{ce}, propor alterações, ou aceitar tacitamente - caso não decida dentro de 42 dias.


Caso o Parlamento opte por alterar o projeto, um Comitê de Conciliação, composto por representantes de ambas as instituições, é convocado para buscar um consenso. Se um acordo for alcançado, o projeto comum é submetido à aprovação do Parlamento e do Conselho. Na ausência de consenso, a Comissão apresenta um novo projeto de orçamento, reiniciando o ciclo.

Vale mencionar que as decisões do Parlamento e do Conselho no que diz respeito às receitas e às despesas devem respeitar os limites das despesas da \acrshort{ue}, definidas no Quadro Financeiro Plurianual, negociado uma vez a cada sete anos.

% despesas anuais fixadas na programação financeira da \acrshort{ue}

Além da co-decisão na elaboração do orçamento anual, o \acrshort{pe} exerce um papel importante no controle da execução orçamentária. A \acrshort{ce}, responsável pela gestão do orçamento, está sujeita ao escrutínio do Parlamento, que avalia a conformidade das ações com as diretrizes estabelecidas e a eficácia na utilização dos recursos.

O Parlamento detém a prerrogativa de conceder ou recusar a quitação à Comissão, ou seja, a aprovação final das contas do exercício. Essa decisão é tomada após uma análise minuciosa dos relatórios financeiros e das atividades da Comissão, incluindo a avaliação do Tribunal de Contas Europeu.

Em suma, o \acrshort{pe} desempenha um papel ativo e influente no processo orçamentário da \acrshort{ue}, compartilhando com o Conselho a responsabilidade pela definição das prioridades de gastos e exercendo um controle rigoroso sobre a execução do orçamento, garantindo a transparência e a \textit{accountability} na gestão dos recursos públicos.

\section{Poderes Fiscalizatórios}

O \acrshort{pe} possui um amplo espectro de poderes fiscalizatórios sobre as principais instituições da \acrshort{ue}, buscando assegurar a \textit{accountability} e a transparência no funcionamento do bloco.

No âmbito do \acrshort{coue}, o \acrshort{pe} assegura sua influência através da participação do seu presidente nas reuniões do Conselho, onde este apresenta a posição do Parlamento sobre os temas em discussão. Além disso, o presidente do \acrshort{coue} reporta os resultados das reuniões ao \acrshort{pe}, permitindo um acompanhamento contínuo das decisões tomadas pelos chefes de Estado e de governo.

A interação com o Conselho da \acrshort{ue} também é fundamental para o exercício do controle parlamentar. O Parlamento debate o programa de cada presidência semestral do Conselho, apresenta perguntas e solicita informações sobre políticas específicas. O Alto Representante para os Negócios Estrangeiros e a Política de Segurança também presta contas ao Parlamento, apresentando relatórios periódicos sobre as ações da UE em matéria de política externa e de segurança.
 
O \acrshort{pe} exerce um controle crucial sobre a \acrshort{ce}, com a prerrogativa de aprovar ou destituir seus membros. Os comissários designados passam por audiências no Parlamento, onde são questionados sobre suas qualificações e planos de ação. A Comissão também apresenta relatórios regulares ao Parlamento, incluindo um relatório anual sobre as atividades da \acrshort{ue} e a execução do orçamento, garantindo a transparência e a prestação de contas. Além disso, o \acrshort{pe} pode apresentar monções de censura à Comissão, com a possibilidade de destituí-la em última instância. 

O Parlamento também exerce controle sobre outras instituições, como o \acrfull{tj}, o \acrfull{bce} e o \acrfull{tce}. O Parlamento pode solicitar ao \acrshort{tj} que tome medidas contra a Comissão ou o Conselho em caso de violação da legislação da \acrshort{ue}. No caso do \acrshort{bce}, o Parlamento aprova a nomeação dos seus principais dirigentes e recebe relatórios periódicos sobre a política monetária da zona euro. O Parlamento também utiliza os relatórios do \acrshort{tce} para avaliar a execução do orçamento da \acrshort{ue} e decidir sobre a concessão da quitação à Comissão.

Além disso, o \acrshort{pe} recebe petições de cidadãos da \acrshort{ue} e pode estabelecer comissões de inquérito para investigar alegações de violações ou má administração na aplicação do direito da \acrshort{ue}. Essas ferramentas permitem que o Parlamento atue como um canal de comunicação entre os cidadãos e as instituições europeias, garantindo que suas preocupações sejam ouvidas e consideradas.

Os poderes fiscalizatórios do \acrshort{pe}, portanto,  são essenciais para o funcionamento democrático e transparente da \acrshort{ue}. Através do exercício desses poderes, o Parlamento busca assegurar que as instituições da \acrshort{ue} atuem em conformidade com os princípios do Estado de Direito e que os interesses dos cidadãos europeus sejam protegidos e promovidos.


        % \section{Poderes Legislativos}
\label{section:podereslegislativos}

O \acrshort{pe} possui um papel crucial no processo legislativo da \acrshort{ue}, atuando em conjunto com o \acrshort{coue} na aprovação da maioria das leis europeias, de acordo com o processo legislativo ordinário (artigo 289 do \acrshort{tfue}). O trabalho legislativo se inicia com a apresentação de um "texto legislativo" pela \acrshort{ce}, que detém o monopólio da iniciativa normativa. Em seguida, a proposta é analisada por um parlamentar no âmbito de uma comissão parlamentar, que elabora um relatório, sob a liderança de um relator designado. A escolha da comissão, normalmente, ocorre por decisão dos coordenadores de comissão. 

Cada proposta legislativa é confiada a um grupo político, o qual designa um "relator" para elaborar o relatório em nome da comissão. Demais grupos políticos podem designar "relatores-sombra" para coordenar as respectivas posições sobre o assunto em debate.

Após votação e possíveis alterações no âmbito da comissão, o relatório é submetido ao plenário do Parlamento para aprovação. O presidente da comissão preside as suas reuniões e dos seus coordenadores. Além disso, o presidente possui ingerência no processo de votação, bem como nas regras relativas à admissibilidade das alterações.

Os coordenadores da comissão são designados pelos grupos políticos. Os coordenadores muitas vezes se reúnem a portas fechadas (\textit{in camera}) às margens das reuniões das comissões. A comissão pode delegar aos coordenadores o poder de alocação dos relatórios e opiniões aos grupos, de decisão sobre audiências públicas na comissão, de pedidos de estudos, entre outras atividades relativas à organização dos trabalhos da comissão.

As comissões parlamentares e seus membros possuem o apoio administrativo pelos secretários de comissões. Esses funcionários organizam as reuniões da comissão, planejam e dão suporte e assessoria a respeito dos assuntos da comissão). Além dos secretários, há os assessores dos grupos políticos, os quais dão suporte e assessoria tanto para o coordenador do seu grupo quanto para os membros individualmente). Outros membros e órgãos de apoios são: os assessores dos membros da comissão, a \acrfull{ual}, o Serviço Jurídico, o Diretório para Atos Legislativos, os Departamentos de Políticas, o Serviço de Pesquisa do Parlamento Europeu, a assessoria de imprensa do \acrshort{pe} e os Diretórios-Gerais para Tradução e Interpretação. Vale mencionar que na terceira leitura, a \acrshort{ual} coordena a assistência administrativa para as delegações do Parlamento no comitê de conciliação.

Aprovado o texto na comissão, ele é encaminhado para o Plenário. Caso obtenha aprovação em sessão plenária, a posição do \acrshort{pe} terá sido adotada. Esse processo pode se repetir uma ou mais vezes a depender do tipo de procedimento e do acordo, ou não, alcançado com o Conselho.

No que diz respeito à adoção de atos legislativos, há dois procedimentos: o \acrfull{plo}, ou também chamado de codecisão; e os processos legislativos especiais, aplicados a casos específicos nos quais o Parlamento possui apenas papel consultivo.

Sobre os últimos, o Conselho atua como único legislador. Ele ocorre em questões definidas nos tratados da \acrshort{ue}, como por exemplo, em temas fiscais. Em tais casos, o \acrshort{pe} emite apenas um parecer consultivo. Em casos específicos, porém, quando assim determinado, o parecer consultivo é obrigatório. A matéria só poderá ter força de lei quando o \acrshort{pe} tiver emitido seu parecer (art. 311º, \acrshort{tfue}).

\begin{figure}
    \caption{Fluxograma resumido do Processo Legislativo Ordinário}
    \includegraphics[width=\textwidth]{imgs/Processo legislativo ordinário.drawio.png}
    \label{fig:plo}
    \centering
    \caption*{Fonte: o autor (2025)}
\end{figure}


No \acrshort{plo}, o Parlamento e o Conselho atuam como colegisladores. Há uma gama vasta de temas em que o processo de codecisão é demandado: governança econômica, imigração, energia, transportes, meio ambiente, proteção dos consumidores, entre outros. Atualmente, a maior parte das leis europeias são adotadas por esse processo. 

O Tratado de Maastricht (1992) introduziu o mecanismo de codecisão, porém ainda num escopo limitado de temas. O Tratado de Amsterdã (1999) ampliou e reforçou a eficácia do mecanismo. Em 2009, com a entrada em vigor do Tratado de Lisboa, a codecisão passou a ser denominada de \acrshort{plo}, tornando-se, assim, o principal processo legislativo adotado pela \acrshort{ue}. A figura \ref{fig:plo} resume o principal rito legislativo da \acrshort{ue}.

No \acrshort{plo}, a posição do Parlamento é encaminhada ao Conselho. Se o texto for aprovado sem alterações, a proposta é adotada. Caso, entretanto, haja alterações por parte do Conselho, o texto é reencaminhado para o \acrshort{pe} para segunda leitura. O Parlamento examina as alterações e pode aprová-las, rejeitá-las, ou propor novas alterações. No primeiro caso, a proposta legislativa é adotada após aprovação em sessão plenária. Se, contudo, as alterações do Conselho forem rejeitadas, a proposta é arquivada. No terceiro caso, o texto alterado é reencaminhado para o Conselho para segunda leitura.

Na segunda leitura do conselho, se o Conselho aprovar todas as alterações do \acrshort{pe}, o texto é, então aprovado. Se, todavia, não aprová-las, convoca-se o Comitê de Conciliação. Tal comitê é composto por igual número de deputados e representantes do Conselho. Seus membros buscam chegar a um texto aceito por ambos. Se a tentativa for fracassada, o texto é arquivado. Se houver sucesso, o texto é reencaminhado para o Parlamento e para o Conselho para terceira leitura. Em terceira leitura, nem o Conselho, nem o Parlamento podem propor alterações. Devem, portanto, ou rejeitar, ou aprovar o texto como foi enviado pelo Comitê de Conciliação.


O \acrshort{pe} também possui participação em atos não-legislativos, chamado de "aprovação", introduzido pelo Ato Único Europeu (1986). O mecanismo aplica-se em acordos de associação e de adesão à \acrshort{ue}, em alguns acordos comerciais, em casos de violações graves aos direitos fundamentais (art. 7º, \acrshort{tfue}). A aprovação também pode ser utilizada como ato legislativo em casos de legislação sobre combate à discriminação.

Há outros procedimentos cuja participação do \acrshort{pe} é destacada: parecer nos termos do artigo 140.º do \acrshort{tfue} (União Monetária), procedimentos relativos ao diálogo social (art. 154º, \acrshort{tfue}), à apreciação de acordos voluntários (art. 48º, \acrshort{tfue})), à codificação (art. 46º do Regimento), à medidas de execução e disposições delegadas.

No âmbito da União Monetária (art. 140.º do \acrshort{tfue}), o Parlamento emite um parecer sobre os progressos dos Estados-Membros em relação à adoção da moeda única, embora o Conselho tome a decisão final.

Nos procedimentos relativos ao diálogo social (artigos 154.º e 155.º do \acrshort{tfue}), o Parlamento atua recepcionando a consulta realizada pela Comissão aos parceiros sociais e avaliando acordos e convenções. 

Adicionalmente, o Parlamento é informado sobre acordos voluntários propostos pela Comissão como alternativa a medidas legislativas (artigo 48.º do Regimento) e pode apresentar uma resolução recomendando sua aprovação ou rejeição.

Quanto à codificação, que visa consolidar atos legislativos em um único texto para maior clareza (artigo 46.º do Regimento), a Comissão de Assuntos Jurídicos do Parlamento fica a cargo dessa consolidação.

O Parlamento também exerce controle sobre medidas de execução e disposições delegadas adotadas pela Comissão, podendo se opor a medidas que não estejam em conformidade com a legislação vigente ou que violem os princípios da subsidiariedade e proporcionalidade.

Embora a iniciativa legislativa na \acrshort{ue} seja de competência da \acrshort{ce}, o Parlamento também possui um direito de iniciativa, garantido pelo Tratado de Maastricht e consolidado no Tratado de Lisboa.

Esse direito permite ao Parlamento solicitar à Comissão a apresentação de propostas legislativas em áreas específicas, mediante aprovação da maioria de seus membros e com base em um relatório elaborado pela comissão parlamentar competente (artigo 225º do \acrshort{tfue}). O Parlamento pode, inclusive, estabelecer um prazo para a apresentação da proposta, mas a Comissão tem a prerrogativa de aceitar ou recusar o pedido.

Além disso, deputados individuais também podem apresentar propostas de atos da União com base nesse direito de iniciativa. A proposta é encaminhada ao Presidente do Parlamento, que a transmite à comissão competente para análise e possível apresentação à sessão plenária.

O Parlamento Europeu também exerce sua iniciativa através da elaboração de relatórios de iniciativa pelas comissões parlamentares, que podem apresentar propostas de resolução sobre assuntos de sua competência, desde que autorizados pela Conferência dos Presidentes (artigos 37, 46 e 52 do Regimento e artigo 17(1) do \acrshort{tfue}).

A participação do Parlamento na programação anual e plurianual da \acrshort{ue} também é relevante. O Parlamento coopera com a Comissão na elaboração do programa de trabalho, que define as prioridades da \acrshort{ue} para o período. Após a adoção do programa pela Comissão, um diálogo tripartido (trílogo) entre o Parlamento, o Conselho e a Comissão busca um acordo sobre a programação, conforme detalhado no anexo XIV do Regimento.

Em resumo, o direito de iniciativa do \acrshort{pe}, seja através de solicitações à Comissão, propostas de deputados individuais ou relatórios de iniciativa, fortalece seu papel no processo legislativo da \acrshort{ue}, complementando a iniciativa da Comissão.

\section{Poderes Orçamentários}

Com a entrada em vigor do Tratado de Lisboa, o \acrshort{pe} adquiriu um papel crucial na elaboração e aprovação do orçamento anual global da \acrshort{ue}, não só em colaboração com o Conselho, mas também com capacidade decisória na matéria.


O processo orçamentário anual inicia-se com a elaboração das previsões de receitas e despesas por todas as instituições da \acrshort{ue}. A \acrshort{ce}, com base nessas previsões, consolida e apresenta um projeto de orçamento ao Parlamento e ao Conselho. O Conselho, por sua vez, adota uma posição sobre o projeto e a encaminha ao Parlamento, que dispõe de um prazo para aprová-la ou emendá-la.

Durante as discussões no \acrshort{pe}, as comissões parlamentares debatem o projeto de orçamento e apresentam os seus pareceres à Comissão dos Orçamentos, responsável pela preparação da posição do Parlamento. A decisão do parlamento é tomada por maioria absoluta dos membros, podendo aceitar o parecer da \acrshort{ce}, propor alterações, ou aceitar tacitamente - caso não decida dentro de 42 dias.


Caso o Parlamento opte por alterar o projeto, um Comitê de Conciliação, composto por representantes de ambas as instituições, é convocado para buscar um consenso. Se um acordo for alcançado, o projeto comum é submetido à aprovação do Parlamento e do Conselho. Na ausência de consenso, a Comissão apresenta um novo projeto de orçamento, reiniciando o ciclo.

Vale mencionar que as decisões do Parlamento e do Conselho no que diz respeito às receitas e às despesas devem respeitar os limites das despesas da \acrshort{ue}, definidas no Quadro Financeiro Plurianual, negociado uma vez a cada sete anos.

% despesas anuais fixadas na programação financeira da \acrshort{ue}

Além da co-decisão na elaboração do orçamento anual, o \acrshort{pe} exerce um papel importante no controle da execução orçamentária. A \acrshort{ce}, responsável pela gestão do orçamento, está sujeita ao escrutínio do Parlamento, que avalia a conformidade das ações com as diretrizes estabelecidas e a eficácia na utilização dos recursos.

O Parlamento detém a prerrogativa de conceder ou recusar a quitação à Comissão, ou seja, a aprovação final das contas do exercício. Essa decisão é tomada após uma análise minuciosa dos relatórios financeiros e das atividades da Comissão, incluindo a avaliação do Tribunal de Contas Europeu.

Em suma, o \acrshort{pe} desempenha um papel ativo e influente no processo orçamentário da \acrshort{ue}, compartilhando com o Conselho a responsabilidade pela definição das prioridades de gastos e exercendo um controle rigoroso sobre a execução do orçamento, garantindo a transparência e a \textit{accountability} na gestão dos recursos públicos.

\section{Poderes Fiscalizatórios}

O \acrshort{pe} possui um amplo espectro de poderes fiscalizatórios sobre as principais instituições da \acrshort{ue}, buscando assegurar a \textit{accountability} e a transparência no funcionamento do bloco.

No âmbito do \acrshort{coue}, o \acrshort{pe} assegura sua influência através da participação do seu presidente nas reuniões do Conselho, onde este apresenta a posição do Parlamento sobre os temas em discussão. Além disso, o presidente do \acrshort{coue} reporta os resultados das reuniões ao \acrshort{pe}, permitindo um acompanhamento contínuo das decisões tomadas pelos chefes de Estado e de governo.

A interação com o Conselho da \acrshort{ue} também é fundamental para o exercício do controle parlamentar. O Parlamento debate o programa de cada presidência semestral do Conselho, apresenta perguntas e solicita informações sobre políticas específicas. O Alto Representante para os Negócios Estrangeiros e a Política de Segurança também presta contas ao Parlamento, apresentando relatórios periódicos sobre as ações da UE em matéria de política externa e de segurança.
 
O \acrshort{pe} exerce um controle crucial sobre a \acrshort{ce}, com a prerrogativa de aprovar ou destituir seus membros. Os comissários designados passam por audiências no Parlamento, onde são questionados sobre suas qualificações e planos de ação. A Comissão também apresenta relatórios regulares ao Parlamento, incluindo um relatório anual sobre as atividades da \acrshort{ue} e a execução do orçamento, garantindo a transparência e a prestação de contas. Além disso, o \acrshort{pe} pode apresentar monções de censura à Comissão, com a possibilidade de destituí-la em última instância. 

O Parlamento também exerce controle sobre outras instituições, como o \acrfull{tj}, o \acrfull{bce} e o \acrfull{tce}. O Parlamento pode solicitar ao \acrshort{tj} que tome medidas contra a Comissão ou o Conselho em caso de violação da legislação da \acrshort{ue}. No caso do \acrshort{bce}, o Parlamento aprova a nomeação dos seus principais dirigentes e recebe relatórios periódicos sobre a política monetária da zona euro. O Parlamento também utiliza os relatórios do \acrshort{tce} para avaliar a execução do orçamento da \acrshort{ue} e decidir sobre a concessão da quitação à Comissão.

Além disso, o \acrshort{pe} recebe petições de cidadãos da \acrshort{ue} e pode estabelecer comissões de inquérito para investigar alegações de violações ou má administração na aplicação do direito da \acrshort{ue}. Essas ferramentas permitem que o Parlamento atue como um canal de comunicação entre os cidadãos e as instituições europeias, garantindo que suas preocupações sejam ouvidas e consideradas.

Os poderes fiscalizatórios do \acrshort{pe}, portanto,  são essenciais para o funcionamento democrático e transparente da \acrshort{ue}. Através do exercício desses poderes, o Parlamento busca assegurar que as instituições da \acrshort{ue} atuem em conformidade com os princípios do Estado de Direito e que os interesses dos cidadãos europeus sejam protegidos e promovidos.


        \section{Poderes Legislativos}
\label{section:podereslegislativos}

O \acrshort{pe} possui um papel crucial no processo legislativo da \acrshort{ue}, atuando em conjunto com o \acrshort{coue} na aprovação da maioria das leis europeias, de acordo com o processo legislativo ordinário (artigo 289 do \acrshort{tfue}). O trabalho legislativo se inicia com a apresentação de um "texto legislativo" pela \acrshort{ce}, que detém o monopólio da iniciativa normativa. Em seguida, a proposta é analisada por um parlamentar no âmbito de uma comissão parlamentar, que elabora um relatório, sob a liderança de um relator designado. A escolha da comissão, normalmente, ocorre por decisão dos coordenadores de comissão. 

Cada proposta legislativa é confiada a um grupo político, o qual designa um "relator" para elaborar o relatório em nome da comissão. Demais grupos políticos podem designar "relatores-sombra" para coordenar as respectivas posições sobre o assunto em debate.

Após votação e possíveis alterações no âmbito da comissão, o relatório é submetido ao plenário do Parlamento para aprovação. O presidente da comissão preside as suas reuniões e dos seus coordenadores. Além disso, o presidente possui ingerência no processo de votação, bem como nas regras relativas à admissibilidade das alterações.

Os coordenadores da comissão são designados pelos grupos políticos. Os coordenadores muitas vezes se reúnem a portas fechadas (\textit{in camera}) às margens das reuniões das comissões. A comissão pode delegar aos coordenadores o poder de alocação dos relatórios e opiniões aos grupos, de decisão sobre audiências públicas na comissão, de pedidos de estudos, entre outras atividades relativas à organização dos trabalhos da comissão.

As comissões parlamentares e seus membros possuem o apoio administrativo pelos secretários de comissões. Esses funcionários organizam as reuniões da comissão, planejam e dão suporte e assessoria a respeito dos assuntos da comissão). Além dos secretários, há os assessores dos grupos políticos, os quais dão suporte e assessoria tanto para o coordenador do seu grupo quanto para os membros individualmente). Outros membros e órgãos de apoios são: os assessores dos membros da comissão, a \acrfull{ual}, o Serviço Jurídico, o Diretório para Atos Legislativos, os Departamentos de Políticas, o Serviço de Pesquisa do Parlamento Europeu, a assessoria de imprensa do \acrshort{pe} e os Diretórios-Gerais para Tradução e Interpretação. Vale mencionar que na terceira leitura, a \acrshort{ual} coordena a assistência administrativa para as delegações do Parlamento no comitê de conciliação.

Aprovado o texto na comissão, ele é encaminhado para o Plenário. Caso obtenha aprovação em sessão plenária, a posição do \acrshort{pe} terá sido adotada. Esse processo pode se repetir uma ou mais vezes a depender do tipo de procedimento e do acordo, ou não, alcançado com o Conselho.

No que diz respeito à adoção de atos legislativos, há dois procedimentos: o \acrfull{plo}, ou também chamado de codecisão; e os processos legislativos especiais, aplicados a casos específicos nos quais o Parlamento possui apenas papel consultivo.

Sobre os últimos, o Conselho atua como único legislador. Ele ocorre em questões definidas nos tratados da \acrshort{ue}, como por exemplo, em temas fiscais. Em tais casos, o \acrshort{pe} emite apenas um parecer consultivo. Em casos específicos, porém, quando assim determinado, o parecer consultivo é obrigatório. A matéria só poderá ter força de lei quando o \acrshort{pe} tiver emitido seu parecer (art. 311º, \acrshort{tfue}).

\begin{figure}
    \caption{Fluxograma resumido do Processo Legislativo Ordinário}
    \includegraphics[width=\textwidth]{imgs/Processo legislativo ordinário.drawio.png}
    \label{fig:plo}
    \centering
    \caption*{Fonte: o autor (2025)}
\end{figure}


No \acrshort{plo}, o Parlamento e o Conselho atuam como colegisladores. Há uma gama vasta de temas em que o processo de codecisão é demandado: governança econômica, imigração, energia, transportes, meio ambiente, proteção dos consumidores, entre outros. Atualmente, a maior parte das leis europeias são adotadas por esse processo. 

O Tratado de Maastricht (1992) introduziu o mecanismo de codecisão, porém ainda num escopo limitado de temas. O Tratado de Amsterdã (1999) ampliou e reforçou a eficácia do mecanismo. Em 2009, com a entrada em vigor do Tratado de Lisboa, a codecisão passou a ser denominada de \acrshort{plo}, tornando-se, assim, o principal processo legislativo adotado pela \acrshort{ue}. A figura \ref{fig:plo} resume o principal rito legislativo da \acrshort{ue}.

No \acrshort{plo}, a posição do Parlamento é encaminhada ao Conselho. Se o texto for aprovado sem alterações, a proposta é adotada. Caso, entretanto, haja alterações por parte do Conselho, o texto é reencaminhado para o \acrshort{pe} para segunda leitura. O Parlamento examina as alterações e pode aprová-las, rejeitá-las, ou propor novas alterações. No primeiro caso, a proposta legislativa é adotada após aprovação em sessão plenária. Se, contudo, as alterações do Conselho forem rejeitadas, a proposta é arquivada. No terceiro caso, o texto alterado é reencaminhado para o Conselho para segunda leitura.

Na segunda leitura do conselho, se o Conselho aprovar todas as alterações do \acrshort{pe}, o texto é, então aprovado. Se, todavia, não aprová-las, convoca-se o Comitê de Conciliação. Tal comitê é composto por igual número de deputados e representantes do Conselho. Seus membros buscam chegar a um texto aceito por ambos. Se a tentativa for fracassada, o texto é arquivado. Se houver sucesso, o texto é reencaminhado para o Parlamento e para o Conselho para terceira leitura. Em terceira leitura, nem o Conselho, nem o Parlamento podem propor alterações. Devem, portanto, ou rejeitar, ou aprovar o texto como foi enviado pelo Comitê de Conciliação.


O \acrshort{pe} também possui participação em atos não-legislativos, chamado de "aprovação", introduzido pelo Ato Único Europeu (1986). O mecanismo aplica-se em acordos de associação e de adesão à \acrshort{ue}, em alguns acordos comerciais, em casos de violações graves aos direitos fundamentais (art. 7º, \acrshort{tfue}). A aprovação também pode ser utilizada como ato legislativo em casos de legislação sobre combate à discriminação.

Há outros procedimentos cuja participação do \acrshort{pe} é destacada: parecer nos termos do artigo 140.º do \acrshort{tfue} (União Monetária), procedimentos relativos ao diálogo social (art. 154º, \acrshort{tfue}), à apreciação de acordos voluntários (art. 48º, \acrshort{tfue})), à codificação (art. 46º do Regimento), à medidas de execução e disposições delegadas.

No âmbito da União Monetária (art. 140.º do \acrshort{tfue}), o Parlamento emite um parecer sobre os progressos dos Estados-Membros em relação à adoção da moeda única, embora o Conselho tome a decisão final.

Nos procedimentos relativos ao diálogo social (artigos 154.º e 155.º do \acrshort{tfue}), o Parlamento atua recepcionando a consulta realizada pela Comissão aos parceiros sociais e avaliando acordos e convenções. 

Adicionalmente, o Parlamento é informado sobre acordos voluntários propostos pela Comissão como alternativa a medidas legislativas (artigo 48.º do Regimento) e pode apresentar uma resolução recomendando sua aprovação ou rejeição.

Quanto à codificação, que visa consolidar atos legislativos em um único texto para maior clareza (artigo 46.º do Regimento), a Comissão de Assuntos Jurídicos do Parlamento fica a cargo dessa consolidação.

O Parlamento também exerce controle sobre medidas de execução e disposições delegadas adotadas pela Comissão, podendo se opor a medidas que não estejam em conformidade com a legislação vigente ou que violem os princípios da subsidiariedade e proporcionalidade.

Embora a iniciativa legislativa na \acrshort{ue} seja de competência da \acrshort{ce}, o Parlamento também possui um direito de iniciativa, garantido pelo Tratado de Maastricht e consolidado no Tratado de Lisboa.

Esse direito permite ao Parlamento solicitar à Comissão a apresentação de propostas legislativas em áreas específicas, mediante aprovação da maioria de seus membros e com base em um relatório elaborado pela comissão parlamentar competente (artigo 225º do \acrshort{tfue}). O Parlamento pode, inclusive, estabelecer um prazo para a apresentação da proposta, mas a Comissão tem a prerrogativa de aceitar ou recusar o pedido.

Além disso, deputados individuais também podem apresentar propostas de atos da União com base nesse direito de iniciativa. A proposta é encaminhada ao Presidente do Parlamento, que a transmite à comissão competente para análise e possível apresentação à sessão plenária.

O Parlamento Europeu também exerce sua iniciativa através da elaboração de relatórios de iniciativa pelas comissões parlamentares, que podem apresentar propostas de resolução sobre assuntos de sua competência, desde que autorizados pela Conferência dos Presidentes (artigos 37, 46 e 52 do Regimento e artigo 17(1) do \acrshort{tfue}).

A participação do Parlamento na programação anual e plurianual da \acrshort{ue} também é relevante. O Parlamento coopera com a Comissão na elaboração do programa de trabalho, que define as prioridades da \acrshort{ue} para o período. Após a adoção do programa pela Comissão, um diálogo tripartido (trílogo) entre o Parlamento, o Conselho e a Comissão busca um acordo sobre a programação, conforme detalhado no anexo XIV do Regimento.

Em resumo, o direito de iniciativa do \acrshort{pe}, seja através de solicitações à Comissão, propostas de deputados individuais ou relatórios de iniciativa, fortalece seu papel no processo legislativo da \acrshort{ue}, complementando a iniciativa da Comissão.

\section{Poderes Orçamentários}

Com a entrada em vigor do Tratado de Lisboa, o \acrshort{pe} adquiriu um papel crucial na elaboração e aprovação do orçamento anual global da \acrshort{ue}, não só em colaboração com o Conselho, mas também com capacidade decisória na matéria.


O processo orçamentário anual inicia-se com a elaboração das previsões de receitas e despesas por todas as instituições da \acrshort{ue}. A \acrshort{ce}, com base nessas previsões, consolida e apresenta um projeto de orçamento ao Parlamento e ao Conselho. O Conselho, por sua vez, adota uma posição sobre o projeto e a encaminha ao Parlamento, que dispõe de um prazo para aprová-la ou emendá-la.

Durante as discussões no \acrshort{pe}, as comissões parlamentares debatem o projeto de orçamento e apresentam os seus pareceres à Comissão dos Orçamentos, responsável pela preparação da posição do Parlamento. A decisão do parlamento é tomada por maioria absoluta dos membros, podendo aceitar o parecer da \acrshort{ce}, propor alterações, ou aceitar tacitamente - caso não decida dentro de 42 dias.


Caso o Parlamento opte por alterar o projeto, um Comitê de Conciliação, composto por representantes de ambas as instituições, é convocado para buscar um consenso. Se um acordo for alcançado, o projeto comum é submetido à aprovação do Parlamento e do Conselho. Na ausência de consenso, a Comissão apresenta um novo projeto de orçamento, reiniciando o ciclo.

Vale mencionar que as decisões do Parlamento e do Conselho no que diz respeito às receitas e às despesas devem respeitar os limites das despesas da \acrshort{ue}, definidas no Quadro Financeiro Plurianual, negociado uma vez a cada sete anos.

% despesas anuais fixadas na programação financeira da \acrshort{ue}

Além da co-decisão na elaboração do orçamento anual, o \acrshort{pe} exerce um papel importante no controle da execução orçamentária. A \acrshort{ce}, responsável pela gestão do orçamento, está sujeita ao escrutínio do Parlamento, que avalia a conformidade das ações com as diretrizes estabelecidas e a eficácia na utilização dos recursos.

O Parlamento detém a prerrogativa de conceder ou recusar a quitação à Comissão, ou seja, a aprovação final das contas do exercício. Essa decisão é tomada após uma análise minuciosa dos relatórios financeiros e das atividades da Comissão, incluindo a avaliação do Tribunal de Contas Europeu.

Em suma, o \acrshort{pe} desempenha um papel ativo e influente no processo orçamentário da \acrshort{ue}, compartilhando com o Conselho a responsabilidade pela definição das prioridades de gastos e exercendo um controle rigoroso sobre a execução do orçamento, garantindo a transparência e a \textit{accountability} na gestão dos recursos públicos.

\section{Poderes Fiscalizatórios}

O \acrshort{pe} possui um amplo espectro de poderes fiscalizatórios sobre as principais instituições da \acrshort{ue}, buscando assegurar a \textit{accountability} e a transparência no funcionamento do bloco.

No âmbito do \acrshort{coue}, o \acrshort{pe} assegura sua influência através da participação do seu presidente nas reuniões do Conselho, onde este apresenta a posição do Parlamento sobre os temas em discussão. Além disso, o presidente do \acrshort{coue} reporta os resultados das reuniões ao \acrshort{pe}, permitindo um acompanhamento contínuo das decisões tomadas pelos chefes de Estado e de governo.

A interação com o Conselho da \acrshort{ue} também é fundamental para o exercício do controle parlamentar. O Parlamento debate o programa de cada presidência semestral do Conselho, apresenta perguntas e solicita informações sobre políticas específicas. O Alto Representante para os Negócios Estrangeiros e a Política de Segurança também presta contas ao Parlamento, apresentando relatórios periódicos sobre as ações da UE em matéria de política externa e de segurança.
 
O \acrshort{pe} exerce um controle crucial sobre a \acrshort{ce}, com a prerrogativa de aprovar ou destituir seus membros. Os comissários designados passam por audiências no Parlamento, onde são questionados sobre suas qualificações e planos de ação. A Comissão também apresenta relatórios regulares ao Parlamento, incluindo um relatório anual sobre as atividades da \acrshort{ue} e a execução do orçamento, garantindo a transparência e a prestação de contas. Além disso, o \acrshort{pe} pode apresentar monções de censura à Comissão, com a possibilidade de destituí-la em última instância. 

O Parlamento também exerce controle sobre outras instituições, como o \acrfull{tj}, o \acrfull{bce} e o \acrfull{tce}. O Parlamento pode solicitar ao \acrshort{tj} que tome medidas contra a Comissão ou o Conselho em caso de violação da legislação da \acrshort{ue}. No caso do \acrshort{bce}, o Parlamento aprova a nomeação dos seus principais dirigentes e recebe relatórios periódicos sobre a política monetária da zona euro. O Parlamento também utiliza os relatórios do \acrshort{tce} para avaliar a execução do orçamento da \acrshort{ue} e decidir sobre a concessão da quitação à Comissão.

Além disso, o \acrshort{pe} recebe petições de cidadãos da \acrshort{ue} e pode estabelecer comissões de inquérito para investigar alegações de violações ou má administração na aplicação do direito da \acrshort{ue}. Essas ferramentas permitem que o Parlamento atue como um canal de comunicação entre os cidadãos e as instituições europeias, garantindo que suas preocupações sejam ouvidas e consideradas.

Os poderes fiscalizatórios do \acrshort{pe}, portanto,  são essenciais para o funcionamento democrático e transparente da \acrshort{ue}. Através do exercício desses poderes, o Parlamento busca assegurar que as instituições da \acrshort{ue} atuem em conformidade com os princípios do Estado de Direito e que os interesses dos cidadãos europeus sejam protegidos e promovidos.


    
    \postextual % a partir daqui são os elementos pós-textuais.
    \bibliographystyle{abntex2-alf}
    \bibliography{refs}

\end{document}
