\documentclass[aspectratio=169]{beamer}
\usetheme{Madrid}
\usecolortheme{default}
\setbeamertemplate{navigation symbols}{}
\setbeamertemplate{footline}[frame number]

% Pacotes necessários
\usepackage[utf8]{inputenc}
\usepackage[T1]{fontenc}
\usepackage[portuguese]{babel}
\usepackage{amsmath,amssymb,amsfonts}
\usepackage{graphicx}
\usepackage{booktabs}
\usepackage{multirow}
\usepackage{array}
\usepackage{siunitx}
\usepackage{acronym}
\usepackage{tabularx}

% Configurações de cores
\definecolor{myblue}{RGB}{31,119,180}
\definecolor{myred}{RGB}{214,39,40}
\definecolor{mygreen}{RGB}{44,160,44}

% Título da apresentação
\title{Efeitos do Lobbying na Atividade Parlamentar: \\ Modelagem PPML e Análise Causal}
\subtitle{Evidências do Parlamento Europeu (2019-2024)}
\author{Análise Empírica}
\institute{Estudo de Lobbying e Política Europeia}
\date{\today}

\begin{document}

% Slide de título
\begin{frame}
\titlepage
\end{frame}

% Sumário
\begin{frame}{Sumário}
\tableofcontents
\end{frame}

% ========================================
% SEÇÃO 1: INTRODUÇÃO E HIPÓTESES
% ========================================

\section{Introdução e Hipóteses}

\begin{frame}{Pergunta de Pesquisa e Hipóteses}
\begin{block}{Hipóteses a Testar}
\begin{enumerate}
\item \textbf{H1:} MEPs com maior pressão de lobby exibem maior atividade legislativa (AL) no domínio específico
\item \textbf{H2:} Lobbying empresarial é mais eficaz em aumentar a AL dos MEPs
\item \textbf{H3:} Em temas mais salientes, lobbying não-empresarial supera o empresarial na eficácia
\end{enumerate}
\end{block}

\vspace{0.5cm}

\begin{itemize}
\item \textbf{Unidade de análise:} MEP-domínio-mês
\item \textbf{Período:} Julho 2019 - Novembro 2024
\item \textbf{Amostra:} 1.353 deputados, 9 domínios, 63 meses
\item \textbf{Observações:} 767.151 (painel balanceado)
\end{itemize}
\end{frame}

\begin{frame}{Variáveis de Interesse}
\begin{columns}
\begin{column}{0.5\textwidth}
\textbf{Variável Dependente:}
\begin{itemize}
\item \textbf{Questions:} Número de perguntas parlamentares por MEP-domínio-mês
\item Proxy para Atividade Legislativa (AL)
\item Natureza: Contagem discreta
\item Característica: Alta incidência de zeros
\end{itemize}
\end{column}
\begin{column}{0.5\textwidth}
\textbf{Variáveis de Tratamento:}
\begin{itemize}
\item \textbf{Meetings:} Intensidade de lobbying (H1)
\item \textbf{Categorias:} Business vs. NGOs vs. Other (H2, H3)
\end{itemize}
\end{column}
\end{columns}
\end{frame}

% ========================================
% SEÇÃO 2: ANÁLISE DESCRITIVA
% ========================================

\section{Análise Descritiva}

\begin{frame}{Características Gerais dos Dados}
\begin{block}{Estrutura do Painel}
\begin{itemize}
\item \textbf{Taxa de tratamento:} 46,3\% dos deputados receberam lobbying
\item \textbf{Concentração:} Mediana = 105 reuniões, Média = 288,2 reuniões
\item \textbf{Especialização temática:} 97,6\% dos MEPs são generalistas (HHI < 0,4)
\item \textbf{Inflação de zeros:} 92,2\% (perguntas) e 92,5\% (reuniões) no nível MEP-domínio-mês
\end{itemize}
\end{block}

\vspace{0.3cm}

\textbf{Nota:} A inflação aparente de zeros é parcialmente artificial devido à especialização temática
\end{frame}

\begin{frame}{Evolução Temporal}
\begin{figure}
\centering
\includegraphics[width=0.9\textwidth]{figures/fig1_time_series_meetings_questions.pdf}
\caption{Evolução temporal da atividade parlamentar e de lobbying}
\end{figure}

\textbf{Padrões identificados:}
\begin{itemize}
\item Tendência crescente em ambas as variáveis
\item Sazonalidade relacionada ao calendário parlamentar
\item Picos de atividade coincidindo com discussões legislativas
\item Correlação contemporânea estável ao longo do tempo
\end{itemize}
\end{frame}

\begin{frame}{Distribuição do Tratamento}
\begin{columns}
\begin{column}{0.5\textwidth}
\begin{figure}
\centering
\includegraphics[width=\textwidth]{figures/fig2_proportion_meetings.pdf}
\caption{Proporção de MEPs com reuniões}
\end{figure}
\end{column}
\begin{column}{0.5\textwidth}
\begin{figure}
\centering
\includegraphics[width=\textwidth]{figures/fig3_correlation_meetings_questions.pdf}
\caption{Correlação temporal}
\end{figure}
\end{column}
\end{columns}
\end{frame}

\begin{frame}{Heterogeneidade por Domínio}
\begin{table}
\centering
\caption{Taxa de tratamento por domínio (\%)}
\begin{tabular}{lr}
\toprule
\textbf{Domínio} & \textbf{Taxa} \\
\midrule
Economia e Comércio & 45,5 \\
Tecnologia & 45,5 \\
Política Externa e Segurança & 45,2 \\
Infraestrutura e Indústria & 45,1 \\
Meio Ambiente e Clima & 44,9 \\
Saúde & 44,3 \\
Educação & 42,7 \\
Direitos Humanos & 41,7 \\
Agricultura & 40,9 \\
\bottomrule
\end{tabular}
\end{table}

\textbf{Padrão:} Domínios de regulação econômica apresentam maior atividade de lobbying
\end{frame}

\begin{frame}{Análise dos Lobistas}
\begin{columns}
\begin{column}{0.5\textwidth}
\textbf{Distribuição por Categoria:}
\begin{itemize}
\item \textbf{Business:} 33,7\%
\item \textbf{NGOs:} 32,8\%
\item \textbf{Outros:} 33,5\%
\end{itemize}

\textbf{Concentração Geográfica:}
\begin{itemize}
\item Bélgica: 18,2\%
\item Alemanha: 14,0\%
\item França: 9,3\%
\end{itemize}
\end{column}
\begin{column}{0.5\textwidth}
\begin{figure}
\centering
\includegraphics[width=\textwidth]{figures/country_distribution_top20.png}
\caption{Top 20 países-sede}
\end{figure}
\end{column}
\end{columns}
\end{frame}

% ========================================
% SEÇÃO 3: MODELAGEM ECONOMÉTRICA
% ========================================

\section{Modelagem Econométrica}

\begin{frame}{Escolha do Estimador: PPML}
\begin{block}{Justificativas para PPML}
\begin{enumerate}
\item \textbf{Natureza das variáveis:} Contagens com alta incidência de zeros
\item \textbf{Consistência:} Robusto a sobredispersão e heterocedasticidade
\item \textbf{Implementação:} Estável com efeitos fixos de alta dimensão
\end{enumerate}
\end{block}

\vspace{0.3cm}

\textbf{Forma funcional:} $\mathbb{E}[y\mid X] = \exp(X\beta)$

\textbf{Interpretação:} Coeficiente $\beta_k$ tem interpretação multiplicativa
\end{frame}

\begin{frame}{Estrutura de Efeitos Fixos}
\begin{block}{Estratégia de Identificação Causal}
\begin{itemize}
\item \textbf{$\mu_{id}$:} Efeitos fixos de membro (controla heterogeneidade não observada)
\item \textbf{$\mu_{ct}$:} Efeitos fixos país×tempo (controla choques comuns por país)
\item \textbf{$\mu_{pt}$:} Efeitos fixos partido×tempo (controla choques por partido)
\item \textbf{$\mu_{dt}$:} Efeitos fixos domínio×tempo (controla choques setoriais)
\end{itemize}
\end{block}

\vspace{0.3cm}

\textbf{Clustering:} Erros-padrão agrupados em domínio×tempo e membro
\end{frame}

\begin{frame}{Especificações do Modelo}
\begin{block}{Modelo Linear (Baseline)}
\begin{equation}
\text{questions}_{ijt} = \exp(\beta_1 \text{meetings}_{ijt} + \mathbf{X}_{ijt}\boldsymbol{\gamma} + \mu_{id} + \mu_{ct} + \mu_{pt} + \mu_{dt})
\end{equation}
\end{block}

\begin{block}{Modelo Quadrático (Retornos Marginais)}
\begin{equation}
\text{questions}_{ijt} = \exp(\beta_1 \text{meetings}_{ijt} + \beta_2 \text{meetings}_{ijt}^2 + \mathbf{X}_{ijt}\boldsymbol{\gamma} + \mu_{id} + \mu_{ct} + \mu_{pt} + \mu_{dt})
\end{equation}
\end{block}

\textbf{Controles:} Dummies de grupo político, país, comitês, delegações
\end{frame}

% ========================================
% SEÇÃO 4: TESTAGEM DAS HIPÓTESES
% ========================================

\section{Testagem das Hipóteses}

\begin{frame}{Hipótese 1: Efeito Geral do Lobbying}
\begin{block}{H1: MEPs com maior pressão de lobby exibem maior AL}
\begin{itemize}
\item \textbf{Teste:} Coeficiente de \textit{meetings} > 0
\item \textbf{Especificação:} PPML linear e quadrático
\item \textbf{Controles:} Efeitos fixos completos
\end{itemize}
\end{block}

\begin{table}
\centering
\caption{Teste da H1: Efeito de reuniões sobre perguntas}
\begin{tabularx}{\textwidth}{>{\raggedright\arraybackslash}p{.22\textwidth} >{\raggedright\arraybackslash}X >{\raggedright\arraybackslash}X}
    \toprule
      & PPML & PPML (Quad.) \\
    \midrule
    Reuni\~oes & 0,025*** (0,002) & 0,098*** (0,007) \\
    Reuni\~oes$^2$ &  & -0,004*** (0,001) \\
    \midrule
    Observa\c{c}\~oes & 600.237 & 600.237 \\
    Efeitos fixos & \multicolumn{2}{p{.72\textwidth}}{\raggedright pa\'is$\times$tempo; partido$\times$tempo; dom\'inio$\times$tempo} \\
    Cluster & \multicolumn{2}{p{.72\textwidth}}{\raggedright dom\'inio$\times$tempo; membro} \\
    \bottomrule
    \end{tabularx}
\end{table}

\textbf{Resultado:} H1 confirmada - coeficiente positivo e significativo
\end{frame}

\begin{frame}{Interpretação da H1}
\begin{block}{Modelo Linear}
\begin{itemize}
\item \textbf{Efeito marginal:} $\frac{\partial \mathbb{E}[y]}{\partial x} = \beta_1 \exp(X\beta)$
\item \textbf{Interpretação:} Aumento de 1 reunião → variação de $100 \times (e^{0.025} - 1) = 2.5\%$ nas perguntas
\end{itemize}
\end{block}

\begin{block}{Modelo Quadrático}
\begin{itemize}
\item \textbf{Efeito marginal:} $\frac{\partial \mathbb{E}[y]}{\partial x} = (\beta_1 + 2\beta_2 x) \exp(X\beta)$
\item \textbf{Retornos decrescentes:} $\beta_2 < 0$ indica saturação de agenda
\item \textbf{Magnitude:} $\beta_2 = -0.004$ sugere retornos decrescentes moderados
\end{itemize}
\end{block}

\textbf{Conclusão H1:} Lobbying aumenta significativamente a atividade parlamentar
\end{frame}


\begin{frame}{Curvas de Efeito: Visualização dos Resultados}
  \begin{columns}
    \begin{column}{0.5\textwidth}
    \begin{figure}
    \centering
    \includegraphics[width=\textwidth]{figures/fig8_effect_linear_ppml.pdf}
    \caption{Especificação linear}
    \end{figure}
  \end{column}
    \begin{column}{0.5\textwidth}
    \begin{figure}
    \centering
    \includegraphics[width=\textwidth]{figures/fig9_effect_quadratic_ppml.pdf}
    \caption{Especificação quadrática}
    \end{figure}
    \end{column}
  \end{columns}
  
  \textbf{Comparação:} Modelo quadrático captura retornos marginais decrescentes, porém pequeno
\end{frame}

\begin{frame}{Hipótese 2: Eficácia do Lobbying Empresarial}
\begin{block}{H2: Lobbying empresarial é mais eficaz em aumentar a AL}
\begin{itemize}
\item \textbf{Teste:} Comparar coeficientes entre categorias organizacionais
\item \textbf{Especificação:} PPML com diferentes tratamentos
\item \textbf{Hipótese:} $\beta_{Business} > \beta_{NGOs}, \beta_{Other}$
\end{itemize}
\end{block}
\end{frame}

\begin{frame}{Hipótese 2: Eficácia do Lobbying Empresarial}
\begin{figure}
\centering
\includegraphics[width=0.6\textwidth]{figures/fig_coeff_treatments_overall.pdf}
\caption{Efeito do lobbying por categoria de organização}
\end{figure}

\textbf{Análise:} Comparar magnitude dos coeficientes entre categorias
\end{frame}

\begin{frame}{Teste da H2: Análise por Categoria - RESULTADOS}
\begin{block}{Comparação de Eficácia - Evidência Empírica}
\begin{itemize}
\item \textbf{Geral:} Coeficiente base para comparação
\item \textbf{NGOs:} Efeito superior ao empresarial
\end{itemize}
\end{block}

\vspace{0.3cm}

\textbf{Resultado H2:} 
\begin{itemize}
\item \textbf{Status:} Rejeitada
\item \textbf{Evidência:} Empresas não são mais eficazes que NGOs. Contudo, retornos marginais descrescentes são pequenos, favorecendo atores com mais recursos
\item \textbf{Interpretação:} Lobbying empresarial é eficaz, mas não exclusivamente superior
\end{itemize}
\end{frame}

\begin{frame}{Hipótese 3: Eficácia em Temas Salientes}
\begin{block}{H3: Em temas mais salientes, lobbying não-empresarial supera o empresarial}
\begin{itemize}
\item \textbf{Teste:} Interação entre categoria organizacional e saliência do domínio
\item \textbf{Especificação:} PPML com efeitos heterogêneos por domínio
\item \textbf{Hipótese:} $\beta_{NGOs}^{Saliente} > \beta_{Business}^{Saliente}$
\end{itemize}
\end{block}
\end{frame}

\begin{frame}{Hipótese 3: Eficácia em Temas Salientes}
\begin{figure}
\centering
\includegraphics[width=0.8\textwidth]{figures/fig_coeff_treatments_by_domain.pdf}
\caption{Efeito do lobbying por categoria e domínio}
\end{figure}

\textbf{Análise:} Comparar eficácia entre categorias em domínios de alta saliência
\end{frame}

\begin{frame}{Teste da H3: Análise por Domínio e Categoria - RESULTADOS}
\begin{block}{Domínios de Alta Saliência - Evidência Empírica}
\begin{itemize}
\item \textbf{Ambiente e Clima:} Alta visibilidade política - NGOs têm vantagem
\item \textbf{Direitos Humanos:} Relevância normativa - NGOs mais eficazes
\item \textbf{Saúde:} Impacto direto na população - Padrão misto
\end{itemize}
\end{block}

\begin{block}{Domínios de Baixa Saliência}
\begin{itemize}
\item \textbf{Agricultura:} Especialização técnica - Business mantém vantagem
\item \textbf{Infraestrutura:} Menor visibilidade pública - Business mais eficaz
\end{itemize}
\end{block}

\textbf{Resultado H3:} 
\begin{itemize}
\item \textbf{Status:} Confirmada
\item \textbf{Evidência:} Em domínios salientes, NGOs superam Business
\item \textbf{Interpretação:} Saliência temática favorece organizações não-empresariais
\end{itemize}
\end{frame}

% ========================================
% SEÇÃO 5: CURVAS DE EFEITO E ROBUSTEZ
% ========================================

\section{Curvas de Efeito e Robustez}

\begin{frame}{Testes de Robustez}
\begin{block}{Especificações Alternativas}
\begin{itemize}
\item \textbf{Modelo linear vs. quadrático:} Resultados consistentes
\item \textbf{Diferentes tratamentos:} Categorias organizacionais, orçamento, experiência
\item \textbf{Análise por domínio:} Efeito positivo em todas as áreas temáticas
\item \textbf{Clustering robusto:} Erros-padrão em múltiplas dimensões
\item \textbf{Endogeneidade:} Utilização de PSM para controle de endogeneidade
\item \textbf{Defasagens:} Testes de event study
\end{itemize}
\end{block}

\vspace{0.3cm}

\textbf{Conclusão:} Resultados robustos a múltiplas especificações
\end{frame}

% ========================================
% SEÇÃO 6: IMPLICAÇÕES E LIMITAÇÕES
% ========================================

\section{Implicações e Limitações}

\begin{frame}{Implicações das Hipóteses Testadas}
\begin{block}{H1 Confirmada: Efeito Geral do Lobbying}
\begin{itemize}
\item \textbf{Evidência:} Lobbying aumenta atividade parlamentar
\item \textbf{Implicação:} Mecanismos de influência política operam efetivamente
\item \textbf{Política:} Importância da transparência e regulação
\end{itemize}
\end{block}

% TODO: REVER
\begin{block}{H2 Rejeitada: Heterogeneidade Organizacional}
\begin{itemize}
\item \textbf{Evidência:} Empresas não são mais eficazes que NGOs
\item \textbf{Implicação:} Diferentes estratégias de lobbying por tipo de organização
\item \textbf{Política:} Necessidade de equilibrar acesso e influência
\end{itemize}
\end{block}

\begin{block}{H3 Confirmada: Eficácia em Temas Salientes}
\begin{itemize}
\item \textbf{Evidência:} ONGs superam Empresas em domínios salientes
\item \textbf{Implicação:} Saliência temática afeta eficácia do lobbying
\item \textbf{Política:} Importância do contexto político para estratégias lobistas
\end{itemize}
\end{block}
\end{frame}

\begin{frame}{Limitações do Estudo}
\begin{block}{Desafios Metodológicos}
\begin{itemize}
\item \textbf{Medição:} Captura apenas reuniões registradas oficialmente
\item \textbf{Generalização:} Resultados específicos ao contexto europeu
\end{itemize}
\end{block}

\vspace{0.3cm}

\textbf{Nota:} Efeitos fixos e clustering mitigam mas não eliminam todas as preocupações
\end{frame}

\begin{frame}{Direções para Pesquisa Futura}
\begin{block}{Extensões Metodológicas}
\begin{itemize}
\item \textbf{Instrumentação:} Identificação de choques exógenos no lobbying
\item \textbf{Mecanismos:} Canais específicos de influência
\end{itemize}
\end{block}

\begin{block}{Extensões Substantivas}
\begin{itemize}
\item \textbf{Outcomes:} Votação, relatórios, emendas
\item \textbf{Contextos:} Outros parlamentos nacionais
\item \textbf{Períodos:} Análise de longo prazo e mudanças institucionais
\end{itemize}
\end{block}
\end{frame}

% ========================================
% SEÇÃO 7: CONCLUSÕES
% ========================================

% \section{Conclusões}


% \begin{frame}{Implicações para a Literatura}
% \begin{block}{Contribuições Principais}
% \begin{itemize}
% \item \textbf{Metodológica:} Aplicação de PPML com efeitos fixos de alta dimensão
% \item \textbf{Empírica:} Comparação entre domínios com dados em painel
% \item \textbf{Teórica:} Confirmação de mecanismos de influência política e heterogeneidade organizacional
% \item \textbf{Política:} Base para discussões sobre regulação do lobbying e equilíbrio de influência
% \end{itemize}
% \end{block}

% \vspace{0.3cm}

% \textbf{Relevância:} Estudo pioneiro que abre caminho para pesquisas futuras sobre lobbying organizacional
% \end{frame}

% % Slide final
% \begin{frame}
% \begin{center}
% \Large \textbf{Obrigado!}

% \vspace{1cm}
% \small
% \textit{Efeitos do Lobbying na Atividade Parlamentar: \\ Testagem de Três Hipóteses}
% \end{center}
% \end{frame}

\end{document}
