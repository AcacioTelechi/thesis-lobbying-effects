\chapter*{Agradecimentos}

Um trabalho científico demanda não apenas rigor metodológico, testes, experimentações e suor, mas, principalmente, aprendizado. Novas ideias, técnicas e desafios se mostram presentes em cada canto que procuramos soluções para testar nossas hipóteses. Trata-se de um desafio não-trivial, mas que se torna menos tortuoso quando há apoio de professores, família, amigos e colegas. Por isso, gostaria de agradecer a todos que contribuíram direta ou indiretamente para este trabalho.

Primeiramente, gostaria de agradecer ao meu orientador, Prof. Dr. Alexsandro Eugênio Pereira, pela orientação acadêmica, pelo apoio, pela paciência, pelos desafios intelectuais e por acreditar no meu projeto desde o início. Mesmo com todas as imensas modificações que o projeto sofreu desde a apresentação do pré-projeto, ele sempre acreditou em mim e me deu suporte para que eu pudesse continuar. Obrigado!

Gostaria de agradecer aos professores do Programa de Pós-Graduação em Ciência Política da UFPR pelo apoio e pelas valiosas sugestões que contribuíram para o aprimoramento desta tese. Em especial, agradeço ao Prof. Dr. Renato Perissinotto não somente pelas aulas sobre o conceito de poder, mas também por ter dado-nos espaço para debatermos e experimentarmos tentativas de mensuração desse intricado conceito.

À banca, deixo meus agradecimentos pelo tempo dedicado à leitura e pelas valiosas sugestões que contribuíram para o aprimoramento desta tese.

Aos meus pais, Danilo e Fernanda, à minha irmã, Eduarda, e à minha esposa, Patricia, o mais profundo agradecimento pelo amor, paciência, sacrifício e incentivo incondicional ao longo de todos estes anos.

Gostaria de agradecer aos meus colegas de pós-graduação pelo apoio e pelas valiosas sugestões que contribuíram para o aprimoramento desta tese.

Durante parte do doutorado contei com bolsa acadêmica concedida pela Coordenação de Aperfeiçoamento de Pessoal de Nível Superior (CAPES), portanto agradeço a instituição pelo apoio financeiro e pela contínua defesa da produção acadêmica.

Por fim, agradeço aos técnicos que trabalham diariamente para manter a estrutura universitária.