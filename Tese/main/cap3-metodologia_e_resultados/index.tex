\chapter{Metodologia}
\label{chapter:metodologia}
Este capítulo dedica-se a apresentar a proposta metodológica da tese. Como vimos, estudar os efeitos do lobby apresenta desafios metodológicos importantes. Primeiro, sistematizemos os seis principais complicadores para identificar os efeitos do lobby; em seguida, apresento a proposta de desenho metodológico para buscar essa identificação.

O primeiro desafio, que podemos chamar de "o problema da persistência temporal", relaciona-se com o fato de que o montante dos esforços de lobby ao longo do tempo tende a ser constante \cite{de_figueiredo_advancing_2014}. Por conta disso, alguns métodos de análise de dados em painel, que exigem variação nas variáveis de interesse, cujo objetivo seja medir os efeitos do esforço de lobby, pode não obter a variabilidade necessária.

A segunda complicação metodológica, ou "o problema do viés de seleção", refere-se ao fato de que a escolha dos alvos dos lobistas não é aleatória. Os lobistas tendem a escolher como alvos prioritários parlamentares aliados \cite{kollman1998outside, caldeira2000lobbying, hojnacki2001pac} ou neutros \cite{holyoke2003choosing, kelleher2009political, bertrand2014whom, gawande2012lobbying} que tenham posições importantes na tramitação de determinado projeto \cite{marshall2010lobby}.

O terceiro desafio, que pode ser denominado de "o problema do lobby contra-ativo", diz respeito a possibilidade de que, mesmo que um grupo não obtenha sucesso, não quer dizer que ele não tenha influência. Pode significar que houve um esforço de lobby no sentido contrário que obteve sucesso \cite{austen1996theory}.

O quarto obstáculo, ou "o problema das faces do poder", dispõe que o lobby pode agir em diferentes etapas do processo de decisão, seja na formação da agenda \cite{bachrach1962two}, seja na sua implementação.

A quinta dificuldade, ou "o problema das variáveis omitidas" \cite{de_figueiredo_advancing_2014}, está relacionado a possibilidade de que variáveis importantes podem não ser observadas. Como exemplo, podemos citar a habilidade de negociação de cada lobista. Outra possibilidade relacionada a esse desafio é caso os lobistas queiram pressionar de maneira não transparente.

Por fim, a sexta complicação, que chamemos de "o problema dos canais de influência", refere-se a capacidade de que os lobistas tem de agir em diferentes canais \cite{dur_measuring_2008}. Eles podem optar desde lobby direto a campanhas de pressão perante a opinião pública, ou \textit{outside lobbying} \cite{kollman1998outside}. Além disso podem mesmo agir na própria seleção dos tomadores de decisão \cite{fordham2003selection}. A seguir, analisemos a proposta para a identificação do efeito do lobby no comportamento parlamentar.

\section{Estratégia de identificação e de operacionalização}
\label{section:identificacao}

A análise causal dos efeitos do lobby enfrenta desafios significativos, conforme discutido. Para superá-los, esta tese adota uma estratégia de identificação que combina um foco analítico restrito com um modelo empírico robusto, fundamentado no \textit{framework} de análise do comportamento parlamentar (\ref{fig:framework}).

A variável dependente utilizada, portanto, é a \acrfull{al}, operacionalizada como o número de perguntas parlamentares que um \acrshort{mpe} ($i$) apresenta em um determinado domínio temático ($d$) e período ($t$). A escolha por focar no comportamento parlamentar, em vez de resultados de políticas públicas, reduz a complexidade da cadeia causal e aproxima a análise da ação individual do legislador.

Esta escolha é amplamente respaldada pela literatura sobre o Parlamento Europeu, que demonstra que as perguntas parlamentares são um instrumento multifacetado. Uma primeira vertente teórica, baseada na literatura de agente-principal, enxerga as perguntas como um mecanismo de fiscalização e responsabilização \cite{jensen2013parliamentary, maricut2020qa, martin2013parliamentary}. Nessa perspectiva, os legisladores (o principal) delegam poder à burocracia (o agente), mas utilizam instrumentos como as perguntas para monitorar suas ações, especialmente em temas de alta saliência política \cite{mccubbin1984congressional, saalfeld2000members, strom2000delegation, koop2011explaining}. 

Uma segunda vertente foca nas perguntas como um mecanismo de sinalização e posicionamento político. Nessa visão, a atividade parlamentar é menos sobre a fiscalização e mais sobre estratégia: os parlamentares utilizam as perguntas para expressar preferências, sinalizar seus interesses para a liderança partidária, eleitores e grupos de interesse, sendo uma ferramenta valiosa especialmente para a oposição \cite{martin2013parliamentary, otjes2017parliamentary, proksch2010parliamentary, bevan2023do, navarro2022banking}. Por serem uma forma de baixo custo para demonstrar atividade e engajamento em temas específicos, as perguntas tornam-se um indicador sensível da alocação de atenção e esforço de um parlamentar, funcionando como uma excelente proxy para medir mudanças no ativismo legislativo em resposta a estímulos externos, como o lobby.

A variável de tratamento, ou esforço de lobby ($L$), é mensurada pelo número de reuniões que o \acrshort{mpe} $i$ teve com lobistas com interesse no domínio $d$ no período $t$. Os dados são extraídos dos registros públicos do Parlamento Europeu. A escolha por essa operacionalização, embora não capture a totalidade dos canais de influência \cite{dur_measuring_2008}, se justifica pela centralidade do contato direto na atividade de lobby. A literatura define o lobby como a transferência de informações por meio de encontros privados \cite{de_figueiredo_advancing_2014} e, no contexto da \acrshort{ue}, as táticas que envolvem engajamento face a face são consideradas particularmente eficazes \cite{Huwyler2022}. Com o fortalecimento do \acrshort{pe} como arena decisória, garantir acesso direto a parlamentares influentes, como relatores, tornou-se uma estratégia prioritária para os grupos de interesse \cite{kluver2015legislative, marshall2010lobby}. Portanto, o número de reuniões funciona como uma proxy robusta e observável para a intensidade do esforço de lobby direcionado a legisladores individuais, refletindo o investimento em um dos recursos mais valiosos para os lobistas: o acesso e as conexões relacionais.

A relação entre o esforço de lobby e a \acrshort{al} é estimada por meio de um modelo de \acrfull{ppml}. O método, popularizado por Silva e Tenreyro (\citeyear{silva2006log}) para modelos de gravidade no comércio internacional, cuja fundamentação teórica remonta a Anderson (\citeyear{anderson1979theoretical}), é uma forma de \acrfull{glm} robusto para dados de contagem, especialmente na presença de muitos zeros. O modelo é estimado por meio de uma distribuição quasi-Poisson com uma função de ligação logarítmica, o que evita os vieses que podem surgir da transformação logarítmica da variável dependente em modelos de \acrfull{mqo}, uma questão comum com dados de contagem. Para a estimação, será utilizada a linguagem de programação estatística R, especificamente a função `fepois` do pacote `fixest`, que é altamente eficiente para estimar modelos Poisson com efeitos fixos de alta dimensão. A especificação do modelo, baseada no \textit{framework} teórico, é a seguinte:

\begin{equation}
    \label{eq:modelo_final}
    AL_{idt} = \beta L_{idt} + \gamma_{ct} + \lambda_{pt} + \theta_{dt} + X'_{it}\delta + \epsilon_{idt}
\end{equation}

Onde:
\begin{itemize}
    \item $AL_{idt}$ é a contagem de perguntas do eurodeputado $i$ no domínio $d$ no mês $t$;
    \item $L_{idt}$ é a contagem de reuniões de lobby do eurodeputado $i$ no domínio $d$ no mês $t$;
    \item $\beta$ é o coeficiente de interesse, que captura o efeito médio de uma reunião adicional sobre a produção de perguntas;
    \item $\gamma_{ct}$, $\lambda_{pt}$ e $\theta_{dt}$ são efeitos fixos de alta dimensão;
    \item $X'_{it}$ é um vetor de controles individuais;
    \item $\epsilon_{idt}$ é o termo de erro.
\end{itemize}

O componente central da estratégia de identificação reside nos efeitos fixos, que absorvem uma vasta gama de fatores de confusão observáveis e não observáveis, alinhando-se diretamente às três dimensões do comportamento parlamentar identificadas no \textit{framework} teórico:
\begin{itemize}
    \item \textbf{Efeito fixo de país-tempo ($\gamma_{ct}$)}: Captura qualquer choque ou tendência comum a todos os parlamentares de um mesmo país $c$ em um dado período $t$. Isso controla por fatores como eleições nacionais, mudanças na opinião pública do país ou estratégias de política externa que poderiam afetar o ativismo de seus representantes.
    \item \textbf{Efeito fixo de partido-tempo ($\lambda_{pt}$)}: Absorve choques comuns a todos os membros de um partido político europeu $p$ no tempo $t$, como mudanças na liderança do partido, alterações na plataforma ideológica ou estratégias coordenadas de atuação.
    \item \textbf{Efeito fixo de domínio-tempo ($\theta_{dt}$)}: Controla por fatores que afetam a saliência de um domínio temático $d$ no tempo $t$ para todos os parlamentares, como crises políticas, novas diretivas da Comissão Europeia ou eventos de grande repercussão midiática.
\end{itemize}

Ao incluir essa estrutura de efeitos fixos, a identificação do coeficiente $\beta$ se dá a partir da variação do número de reuniões \textit{dentro} de um mesmo parlamentar ao longo do tempo, após controlar por todas as fontes de variação comuns ao seu país, partido e aos temas em que atua.

A especificação econométrica adotada permite mitigar os principais desafios metodológicos da literatura. Primeiramente, a abordagem lida com o \textbf{viés de seleção e o problema das variáveis omitidas}, considerados os desafios mais críticos. Os lobistas não escolhem seus alvos aleatoriamente; a seleção baseia-se em alinhamento prévio, posição em comitês ou influência. A estratégia de efeitos fixos controla grande parte desses critérios de seleção: fatores não observáveis e estáveis do parlamentar (como sua ideologia ou competência intrínseca) são absorvidos, enquanto fatores variantes no tempo são capturados pelos efeitos fixos de país-tempo, partido-tempo e domínio-tempo. A premissa de identificação é que, uma vez controlados esses fatores, a variação residual no número de reuniões que um parlamentar recebe é exógena à sua produção legislativa.

Para além dos efeitos fixos, o modelo inclui um vetor de variáveis de controle ($X'_{it}$) que capturam características individuais e variantes no tempo do parlamentar, com especial atenção para os cargos que ocupa. Variáveis \textit{dummy} indicam se o eurodeputado exerce funções de liderança, como a presidência de comissões, delegações ou grupos de trabalho. A inclusão desses controles é crucial para a estratégia de identificação, pois tais posições de poder são um critério central na seleção de alvos pelos lobistas e, ao mesmo tempo, podem influenciar diretamente o ativismo legislativo do parlamentar. Ao controlar por essas posições, isolamos o efeito do lobby de um importante fator de confusão, fortalecendo a premissa de que a variação residual no número de reuniões é exógena.

Adicionalmente, o modelo contorna o problema da \textbf{persistência temporal}. Como a estimação utiliza apenas a variação intra-parlamentar, o fato de alguns parlamentares receberem consistentemente mais lobby do que outros não enviesa a estimativa de $\beta$. O efeito é identificado a partir de parlamentares que alteram seu nível de engajamento com lobistas ao longo do tempo. A questão do \textbf{lobby contra-ativo} também é endereçada, uma vez que a variável de tratamento ($L_{idt}$) representa o volume total de reuniões, sem distinguir a direção da pressão. Assim, o coeficiente $\beta$ estimado representa o efeito \textit{líquido} de uma reunião adicional. Para explorar a heterogeneidade desse efeito, o modelo é também estimado separadamente para diferentes tipos de lobistas (e.g., empresariais, ONGs), permitindo analisar se a natureza do grupo de interesse altera o resultado.

Finalmente, em relação às múltiplas \textbf{faces do poder e canais de influência}, esta pesquisa delimita seu escopo para um canal (lobby direto via reuniões) e um tipo de resultado (ativismo legislativo). Essa delimitação é uma escolha metodológica deliberada para obter maior validade interna e uma identificação causal mais crível, ainda que se reconheça que o lobby opera por múltiplos canais e visa a diferentes resultados.

Em suma, a estratégia de identificação se baseia em um modelo de painel com efeitos fixos de alta dimensão que exploram a estrutura de variação dos dados (parlamentar, país, partido, domínio e tempo) para isolar o efeito do lobby de inúmeros fatores de confusão, conforme fundamentado pelo \textit{framework} de análise do comportamento parlamentar.
% como foi testa a da primeira hipotese?
Para testar a primeira hipótese (H1) — de que \acrshort{mpe}s sujeitos a maior pressão de lobby exibem maior \acrshort{al} — foi empregada a estratégia de identificação detalhada na Seção \ref{section:identificacao}. O teste empírico baseia-se na estimação do modelo de \acrshort{ppml} apresentado na Equação \ref{eq:modelo_final}, que utiliza o número de perguntas parlamentares como variável dependente e o número de reuniões com lobistas como a principal variável de tratamento.

A escolha pelo \acrshort{ppml}, estimado com a função `fepois` do pacote \texttt{fixest} em R, justifica-se pela natureza da variável dependente (dados de contagem com excesso de zeros), evitando os vieses de modelos lineares com transformações logarítmicas. A hipótese H1 é corroborada se o coeficiente de interesse, $\beta$, associado à variável de reuniões de lobby ($L_{idt}$), for positivo e estatisticamente significativo. Os erros padrão foram clusterizados no nível de domínio-tempo para corrigir a possível autocorrelação nos resíduos dentro de cada área temática ao longo do tempo.

Adicionalmente, para investigar a possibilidade de retornos marginais decrescentes, foi estimado um modelo quadrático, adicionando o termo de reuniões ao quadrado ($L_{idt}^2$) à especificação. Este modelo, apresentado na Equação \ref{eq:modelo_final_quad}, permite verificar se o efeito do lobby se atenua em níveis mais altos de intensidade.

\begin{equation}
    \label{eq:modelo_final_quad}
    AL_{idt} = \beta_1 L_{idt} + \beta_2 L_{idt}^2 + \gamma_{ct} + \lambda_{pt} + \theta_{dt} + X'_{it}\delta + \epsilon_{idt}
\end{equation}

Um coeficiente $\beta_2$ negativo e significativo indicaria que, embora reuniões adicionais aumentem a atividade legislativa, o impacto de cada nova reunião é progressivamente menor. Isso sugere a existência de um ponto de saturação, a partir do qual o esforço de lobby adicional perde eficácia.

Por fim, para explorar a heterogeneidade do efeito entre diferentes áreas de política pública, o modelo \acrshort{ppml} principal foi estimado separadamente para cada um dos domínios temáticos. Essa análise permite avaliar se a influência do lobby é um fenômeno homogêneo ou se sua magnitude e significância variam conforme o contexto temático em que o \acrshort{mpe} atua.



% Como foi testada a segunda hipótese?
Para testar a segunda hipótese (H2) — de que as empresas exercem influência agregada superior sobre a \acrshort{al} quando comparadas a outras categorias de atores — adotamos uma estratégia em duas etapas que espelha os dois mecanismos centrais do lobby: (i) \textit{persuasão por reunião} e (ii) \textit{acesso via volume de reuniões}. Essa decomposição permite separar a eficácia marginal de cada encontro do poder de garantir encontros em maior quantidade.

Primeiro, estimamos a eficácia marginal de uma reunião por tipo de ator com o modelo \acrshort{ppml}, compatível com a especificação base (Seção \ref{section:identificacao}). Depois, modelamos o volume de acesso com uma regressão Binomial Negativa sobre dados de contagem de reuniões por lobista. Por fim, aproximamos o efeito agregado multiplicando o efeito marginal por reunião previsto pelo \acrshort{ppml} pelo número esperado de reuniões previsto pelo modelo de acesso.

% \paragraph{Etapa 1 — Persuasão: comparação de tratamentos com PPML.}
Reutilizamos a especificação central do \acrshort{ppml} (pacote \texttt{fixest}, função \texttt{fepois}), incluindo efeitos fixos de alta dimensão de país-tempo, partido-tempo e domínio-tempo, e o mesmo vetor de controlos individuais $X'_{it}$ utilizado em H1. Para comparar a eficácia por tipo de ator, substituímos a variável de tratamento $L_{idt}$ por versões alternativas que isolam o número de reuniões atribuíveis a cada categoria: Empresas, \acrshort{ong}s e Outros. A especificação estimada é

\begin{equation}
    AL_{idt} = \beta^{(k)} L_{idt}^{(k)} + \gamma_{ct} + \lambda_{pt} + \theta_{dt} + X'_{it}\delta + \epsilon_{idt}, \quad k \in \{\text{Business},\ \text{NGOs},\ \text{Other}\}
\end{equation}

onde $\beta^{(k)}$ é o coeficiente de interesse para o tratamento $k$. Os erros padrão são clusterizados em domínio-tempo, mitigando correlações intracluster ao longo do tempo dentro de cada área temática. A evidência a favor de maior eficácia marginal por reunião para um dado tipo de ator decorre de um $\beta^{(k)}$ maior e estatisticamente significativo.

% \paragraph{Justificativa.}
O \acrshort{ppml} é apropriado para dados de contagem com muitos zeros e evita vieses de transformações logarítmicas em \acrshort{mqo}. A estrutura de efeitos fixos (país-tempo, partido-tempo e domínio-tempo) aborda choques comuns e heterogeneidade não observada correlacionada com a variável de tratamento, preservando identificação na variação intra-parlamentar ao longo do tempo, conforme discutido na Seção \ref{section:identificacao}.

% \paragraph{Etapa 2 — Acesso: frequência de reuniões com Binomial Negativo.}
Para captar o componente de acesso, estimamos o número de reuniões por lobista com um modelo Binomial Negativo (\texttt{MASS::glm.nb}), apropriado para contagens com sobredispersão. A variável dependente é o total de reuniões realizadas pelo lobista, explicada por: (i) categoria do ator (\acrshort{ong}, Empresa, Outros), (ii) orçamento máximo de lobby em log, (iii) interação entre categoria e orçamento, e (iv) controlos setoriais de atuação do lobista e país da sede. Para interpretação limpa, recentramos o orçamento em seu valor médio. Em termos funcionais:

\begin{equation}
    \mathbb{E}[\text{meetings}_\ell\,|\,Z_\ell] = \exp\big(\alpha_0 + \alpha_1\,\mathbb{1}\{\text{Business}\} + \alpha_2\,\ln B_\ell + \alpha_3\,\mathbb{1}\{\text{Business}\}\cdot\ln B_\ell + W_\ell'\eta\big),
\end{equation}

onde $\ell$ indexa lobistas, $B_\ell$ é o orçamento, $W_\ell$ agrega os controlos e a interação permite que o efeito do orçamento sobre o número de reuniões varie por categoria. Este modelo gera previsões de reuniões esperadas por categoria e nível de orçamento, que embasam as figuras e tabelas correspondentes.

Combinamos as duas etapas para aproximar o impacto total do lobby por categoria e orçamento:

\begin{equation}
    \text{Efeito total}_{(k,\,B)} \;\approx\; \underbrace{\mathbb{E}[\text{reuniões}\,|\,k,\,B]}_{\text{Acesso}}\;\times\; \underbrace{\beta^{(k)}}_{\text{Persuasão por reunião (PPML)}}.
\end{equation}

Essa métrica captura a ideia de que influência política depende tanto da eficácia marginal de cada encontro quanto da capacidade de produzir muitos encontros. A interpretação requer a premissa de \textit{separabilidade} entre os processos de acesso e persuasão: condicionais às variáveis observadas (e aos efeitos fixos na Etapa 1), os fatores não observados que afetam a capacidade de agendar reuniões não devem enviesar sistematicamente a eficácia por reunião. Discutimos essa premissa e suas implicações na seção de resultados.

% \paragraph{Implementação e reprodutibilidade.}
Toda a estimação foi conduzida em R. A Etapa 1 utiliza \texttt{fixest::fepois} com efeitos fixos de país-tempo, partido-tempo e domínio-tempo, controlos individuais detalhados e \textit{cluster} de erros em domínio-tempo. A Etapa 2 utiliza \texttt{MASS::glm.nb} com controles setoriais e geográficos, além da interação categoria $\times$ orçamento. As figuras e tabelas foram exportadas para \texttt{Tese/figures/h2\_test} e \texttt{Tese/tables/h2\_test}, respeitando o padrão do projeto. A discussão substantiva dos resultados e sua interpretação estão em \texttt{Tese/main/cap4-resultados/h2.tex}.


% Como foi testada a terceira hipótese?
Para testar a terceira hipótese (H3) — de que, em temas mais salientes, o lobby não empresarial (\acrshort{ong}s) é relativamente mais eficaz do que o lobby empresarial — estendemos a especificação \acrshort{ppml} com efeitos fixos (Seção \ref{section:identificacao}) para incorporar uma medida de saliência do tema e suas interações com o tipo de ator.

% \paragraph{Medida de saliência do tema.}
A saliência é operacionalizada como o volume total de reuniões de lobby observadas em cada combinação domínio-tempo, transformado por \(\log(1+x)\) para lidar com zeros e reduzir assimetria, e posteriormente padronizado para média zero e desvio-padrão um (\(\text{salience\_std}\)). Esta escolha alinha-se à literatura que utiliza intensidade de atividade como proxy de saliência agregada do tema e permite interpretar os coeficientes de interação como variações no efeito marginal ao longo do gradiente de saliência.

% \paragraph{Especificação com interações (PPML).}
Mantemos os mesmos efeitos fixos de alta dimensão e o vetor de controlos individuais utilizados em H1 e H2 (país-tempo, partido-tempo e domínio-tempo; ver \ref{section:identificacao}). A novidade é a inclusão de termos de interação entre o número de reuniões atribuíveis a cada categoria de ator e a saliência padronizada. A forma geral estimada é:

\begin{equation}
    \label{eq:modelo_h3}
    AL_{idt} = \sum_{k \in \{\text{Business},\text{NGOs},\text{Other}\}} \Big[ \beta_k L_{idt}^{(k)} + \phi_k L_{idt}^{(k)} \cdot \text{Sal}_{dt} \Big] + \gamma_{ct} + \lambda_{pt} + \theta_{dt} + X'_{it}\delta + \epsilon_{idt}
\end{equation}

onde \(L_{idt}^{(k)}\) é o número de reuniões do \acrshort{mpe} \(i\) com a categoria \(k\) no domínio \(d\) e tempo \(t\), e \(\text{Sal}_{dt}\) é a saliência (padronizada) do domínio \(d\) no tempo \(t\). Os termos \(\beta_k\) capturam o efeito marginal por reunião quando a saliência está no seu valor médio, enquanto \(\phi_k\) capturam como esse efeito varia com a saliência. Erros-padrão são clusterizados em domínio-tempo para acomodar correlação serial dentro de cada área temática.

% \paragraph{Identificação e interpretação.}
A identificação repousa na variação intra-parlamentar ao longo do tempo, com choques comuns de país, partido e domínio absorvidos pelos efeitos fixos. Um \(\phi_{\text{NGOs}}\) menos negativo (ou mais positivo) do que \(\phi_{\text{Business}}\) indica que, à medida que a saliência aumenta, o efeito marginal das \acrshort{ong}s se deteriora mais lentamente ou permanece mais resiliente que o das empresas, implicando uma vantagem relativa em temas sob maior escrutínio.

% \paragraph{Implementação e saídas.}
Estimamos o modelo com \texttt{fixest::fepois}, preservando os controlos e efeitos fixos das hipóteses anteriores e \textit{cluster} em domínio-tempo. As tabelas e figuras associadas ao teste da H3 são exportadas para \texttt{Tese/tables/h3\_test} e \texttt{Tese/figures/h3\_test}, e a análise substantiva encontra-se em \texttt{Tese/main/cap4-resultados/h3.tex}.
