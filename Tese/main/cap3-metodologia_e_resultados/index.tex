\chapter{Metodologia}
\label{chapter:metodologia}
Este capítulo dedica-se a apresentar a proposta metodológica da tese. Como vimos, estudar os efeitos do lobby apresenta desafios metodológicos importantes. Primeiro, sistematizemos os seis principais complicadores para identificar os efeitos do lobby; em seguida, apresento a proposta de desenho metodológico para buscar essa identificação.

O primeiro desafio, que podemos chamar de "o problema da persistência temporal", relaciona-se com o fato de que o montante dos esforços de lobby ao longo do tempo tende a ser constante \cite{de_figueiredo_advancing_2014}. Por conta disso, alguns métodos de análise de dados em painel, que exigem variação nas variáveis de interesse, cujo objetivo seja medir os efeitos do esforço de lobby, pode não obter a variabilidade necessária.

A segunda complicação metodológica, ou "o problema do viés de seleção", refere-se ao fato de que a escolha dos alvos dos lobistas não é aleatória. Os lobistas tendem a escolher como alvos prioritários parlamentares aliados \cite{kollman1998outside, caldeira2000lobbying, hojnacki2001pac} ou neutros \cite{holyoke2003choosing, kelleher2009political, bertrand2014whom, gawande2012lobbying} que tenham posições importantes na tramitação de determinado projeto \cite{marshall2010lobby}.

O terceiro desafio, que pode ser denominado de "o problema do lobby contra-ativo", diz respeito a possibilidade de que, mesmo que um grupo não obtenha sucesso, não quer dizer que ele não tenha influência. Pode significar que houve um esforço de lobby no sentido contrário que obteve sucesso \cite{austen1996theory}.

O quarto obstáculo, ou "o problema das faces do poder", dispõe que o lobby pode agir em diferentes etapas do processo de decisão, seja na formação da agenda \cite{bachrach1962two}, seja na sua implementação.

A quinta dificuldade, ou "o problema das variáveis omitidas" \cite{de_figueiredo_advancing_2014}, está relacionado a possibilidade de que variáveis importantes podem não ser observadas. Como exemplo, podemos citar a habilidade de negociação de cada lobista. Outra possibilidade relacionada a esse desafio é caso os lobistas queiram pressionar de maneira não transparente.

Por fim, a sexta complicação, que chamemos de "o problema dos canais de influência", refere-se a capacidade de que os lobistas tem de agir em diferentes canais \cite{dur_measuring_2008}. Eles podem optar desde lobby direto a campanhas de pressão perante a opinião pública, ou \textit{outside lobbying} \cite{kollman1998outside}. Além disso podem mesmo agir na própria seleção dos tomadores de decisão \cite{fordham2003selection}. A seguir, analisemos a proposta para a identificação do efeito do lobby no comportamento parlamentar.

\section{Estratégia de identificação e de operacionalização}

Muitos trabalhos sobre os efeitos lobby preocupam-se em medi-los resultados de políticas (como impacto em tarifas comerciais), de volume de contratos com o poder público, benefícios fiscais, entre outras variáveis dependentes amplas. Ora, os determinantes desses resultados são complexos. Ao focarmos no comportamento parlamentar conseguimos controlar melhorar os seus determinantes.

O comportamento parlamentar pode se dar de diferentes formas. Os trabalhos que buscam estudar os efeitos sobre os votos dos parlamentares, caem no problema de que o voto possui, também, determinantes complexos, tais como o partido, relação com os pares, entre outros. Buscando minimizar esse desafio, este trabalho analisa a \acrfull{al}, isto é, o quão ativo determinado parlamentar é em determinado tema. A atividade, portanto, envolve a proposição de projetos, requisições, emendas e discursos. a \acrshort{al}, então, é majoritariamente determinada pelos interesses dos parlamentares. Não quero dizer que as variáveis como partido e relação entre os pares não influenciam, apenas defendo que a \acrshort{al} está mais fortemente correlacionada com os objetivos de um parlamentar, como visto no \textit{framework} de análise do comportamento parlamentar (\ref{fig:framework}). Assim buscamos minimizar os impactos do problema das variáveis omitidas. De tal forma que esperamos que o \acrshort{al} seja uma função de características individuais dos parlamentares, dos países pelos quais os eurodeputados são eleitos e dos partidos dos quais fazem parte. 

Consideremos então que:
\begin{center}
$AL_{1,icpdt}$ = o ativismo legislativo do parlamentar $i$, eleito pelo país $c$ e pertencente ao grupo político $p$, caso tenha recebido esforço de lobby significativo no domínio temático $d$ no tempo $t$; e

$AL_{0,icpdt}$ = o ativismo legislativo do parlamentar $i$, eleito pelo país $c$ e pertencente ao grupo político $p$, caso \emph{não} tenha recebido esforço de lobby significativo no domínio temático $d$ no tempo $t$.
\end{center}

Esses resultados são, contudo, potenciais, isto é, não observamos $AL_{1idt}$ e $AL_{0idt}$ ao mesmo tempo, apenas um ou outro. Partindo do \textit{framework} de análise do comportamento parlamentares, esperamos que:

% VERSÃO 3
\begin{equation}
    \label{eq:esperanca}
    E(AL_{0,icpdt} \vert c,p,d,t,X) = \gamma_{ct} + \lambda_{pt} + \theta_{dt} + X'_{cpt} \delta
\end{equation}

A equação \ref{eq:esperanca} denota que, na ausência de pressão significativa de lobby, a \acrshort{al} é determinada pela soma dos efeitos:
\begin{itemize}
    \item $\gamma_{ct}$: captura o efeito específico do país no tempo $t$. Ele leva em conta fatores que podem influenciar o ativismo legislativo em todos os partidos dentro de um determinado país em um momento específico (por exemplo, sistema eleitoral, características do eleitorado, condições econômicas, etc.);
    \item $\lambda_{pt}$: representa o efeito específico do partido no tempo $t$. Ele leva em conta fatores que podem influenciar o ativismo legislativo em todos os países para um determinado partido em um momento específico (por exemplo, ideologia do partido, mudanças de liderança, etc.);
    \item $\theta_{dt}$: captura o efeito do domínio temático no tempo $t$, tais como a competitividade, saliência, etc.
    \item $X'_{cpt} \delta$: este termo leva em conta os efeitos de outros fatores relevantes (variáveis de controle) capturados no vetor $X$ específicos de cada parlamentar, como expertise, experiência, gênero, etc.
\end{itemize}

Esperamos que os efeitos de interação entre país e partido seja zero, uma vez que os grupos políticos se organizam transnacionalmente. Por conta disso, não incluí um termo que capturasse um efeito do partido no ativismo legislativo que variasse em diferentes países. 

Considerando que o esforço de lobby varia de acordo com o domínio temático, podemos utilizar o método da \acrfull{ddd}, cuja especificação se daria por:

\begin{equation}
    \begin{split}
        AL_{icpdt} &= \gamma_{ct} + \lambda_{pt} + \theta_{dt}\\
        &+ \beta_1 L_{idt}\\
        &+ \beta_2 (L_{idt} * T_{dt})\\
        &+ \beta_3 (L_{idt} * D_d)\\
        &+ \beta_4 (L_{idt} * T_{dt} * D_d)\\
        &+ X'_{icpdt} \delta
    \end{split}
\end{equation}

Onde:
\begin{itemize}
    \item $AL_{icpdt}$: denota o ativismo legislativo esperado do parlamentar $i$, dado que foi eleito pelo país $c$, pertence grupo político $p$, no domínio $d$ no tempo $t$;
    \item $L_{idt}$: o esforço de lobby sobre o parlamentar $i$ no domínio $d$ no tempo $t$;
    \item $T_{dt}$: variável \textit{dummy} que recebe o valor 1 a partir do momento em que o esforço do lobby é realizado no domínio $d$; e
    \item $D_d$: variável \textit{dummy} que recebe o valor 1 nos temas em que houve esforço significativo de lobby.
\end{itemize}

De tal forma que $\beta_1$ representa o efeito médio do lobby no ativismo legislativo em todos os domínios e períodos, independente do tratamento (receber ou não lobby); $\beta_2$ captura o efeito do lobby após o tratamento; $\beta_3$ mensura a diferença no efeito do lobby entre tratados e grupo de controle antes do tratamento; e $\beta_4$ é estimador principal do \acrshort{ddd}, pois captura o efeito médio do tratamento sobre os tratados.

Para operacionalização, irei mensurar o $AL$ por meio de um indicador de ativismo legislativo, o qual levará em consideração a autoria de requisições, discursos e emendas realizadas pelo parlamentar por domínio temático. Para a captação de dados, utilizarei a API disponibilizada pelo próprio Parlamento e, adicionalmente, lançarei mão de \textit{scraper} em Python a fim de pegar os dados mais detalhados dos trâmites que não estão disponíveis na API, porém estão no site da \acrshort{pe}.

Os esforços de lobby ($L$) será mensurado pela quantidade de reuniões que determinado parlamentar realizou com representantes de grupos de pressão. Os eurodeputados são obrigados a publicizá-las, quando forem relatores. Esses dados também são possíveis de serem extraídos por meio de \textit{scraper}. Com isso, teremos o registro de quem se reuniu com quem e a data do encontro, o que permitirá calcular as variáveis \textit{dummy} ($T$ e $D$).

Para encontrar o domínio em que o lobby foi realizado, faremos uma inferência a partir do representante de interesse lobista. Ou seja, partiremos do pressuposto que as organizações tem interesses específicos em temas específicos, que tem relação com a sua natureza, de tal modo que, por exemplo, uma associação comercial tem interesses em temas comerciais e econômicos, já uma organização da sociedade civil de defesa do meio ambiente tem interesses no domínio de meio ambiente. O Registro de Transparência da \acrshort{ue} nos permite saber quem são os atores registrados, bem como suas áreas de interesse. Assim, ao cruzarmos com os dados dos encontros publicados, poderemos saber em qual área determinado parlamentar recebeu pressão de lobby. Demais dados podem ser obtidos pela API, tais como grupo político, gênero, ocupação, país eleito de um eurodeputado. 



% Justificar questions como proxy

% Justificar meetings como proxy
% como foi testa a da primeira hipotese?
Para testar a primeira hipótese (H1) — de que \acrshort{mpe}s sujeitos a maior pressão de lobby exibem maior \acrfull{al} — foi empregada a estratégia de identificação detalhada na Seção \ref{section:identificacao}. O teste empírico baseia-se na estimação do modelo de \acrfull{ppml} apresentado na Equação \ref{eq:modelo_final}, que utiliza o número de perguntas parlamentares como variável dependente e o número de reuniões com lobistas como a principal variável de tratamento.

A escolha pelo \acrshort{ppml}, estimado com a função `fepois` do pacote \texttt{fixest} em R, justifica-se pela natureza da variável dependente (dados de contagem com excesso de zeros), evitando os vieses de modelos lineares com transformações logarítmicas. A hipótese H1 é corroborada se o coeficiente de interesse, $\beta$, associado à variável de reuniões de lobby ($L_{idt}$), for positivo e estatisticamente significativo. Os erros padrão foram clusterizados no nível de domínio-tempo para corrigir a possível autocorrelação nos resíduos dentro de cada área temática ao longo do tempo.

Adicionalmente, para investigar a possibilidade de retornos marginais decrescentes, foi estimado um modelo quadrático, adicionando o termo de reuniões ao quadrado ($L_{idt}^2$) à especificação. Este modelo, apresentado na Equação \ref{eq:modelo_final_quad}, permite verificar se o efeito do lobby se atenua em níveis mais altos de intensidade.

\begin{equation}
    \label{eq:modelo_final_quad}
    AL_{idt} = \beta_1 L_{idt} + \beta_2 L_{idt}^2 + \gamma_{ct} + \lambda_{pt} + \theta_{dt} + X'_{it}\delta + \epsilon_{idt}
\end{equation}

Um coeficiente $\beta_2$ negativo e significativo indicaria que, embora reuniões adicionais aumentem a atividade legislativa, o impacto de cada nova reunião é progressivamente menor. Isso sugere a existência de um ponto de saturação, a partir do qual o esforço de lobby adicional perde eficácia.

Por fim, para explorar a heterogeneidade do efeito entre diferentes áreas de política pública, o modelo \acrshort{ppml} principal foi estimado separadamente para cada um dos domínios temáticos. Essa análise permite avaliar se a influência do lobby é um fenômeno homogêneo ou se sua magnitude e significância variam conforme o contexto temático em que o \acrshort{mpe} atua.


% Como foi testada a segunda hipótese?
Para testar a segunda hipótese (H2) — de que as empresas exercem influência agregada superior sobre a \acrshort{al} quando comparadas a outras categorias de atores — adotamos uma estratégia em duas etapas que espelha os dois mecanismos centrais do lobby: (i) \textit{persuasão por reunião} e (ii) \textit{acesso via volume de reuniões}. Essa decomposição permite separar a eficácia marginal de cada encontro do poder de garantir encontros em maior quantidade.

Primeiro, estimamos a eficácia marginal de uma reunião por tipo de ator com o modelo \acrshort{ppml}, compatível com a especificação base (Seção \ref{section:identificacao}). Depois, modelamos o volume de acesso com uma regressão Binomial Negativa sobre dados de contagem de reuniões por lobista. Por fim, aproximamos o efeito agregado multiplicando o efeito marginal por reunião previsto pelo \acrshort{ppml} pelo número esperado de reuniões previsto pelo modelo de acesso.

% \paragraph{Etapa 1 — Persuasão: comparação de tratamentos com PPML.}
Reutilizamos a especificação central do \acrshort{ppml} (pacote \texttt{fixest}, função \texttt{fepois}), incluindo efeitos fixos de alta dimensão de país-tempo, partido-tempo e domínio-tempo, e o mesmo vetor de controlos individuais $X'_{it}$ utilizado em H1. Para comparar a eficácia por tipo de ator, substituímos a variável de tratamento $L_{idt}$ por versões alternativas que isolam o número de reuniões atribuíveis a cada categoria: Empresas, \acrshort{ong}s e Outros. A especificação estimada é

\begin{equation}
    AL_{idt} = \beta^{(k)} L_{idt}^{(k)} + \gamma_{ct} + \lambda_{pt} + \theta_{dt} + X'_{it}\delta + \epsilon_{idt}, \quad k \in \{\text{Business},\ \text{NGOs},\ \text{Other}\}
\end{equation}

onde $\beta^{(k)}$ é o coeficiente de interesse para o tratamento $k$. Os erros padrão são clusterizados em domínio-tempo, mitigando correlações intracluster ao longo do tempo dentro de cada área temática. A evidência a favor de maior eficácia marginal por reunião para um dado tipo de ator decorre de um $\beta^{(k)}$ maior e estatisticamente significativo.

% \paragraph{Justificativa.}
O \acrshort{ppml} é apropriado para dados de contagem com muitos zeros e evita vieses de transformações logarítmicas em \acrshort{mqo}. A estrutura de efeitos fixos (país-tempo, partido-tempo e domínio-tempo) aborda choques comuns e heterogeneidade não observada correlacionada com a variável de tratamento, preservando identificação na variação intra-parlamentar ao longo do tempo, conforme discutido na Seção \ref{section:identificacao}.

% \paragraph{Etapa 2 — Acesso: frequência de reuniões com Binomial Negativo.}
Para captar o componente de acesso, estimamos o número de reuniões por lobista com um modelo Binomial Negativo (\texttt{MASS::glm.nb}), apropriado para contagens com sobredispersão. A variável dependente é o total de reuniões realizadas pelo lobista, explicada por: (i) categoria do ator (\acrshort{ong}, Empresa, Outros), (ii) orçamento máximo de lobby em log, (iii) interação entre categoria e orçamento, e (iv) controlos setoriais de atuação do lobista e país da sede. Para interpretação limpa, recentramos o orçamento em seu valor médio. Em termos funcionais:

\begin{equation}
    \mathbb{E}[\text{meetings}_\ell\,|\,Z_\ell] = \exp\big(\alpha_0 + \alpha_1\,\mathbb{1}\{\text{Business}\} + \alpha_2\,\ln B_\ell + \alpha_3\,\mathbb{1}\{\text{Business}\}\cdot\ln B_\ell + W_\ell'\eta\big),
\end{equation}

onde $\ell$ indexa lobistas, $B_\ell$ é o orçamento, $W_\ell$ agrega os controlos e a interação permite que o efeito do orçamento sobre o número de reuniões varie por categoria. Este modelo gera previsões de reuniões esperadas por categoria e nível de orçamento, que embasam as figuras e tabelas correspondentes.

Combinamos as duas etapas para aproximar o impacto total do lobby por categoria e orçamento:

\begin{equation}
    \text{Efeito total}_{(k,\,B)} \;\approx\; \underbrace{\mathbb{E}[\text{reuniões}\,|\,k,\,B]}_{\text{Acesso}}\;\times\; \underbrace{\beta^{(k)}}_{\text{Persuasão por reunião (PPML)}}.
\end{equation}

Essa métrica captura a ideia de que influência política depende tanto da eficácia marginal de cada encontro quanto da capacidade de produzir muitos encontros. A interpretação requer a premissa de \textit{separabilidade} entre os processos de acesso e persuasão: condicionais às variáveis observadas (e aos efeitos fixos na Etapa 1), os fatores não observados que afetam a capacidade de agendar reuniões não devem enviesar sistematicamente a eficácia por reunião. Discutimos essa premissa e suas implicações na seção de resultados.

% \paragraph{Implementação e reprodutibilidade.}
Toda a estimação foi conduzida em R. A Etapa 1 utiliza \texttt{fixest::fepois} com efeitos fixos de país-tempo, partido-tempo e domínio-tempo, controlos individuais detalhados e \textit{cluster} de erros em domínio-tempo. A Etapa 2 utiliza \texttt{MASS::glm.nb} com controles setoriais e geográficos, além da interação categoria $\times$ orçamento. As figuras e tabelas foram exportadas para \texttt{Tese/figures/h2\_test} e \texttt{Tese/tables/h2\_test}, respeitando o padrão do projeto. A discussão substantiva dos resultados e sua interpretação estão em \texttt{Tese/main/cap4-resultados/h2.tex}.


% Como foi testada a terceira hipótese?
Para testar a terceira hipótese (H3) — de que, em temas mais salientes, o lobby não empresarial (\acrshort{ong}s) é relativamente mais eficaz do que o lobby empresarial — estendemos a especificação \acrshort{ppml} com efeitos fixos (Seção \ref{section:identificacao}) para incorporar uma medida de saliência do tema e suas interações com o tipo de ator.

% \paragraph{Medida de saliência do tema.}
A saliência é operacionalizada como o volume total de reuniões de lobby observadas em cada combinação domínio-tempo, transformado por \(\log(1+x)\) para lidar com zeros e reduzir assimetria, e posteriormente padronizado para média zero e desvio-padrão um (\(\text{salience\_std}\)). Esta escolha alinha-se à literatura que utiliza intensidade de atividade como proxy de saliência agregada do tema e permite interpretar os coeficientes de interação como variações no efeito marginal ao longo do gradiente de saliência.

% \paragraph{Especificação com interações (PPML).}
Mantemos os mesmos efeitos fixos de alta dimensão e o vetor de controlos individuais utilizados em H1 e H2 (país-tempo, partido-tempo e domínio-tempo; ver \ref{section:identificacao}). A novidade é a inclusão de termos de interação entre o número de reuniões atribuíveis a cada categoria de ator e a saliência padronizada. A forma geral estimada é:

\begin{equation}
    \label{eq:modelo_h3}
    AL_{idt} = \sum_{k \in \{\text{Business},\text{NGOs},\text{Other}\}} \Big[ \beta_k L_{idt}^{(k)} + \phi_k L_{idt}^{(k)} \cdot \text{Sal}_{dt} \Big] + \gamma_{ct} + \lambda_{pt} + \theta_{dt} + X'_{it}\delta + \epsilon_{idt}
\end{equation}

onde \(L_{idt}^{(k)}\) é o número de reuniões do \acrshort{mpe} \(i\) com a categoria \(k\) no domínio \(d\) e tempo \(t\), e \(\text{Sal}_{dt}\) é a saliência (padronizada) do domínio \(d\) no tempo \(t\). Os termos \(\beta_k\) capturam o efeito marginal por reunião quando a saliência está no seu valor médio, enquanto \(\phi_k\) capturam como esse efeito varia com a saliência. Erros-padrão são clusterizados em domínio-tempo para acomodar correlação serial dentro de cada área temática.

% \paragraph{Identificação e interpretação.}
A identificação repousa na variação intra-parlamentar ao longo do tempo, com choques comuns de país, partido e domínio absorvidos pelos efeitos fixos. Um \(\phi_{\text{NGOs}}\) menos negativo (ou mais positivo) do que \(\phi_{\text{Business}}\) indica que, à medida que a saliência aumenta, o efeito marginal das \acrshort{ong}s se deteriora mais lentamente ou permanece mais resiliente que o das empresas, implicando uma vantagem relativa em temas sob maior escrutínio.
