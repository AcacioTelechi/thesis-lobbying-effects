% Como foi testada a terceira hipótese?
Para testar a terceira hipótese (H3) — de que, em temas mais salientes, o lobby não empresarial (\acrshort{ong}s) é relativamente mais eficaz do que o lobby empresarial — estendemos a especificação \acrshort{ppml} com efeitos fixos (Seção \ref{section:identificacao}) para incorporar uma medida de saliência do tema e suas interações com o tipo de ator.

% \paragraph{Medida de saliência do tema.}
A saliência é operacionalizada como o volume total de reuniões de lobby observadas em cada combinação domínio-tempo, transformado por \(\log(1+x)\) para lidar com zeros e reduzir assimetria, e posteriormente padronizado para média zero e desvio-padrão unitário (\(\text{salience\_std}\)). Esta escolha alinha-se à literatura que utiliza intensidade de atividade como proxy de saliência agregada do tema e permite interpretar os coeficientes de interação como variações no efeito marginal ao longo do gradiente de saliência.

% \paragraph{Especificação com interações (PPML).}
Mantemos os mesmos efeitos fixos de alta dimensão e o vetor de controles individuais utilizados em H1 e H2 (país-tempo, partido-tempo e domínio-tempo; ver \ref{section:identificacao}). A novidade é a inclusão de termos de interação entre o número de reuniões atribuíveis a cada categoria de ator e a saliência padronizada. A forma geral estimada é:

\begin{equation}
    \label{eq:modelo_h3}
    AL_{idt} = \sum_{k \in \{\text{Business},\text{NGOs},\text{Other}\}} \Big[ \beta_k L_{idt}^{(k)} + \phi_k L_{idt}^{(k)} \cdot \text{Sal}_{dt} \Big] + \gamma_{ct} + \lambda_{pt} + \theta_{dt} + X'_{it}\delta + \epsilon_{idt}
\end{equation}

onde \(L_{idt}^{(k)}\) é o número de reuniões do \acrshort{mpe} \(i\) com a categoria \(k\) no domínio \(d\) e tempo \(t\), e \(\text{Sal}_{dt}\) é a saliência (padronizada) do domínio \(d\) no tempo \(t\). Os termos \(\beta_k\) capturam o efeito marginal por reunião quando a saliência está no seu valor médio, enquanto \(\phi_k\) capturam como esse efeito varia com a saliência. Erros-padrão são clusterizados em domínio-tempo para acomodar correlação serial dentro de cada área temática.

% \paragraph{Identificação e interpretação.}
A identificação repousa na variação intra-parlamentar ao longo do tempo, com choques comuns de país, partido e domínio absorvidos pelos efeitos fixos. Um \(\phi_{\text{NGOs}}\) menos negativo (ou mais positivo) do que \(\phi_{\text{Business}}\) indica que, à medida que a saliência aumenta, o efeito marginal das \acrshort{ong}s se deteriora mais lentamente ou permanece mais resiliente que o das empresas, implicando uma vantagem relativa em temas sob maior escrutínio.
