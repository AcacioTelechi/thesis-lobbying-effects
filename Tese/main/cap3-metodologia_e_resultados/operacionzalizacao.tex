\section{Operacionalização dos Dados}
\label{sec:operacionalizacao}

Esta seção descreve detalhadamente o processo de operacionalização dos dados utilizados na presente pesquisa, desde a extração das fontes primárias até a construção final dos painéis de dados utilizados nas análises econométricas. A transparência e replicabilidade deste processo são fundamentais para assegurar a confiabilidade dos resultados obtidos.

\subsection{Fontes de Dados}

A pesquisa baseia-se em três principais fontes de dados, todas públicas e oficiais:

\textbf{(1) API do Parlamento Europeu:} Utilizada para extrair dados sobre perguntas parlamentares e reuniões com representantes de interesse. A API fornece dados estruturados em formato JSON sobre atividades parlamentares desde 2009, incluindo metadados como datas, participantes, e identificadores únicos.

\textbf{(2) Registro de Transparência da União Europeia:} Base de dados em formato Excel contendo informações sobre organizações registradas como representantes de interesse, incluindo categoria, orçamento, área de atuação e data de registro.

\textbf{(3) Base de Dados de Membros do Parlamento Europeu (MEPs):} Informações sobre os parlamentares europeus, incluindo filiação partidária, país de origem, comissões de participação e períodos de mandato.

\subsection{Processo de Extração de Dados}

\subsubsection{Extração de Reuniões}

O processo de extração de dados sobre reuniões foi implementado através de um sistema de \textit{web scraping} paralelo (\textit{script} \texttt{1\_extraction.ipynb}). O sistema utiliza a classe \texttt{ParliamentDataHandler} para realizar requisições à API do Parlamento Europeu com controle de taxa de requisições para evitar sobrecarga do servidor.

A extração cobriu o período de janeiro de 2024 a outubro de 2024 inicialmente, sendo posteriormente expandida para o período completo de 2009 a 2024. O sistema implementa:
\begin{itemize}
    \item Processamento paralelo com controle de concorrência (máximo de 10 \textit{threads})
    \item Controle de taxa de requisições (500 requisições por janela de 5 minutos)
    \item Sistema de \textit{cache} para evitar requisições duplicadas
    \item Tratamento robusto de erros com \textit{retry} automático
\end{itemize}

Cada reunião extraída contém informações sobre: data, parlamentar(es) envolvido(s), representantes presentes, capacidade do parlamentar na reunião, e identificadores únicos.

\subsubsection{Extração de Perguntas Parlamentares}

A extração de perguntas parlamentares seguiu processo similar, mas com etapas adicionais devido à necessidade de acessar o conteúdo textual completo (\textit{script} \texttt{1.2\_extraction\_questions\_data.ipynb}). O processo envolveu:

\textbf{Fase 1 - Metadados:} Extração dos metadados das perguntas através da API, obtendo identificadores únicos, autores, datas de publicação e tipos de pergunta.

\textbf{Fase 2 - Conteúdo Textual:} Download dos documentos PDF correspondentes a cada pergunta a partir do repositório documental do Parlamento Europeu. Esta fase processou 72.154 documentos, implementando sistema de \textit{retry} para documentos inacessíveis.

\textbf{Fase 3 - Processamento Textual:} Extração e limpeza do conteúdo textual dos PDFs utilizando a biblioteca \texttt{PyPDF2}, com foco na seção "Subject" de cada pergunta para análise temática posterior.

\subsection{Tratamento e Harmonização dos Dados}

\subsubsection{Matching de Organizações}

Um dos desafios centrais foi o \textit{matching} entre os nomes das organizações presentes nas reuniões e aquelas registradas no Registro de Transparência (\textit{script} \texttt{2.0\_treat\_fuzzy.ipynb}). Este processo utilizou técnicas de correspondência \textit{fuzzy} devido a variações na grafia e denominações oficiais.

O algoritmo implementado utiliza múltiplas métricas de similaridade:
\begin{itemize}
    \item \textit{Ratio}: Similaridade geral entre strings
    \item \textit{Partial ratio}: Melhor correspondência de subsequências
    \item \textit{Token sort ratio}: Similaridade após ordenação de tokens
    \item \textit{Token set ratio}: Similaridade de conjuntos de tokens únicos
\end{itemize}

O processo foi paralelizado em lotes de 10 organizações, processando 99.370 correspondências potenciais em 5 horas e 40 minutos. Foram consideradas correspondências de alta confiança aquelas com score $\geq 85\%$, resultando em 2.166 matches confirmados.

\subsubsection{Classificação Temática}

A classificação temática das perguntas parlamentares foi realizada utilizando técnicas de processamento de linguagem natural (\textit{script} \texttt{2.6\_analyze\_questions\_docs.ipynb}). O processo empregou o modelo \texttt{facebook/bart-large-mnli} para classificação \textit{zero-shot} multi-rótulo.

\textbf{Desenvolvimento da Taxonomia:} Após iterações exploratórias com diferentes conjuntos de categorias, definiu-se uma taxonomia final de 9 domínios temáticos:
\begin{enumerate}
    \item Assuntos Externos e Segurança
    \item Economia e Comércio  
    \item Tecnologia
    \item Infraestrutura e Indústria
    \item Meio Ambiente e Clima
    \item Saúde
    \item Direitos Humanos
    \item Educação
    \item Agricultura
\end{enumerate}

O processo classificou 72.155 perguntas em lotes de 8 documentos, executado em GPU durante 28 horas e 58 minutos. Cada pergunta recebeu scores de probabilidade para todos os domínios, permitindo análises tanto de classificação única (domínio mais provável) quanto multi-rótulo.

\subsubsection{Tratamento de Dados dos Parlamentares}

O tratamento dos dados dos parlamentares envolveu a criação de um \textit{timeline} temporal para capturar mudanças em filiações partidárias, participação em comissões e outros aspectos institucionais (\textit{script} \texttt{2.3\_treat\_meps.ipynb}).

Para cada parlamentar e data relevante, o sistema determina:
\begin{itemize}
    \item Grupo político de filiação
    \item País de origem
    \item Participação em comissões (tipos e funções)
    \item Outras funções institucionais
\end{itemize}

Este processo gerou um dataset temporal com 1.353 parlamentares únicos ao longo de todas as datas relevantes do período de análise.

\subsubsection{Categorização de Lobistas}

Os lobistas registrados foram categorizados em três grupos principais com base em sua natureza organizacional (\textit{script} \texttt{2.1\_treatment\_meetings\_orgs.ipynb}):

\textbf{Business:} Empresas, grupos empresariais e associações comerciais e de negócios.

\textbf{NGOs:} Organizações não-governamentais, plataformas e redes similares.

\textbf{Other:} Demais categorias incluindo sindicatos, think tanks, instituições acadêmicas, consultorias, entidades públicas mistas, e representações de terceiros países.

Adicionalmente, foram criadas variáveis de classificação orçamentária (baixo, médio, alto) baseadas na distribuição dos orçamentos declarados e de tempo desde registro (em dias).

\subsection{Construção do Painel de Dados}

\subsubsection{Estrutura Temporal}

A construção do painel de dados envolveu decisões importantes sobre a agregação temporal. Testaram-se três níveis: diário, semanal e mensal. A agregação mensal foi selecionada como padrão por:
\begin{itemize}
    \item Reduzir esparsidade excessiva dos dados diários
    \item Manter granularidade suficiente para capturar variações temporais
    \item Facilitar a identificação econométrica com maior variabilidade
\end{itemize}

\subsubsection{Estrutura de Painel Tridimensional}

O painel final possui estrutura tridimensional (parlamentar × tempo × domínio temático), permitindo análises sofisticadas dos efeitos heterogêneos por área temática. A construção envolveu (\textit{scripts} \texttt{2.4\_panel\_data.ipynb} e \texttt{2.7\_panel\_data\_w\_topics.ipynb}):

\textbf{Passo 1 - Agregação Temporal:} Contagem de perguntas e reuniões por parlamentar-mês-domínio, considerando múltiplas classificações temáticas quando aplicável.

\textbf{Passo 2 - Junção com Dados Institucionais:} Para cada observação parlamentar-tempo, incorporação das características institucionais vigentes (partido, país, comissões).

\textbf{Passo 3 - Preenchimento de Missings:} Criação de observações com valor zero para todas as combinações parlamentar-tempo-domínio não observadas, assegurando painel balanceado.

\textbf{Passo 4 - Construção de Variáveis de Controle:} Criação de variáveis \textit{dummy} para grupos políticos, países, e funções em comissões, utilizando categorias de referência (França para país, não especificado).

\subsubsection{Dataset Final}

O dataset final (\texttt{df\_long\_v2.csv}) contém:
\begin{itemize}
    \item \textbf{Observações:} Painel tridimensional parlamentar × mês × domínio
    \item \textbf{Período:} 2014-2019 (8ª legislatura) e 2019-2024 (9ª legislatura)
    \item \textbf{Variáveis principais:} \texttt{questions} (contagem de perguntas), \texttt{meetings} (contagem de reuniões)
    \item \textbf{Controles institucionais:} Dummies para grupos políticos, países, funções em comissões
    \item \textbf{Identificadores:} \texttt{member\_id}, \texttt{Y.m} (ano-mês), \texttt{domain} (domínio temático)
\end{itemize}

\subsection{Validação e Controle de Qualidade}

Durante todo o processo de operacionalização, foram implementados múltiplos controles de qualidade:

\textbf{Validação de Extração:} Comparação de amostras extraídas com dados disponíveis publicamente no site do Parlamento Europeu.

\textbf{Validação de Matching:} Revisão manual de amostras aleatórias de matches de organizações, confirmando precisão superior a 90\% para matches com score $\geq 85\%$.

\textbf{Validação Temática:} Avaliação qualitativa de classificações temáticas através de amostras aleatórias, confirmando adequação das categorias desenvolvidas.

\textbf{Verificações de Consistência:} Testes de integridade referencial entre datasets, verificação de períodos de mandato dos parlamentares, e consistência temporal das observações.

Este processo rigoroso de operacionalização assegurou a qualidade e confiabilidade dos dados utilizados nas análises subsequentes, fornecendo base sólida para as inferências causais propostas na tese.