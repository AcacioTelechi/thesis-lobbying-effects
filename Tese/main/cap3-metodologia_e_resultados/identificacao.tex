\section{Estratégia de identificação e de operacionalização}
\label{section:identificacao}

A análise causal dos efeitos do lobby enfrenta desafios significativos, conforme discutido. Para superá-los, esta tese adota uma estratégia de identificação que combina um foco analítico restrito com um modelo empírico robusto, fundamentado no \textit{framework} de análise do comportamento parlamentar (\ref{fig:framework}).

A variável dependente utilizada, portanto, é a \acrfull{al}, operacionalizada como o número de perguntas parlamentares que um eurodeputado ($i$) apresenta em um determinado domínio temático ($d$) e período ($t$). A escolha por focar no comportamento parlamentar, em vez de resultados de políticas públicas, reduz a complexidade da cadeia causal e aproxima a análise da ação individual do legislador.

Esta escolha é amplamente respaldada pela literatura sobre o Parlamento Europeu, que demonstra que as perguntas parlamentares são um instrumento multifacetado. Uma primeira vertente teórica, baseada na literatura de ator-principal, enxerga as perguntas como um mecanismo de fiscalização e responsabilização \cite{jensen2013parliamentary, maricut2020qa, martin2013parliamentary}. Nessa perspectiva, os legisladores (o principal) delegam poder à burocracia (o agente), mas utilizam instrumentos como as perguntas para monitorar suas ações, especialmente em temas de alta saliência política \cite{mccubbin1984congressional, saalfeld2000members, strom2000delegation, koop2011explaining}. 

Uma segunda vertente foca nas perguntas como um mecanismo de sinalização e posicionamento político. Nessa visão, a atividade parlamentar é menos sobre a fiscalização e mais sobre estratégia: os parlamentares utilizam as perguntas para expressar preferências, sinalizar seus interesses para a liderança partidária, eleitores e grupos de interesse, sendo uma ferramenta valiosa especialmente para a oposição \cite{martin2013parliamentary, otjes2017parliamentary, proksch2010parliamentary, bevan2023do, navarro2022banking}. Por serem uma forma de baixo custo para demonstrar atividade e engajamento em temas específicos, as perguntas tornam-se um indicador sensível da alocação de atenção e esforço de um parlamentar, funcionando como uma excelente proxy para medir mudanças no ativismo legislativo em resposta a estímulos externos, como o lobby.

A variável de tratamento, ou esforço de lobby ($L$), é mensurada pelo número de reuniões que o eurodeputado $i$ teve com grupos de interesse sobre o domínio $d$ no período $t$. Os dados são extraídos dos registros públicos do Parlamento Europeu. A escolha por essa operacionalização, embora não capture a totalidade dos canais de influência \cite{dur_measuring_2008}, se justifica pela centralidade do contato direto na atividade de lobby. A literatura define o lobby como a transferência de informações por meio de encontros privados \cite{de_figueiredo_advancing_2014} e, no contexto da \acrshort{ue}, as táticas que envolvem engajamento face a face são consideradas particularmente eficazes \cite{Huwyler2022}. Com o fortalecimento do \acrshort{pe} como arena decisória, garantir acesso direto a parlamentares influentes, como relatores, tornou-se uma estratégia prioritária para os grupos de interesse \cite{kluver2015legislative, marshall2010lobby}. Portanto, o número de reuniões funciona como uma proxy robusta e observável para a intensidade do esforço de lobby direcionado a legisladores individuais, refletindo o investimento em um dos recursos mais valiosos para os lobistas: o acesso e as conexões relacionais.

A relação entre o esforço de lobby e a \acrshort{al} é estimada por meio de um modelo de \acrfull{ppml}. O método, popularizado por Silva e Tenreyro (\citeyear{silva2006log}) para modelos de gravidade no comércio internacional, cuja fundamentação teórica remonta a Anderson (\citeyear{anderson1979theoretical}), é uma forma de \acrfull{glm} robusto para dados de contagem, especialmente na presença de muitos zeros. O modelo é estimado por meio de uma distribuição quasi-Poisson com uma função de ligação logarítmica, o que evita os vieses que podem surgir da transformação logarítmica da variável dependente em modelos de \acrfull{mqo}, uma questão comum com dados de contagem. Para a estimação, será utilizada a linguagem de programação estatística R, especificamente a função `fepois` do pacote `fixest`, que é altamente eficiente para estimar modelos Poisson com efeitos fixos de alta dimensão. A especificação do modelo, baseada no \textit{framework} teórico, é a seguinte:

\begin{equation}
    \label{eq:modelo_final}
    AL_{idt} = \beta L_{idt} + \gamma_{ct} + \lambda_{pt} + \theta_{dt} + X'_{it}\delta + \epsilon_{idt}
\end{equation}

Onde:
\begin{itemize}
    \item $AL_{idt}$ é a contagem de perguntas do eurodeputado $i$ no domínio $d$ no mês $t$;
    \item $L_{idt}$ é a contagem de reuniões de lobby do eurodeputado $i$ no domínio $d$ no mês $t$;
    \item $\beta$ é o coeficiente de interesse, que captura o efeito médio de uma reunião adicional sobre a produção de perguntas;
    \item $\gamma_{ct}$, $\lambda_{pt}$ e $\theta_{dt}$ são efeitos fixos de alta dimensão;
    \item $X'_{it}$ é um vetor de controles individuais;
    \item $\epsilon_{idt}$ é o termo de erro.
\end{itemize}

O componente central da estratégia de identificação reside nos efeitos fixos, que absorvem uma vasta gama de fatores de confusão observáveis e não observáveis, alinhando-se diretamente às três dimensões do comportamento parlamentar identificadas no \textit{framework} teórico:
\begin{itemize}
    \item \textbf{Efeito fixo de país-tempo ($\gamma_{ct}$)}: Captura qualquer choque ou tendência comum a todos os parlamentares de um mesmo país $c$ em um dado período $t$. Isso controla por fatores como eleições nacionais, mudanças na opinião pública do país ou estratégias de política externa que poderiam afetar o ativismo de seus representantes.
    \item \textbf{Efeito fixo de partido-tempo ($\lambda_{pt}$)}: Absorve choques comuns a todos os membros de um partido político europeu $p$ no tempo $t$, como mudanças na liderança do partido, alterações na plataforma ideológica ou estratégias coordenadas de atuação.
    \item \textbf{Efeito fixo de domínio-tempo ($\theta_{dt}$)}: Controla por fatores que afetam a saliência de um domínio temático $d$ no tempo $t$ para todos os parlamentares, como crises políticas, novas diretivas da Comissão Europeia ou eventos de grande repercussão midiática.
\end{itemize}

Ao incluir essa estrutura de efeitos fixos, a identificação do coeficiente $\beta$ se dá a partir da variação do número de reuniões \textit{dentro} de um mesmo parlamentar ao longo do tempo, após controlar por todas as fontes de variação comuns ao seu país, partido e aos temas em que atua.

A especificação econométrica adotada permite mitigar os principais desafios metodológicos da literatura. Primeiramente, a abordagem lida com o \textbf{viés de seleção e o problema das variáveis omitidas}, considerados os desafios mais críticos. Os lobistas não escolhem seus alvos aleatoriamente; a seleção baseia-se em alinhamento prévio, posição em comitês ou influência. A estratégia de efeitos fixos controla grande parte desses critérios de seleção: fatores não observáveis e estáveis do parlamentar (como sua ideologia ou competência intrínseca) são absorvidos, enquanto fatores variantes no tempo são capturados pelos efeitos fixos de país-tempo, partido-tempo e domínio-tempo. A premissa de identificação é que, uma vez controlados esses fatores, a variação residual no número de reuniões que um parlamentar recebe é exógena à sua produção legislativa.

Para além dos efeitos fixos, o modelo inclui um vetor de variáveis de controle ($X'_{it}$) que capturam características individuais e variantes no tempo do parlamentar, com especial atenção para os cargos que ocupa. Variáveis \textit{dummy} indicam se o eurodeputado exerce funções de liderança, como a presidência de comissões, delegações ou grupos de trabalho. A inclusão desses controles é crucial para a estratégia de identificação, pois tais posições de poder são um critério central na seleção de alvos pelos lobistas e, ao mesmo tempo, podem influenciar diretamente o ativismo legislativo do parlamentar. Ao controlar por essas posições, isolamos o efeito do lobby de um importante fator de confusão, fortalecendo a premissa de que a variação residual no número de reuniões é exógena.

Adicionalmente, o modelo contorna o problema da \textbf{persistência temporal}. Como a estimação utiliza apenas a variação intra-parlamentar, o fato de alguns parlamentares receberem consistentemente mais lobby do que outros não enviesa a estimativa de $\beta$. O efeito é identificado a partir de parlamentares que alteram seu nível de engajamento com lobistas ao longo do tempo. A questão do \textbf{lobby contra-ativo} também é endereçada, uma vez que a variável de tratamento ($L_{idt}$) representa o volume total de reuniões, sem distinguir a direção da pressão. Assim, o coeficiente $\beta$ estimado representa o efeito \textit{líquido} de uma reunião adicional. Para explorar a heterogeneidade desse efeito, o modelo é também estimado separadamente para diferentes tipos de lobistas (e.g., empresariais, ONGs), permitindo analisar se a natureza do grupo de interesse altera o resultado.

Finalmente, em relação às múltiplas \textbf{faces do poder e canais de influência}, esta pesquisa delimita seu escopo para um canal (lobby direto via reuniões) e um tipo de resultado (ativismo legislativo). Essa delimitação é uma escolha metodológica deliberada para obter maior validade interna e uma identificação causal mais crível, ainda que se reconheça que o lobby opera por múltiplos canais e visa a diferentes resultados.

Em suma, a estratégia de identificação se baseia em um modelo de painel com efeitos fixos de alta dimensão que exploram a estrutura de variação dos dados (parlamentar, país, partido, domínio e tempo) para isolar o efeito do lobby de inúmeros fatores de confusão, conforme fundamentado pelo \textit{framework} de análise do comportamento parlamentar.