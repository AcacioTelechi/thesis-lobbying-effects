% como foi testa a da primeira hipotese?
Para testar a primeira hipótese (H1) — de que \acrshort{mpe}s sujeitos a maior pressão de lobby exibem maior \acrshort{al} — foi empregada a estratégia de identificação detalhada na Seção \ref{section:identificacao}. O teste empírico baseia-se na estimação do modelo de \acrshort{ppml} apresentado na Equação \ref{eq:modelo_final}, que utiliza o número de perguntas parlamentares como variável dependente e o número de reuniões com lobistas como a principal variável de tratamento.

A hipótese H1 é corroborada se o coeficiente de interesse, $\beta$, associado à variável de reuniões de lobby ($L_{idt}$), for positivo e estatisticamente significativo. Os erros padrão foram clusterizados no nível de domínio-tempo para corrigir a possível autocorrelação nos resíduos dentro de cada área temática ao longo do tempo.

Adicionalmente, para investigar a possibilidade de retornos marginais decrescentes, foi estimado um modelo quadrático, adicionando o termo de reuniões ao quadrado ($L_{idt}^2$) à especificação. Este modelo, apresentado na Equação \ref{eq:modelo_final_quad}, permite verificar se o efeito do lobby se atenua em níveis mais altos de intensidade.

\begin{equation}
    \label{eq:modelo_final_quad}
    AL_{idt} = \beta_1 L_{idt} + \beta_2 L_{idt}^2 + \gamma_{ct} + \lambda_{pt} + \theta_{dt} + X'_{it}\delta + \epsilon_{idt}
\end{equation}

Um coeficiente $\beta_2$ negativo e significativo indicaria que, embora reuniões adicionais aumentem a atividade legislativa, o impacto de cada nova reunião é progressivamente menor. Isso sugere a existência de um ponto de saturação, a partir do qual o esforço de lobby adicional perde eficácia, ou efeitos marginais decrescentes.

Por fim, para explorar a heterogeneidade do efeito entre diferentes áreas de política pública, o modelo \acrshort{ppml} principal foi estimado separadamente para cada um dos domínios temáticos. Essa análise permite avaliar se a influência do lobby é um fenômeno homogêneo ou se sua magnitude e significância variam conforme o contexto temático em que o \acrshort{mpe} atua.

