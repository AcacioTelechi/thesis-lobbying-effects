Esta tese é teoricamente justificado por vários aspectos chave. Primeiramente, há uma necessidade por estudos de larga escala (\textit{large-n}), como destacado por Bunea e Baumgartner(\citeyear{Bunea2014}), para aprimorar a generalização e validade dos resultados. Em segundo lugar, a falta de análises comparativas de lobby em diferentes domínios de políticas públicas é aparente \cite{Bunea2014}. Em terceiro lugar, há uma escassez de trabalhos formais e experimentais no campo \cite{Bunea2014}. Em quarto lugar, a literatura existente é predominantemente caracterizada por análises descritivas e estudos de caso qualitativos, como observado por Coen (\citeyear{Coen2007}) e Bunea e Baumgartner (\citeyear{Bunea2014}). Esta tese visa contribuir empregando uma abordagem mais quantitativa e comparativa. 

Além disso, a exploração limitada da relação entre lobby e comportamento parlamentar, como exemplificado por Huwyler e Martin (\citeyear{Huwyler2022}), motiva o exame abrangente dessa associação neste trabalho. Esses aspectos ressaltam a significância da pesquisa proposta, abordando lacunas e oportunidades de contribuição no campo.

