% \section{Antecedentes e Relevância}

A influência de grupos de interesse, incluindo lobistas, na tomada de decisões políticas tem sido objeto de intenso debate e escrutínio nos últimos anos. A \acrfull{ue} não é exceção, pois se tornou um alvo altamente atrativo para esforços de lobby devido a sua natureza supranacional única e processos complexos de tomada de decisão.

O \acrfull{pe} tornou-se cada vez mais uma arena proeminente para atividades de lobby, particularmente no período pós-Tratado de Lisboa. Isso é atribuído em grande parte ao influente processo de co-decisão, que concedeu ao Parlamento poderes legislativos aprimorados \cite{Dionigi2017}. Como resultado, entender os efeitos do lobby sobre o comportamento dos Membros do Parlamento Europeu (\acrshort{mpe}s) é crucial para avaliar a legitimidade democrática geral da \acrshort{ue}.

Além disso, pouco se estudou sobre os efeitos do lobby no comportamento parlamentar \cite{Huwyler2022}. Portanto, o objetivo principal desta tese será medir os efeitos do lobby no comportamento parlamentar dos \acrshort{mpe}s.

No contexto desta pesquisa, o conceito de comportamento parlamentar é definido como o nível de envolvimento exibido por um parlamentar em um domínio político específico, que será referido como \acrfull{al}. É crucial distinguir \acrshort{al} do comportamento de voto, que foi extensivamente examinado por \cite{Hix2002} e engloba um aspecto distinto do engajamento parlamentar.

O lobby pode ser conceituado como uma forma de subsídio legislativo, como destacado por \cite{Hall2006}. Nesta perspectiva, os lobistas oferecem recursos valiosos, como informações custosas, inteligência política e mão de obra para legisladores estrategicamente escolhidos. O argumento central apresentado por \cite{Hall2006} é que os lobistas visam não persuadir ou alterar a mentalidade dos legisladores, mas sim apoiar aliados com ideias semelhantes na realização de seus objetivos compartilhados.

De acordo com \cite{muller1999}, políticos e partidos podem ser caracterizados como atores em busca de carreira, votos e políticas. Dada a natureza intrincada da formulação de políticas, os legisladores enfrentam restrições em sua atenção e devem priorizar domínios políticos específicos \cite{Hall2006, jones2005politics}. Como resultado, os grupos de interesse, ao estabelecerem relacionamentos com legisladores, podem potencialmente moldar a agenda e o foco da atenção dos legisladores \cite{Huwyler2022}. Assim, espero que \emph{\acrshort{mpe}s que experimentam níveis mais altos de pressão de lobby em relação a um determinado domínio temático estão mais inclinados a exibir um maior grau de \acrshort{al} dentro dessa área política específica} (\ref{item:H1}).

Outro aspecto crucial abordado nos estudos de lobby é a questão da desigualdade de representação. O lobby na \acrshort{ue} é caro. Como resultado, o acesso aos legisladores é uma função positiva dos recursos das organizações. Vários estudos observaram um aumento nas atividades de lobby corporativo dentro da \acrshort{ue} \cite{Bouwen2002, Poletti2016, Berkhout2018, Hanegraaff2021}.

O aumento do lobby empresarial é acompanhado por uma maior probabilidade de sucesso para os interesses comerciais, o que pode ser atribuído a três fatores chave. Em primeiro lugar, os interesses comerciais tendem a exibir melhores estruturas organizacionais e coordenação \cite{olson1971logic}. Em segundo lugar, eles geralmente possuem informações técnicas que são altamente relevantes e valiosas para os formuladores de políticas \cite{kerwin2018rulemaking}. Por fim, em terceiro lugar, os interesses empresariais possuem um papel importante na economia \cite{lindblom1980politics}, o que pode gerar benefícios eleitorais para os parlamentares. Portanto, espero que \emph{as pressões de lobby de organizações empresariais são mais propensas a aumentar a \acrshort{al} dos \acrshort{mpe}s}(\ref{item:H2}).

Em temas com maior saliência, isto é, maior relevância na opinião pública, a probabilidade de sucesso dessas organizações, contundo, tende a ser menor \cite{caldeira2000lobbying, baumgartner2010agendas, bonardi2005corporate, mahoney_lobbying_2007}. Temas com alta atenção pública podem ter consequências eleitorais para os parlamentares, afetando a capacidade de grupos de interesse empresariais de influenciar o processo decisório \cite{mahoney_lobbying_2007, mahoney2008brussels}. Nesse sentido esperamos encontrar, \emph{em temas com maior saliência, um maior efeito do lobby de organizações não-empresariais obre a \acrshort{al} dos \acrshort{mpe}s em comparação com os das organizações empresariais} (\ref{item:H3}).

Em suma, com base na literatura apresentada, espero que:
\begin{hypotheses}
    \item \label{item:H1} \acrshort{mpe}s que experimentam níveis mais altos de pressão de lobby em relação a um determinado domínio de questão estão mais inclinados a exibir um maior grau de \acrshort{al} dentro dessa área política específica.
    
    \item \label{item:H2} As pressões de lobby de organizações empresariais são mais propensas a aumentar a \acrshort{al} dos \acrshort{mpe}s.
    
    \item \label{item:H3} Em temas mais salientes, as pressões de lobby de organizações não empresariais são mais propensas a aumentar a \acrshort{al} dos \acrshort{mpe}s em comparação com as de organizações empresariais.
\end{hypotheses}

