% \section{Antecedentes e Relevância}

A influência de grupos de interesse, incluindo lobistas, na tomada de decisões políticas tem sido objeto de intenso debate e escrutínio nos últimos anos. A \acrfull{ue} não é exceção, pois se tornou um alvo altamente atrativo para esforços de lobby devido a sua natureza supranacional única e processos complexos de tomada de decisão.

O \acrfull{pe} tornou-se cada vez mais uma arena proeminente para atividades de lobby, particularmente no período pós-Tratado de Lisboa. Isso é atribuído em grande parte ao influente processo de co-decisão, que concedeu ao Parlamento poderes legislativos aprimorados \cite{Dionigi2017}. Como resultado, entender os efeitos do lobby sobre o comportamento dos Membros do Parlamento Europeu (\acrshort{mpe}s) é crucial para avaliar a legitimidade democrática geral da \acrshort{ue}.

Além disso, pouco se estudou sobre os efeitos do lobby no comportamento parlamentar \cite{Huwyler2022}. Portanto, o objetivo principal desta tese será medir os efeitos do lobby no comportamento parlamentar dos \acrshort{mpe}s.

No contexto desta pesquisa, o conceito de comportamento parlamentar é definido como \textit{o nível de envolvimento exibido por um parlamentar em um domínio político específico}, que será referido como \acrfull{al}. É crucial distinguir \acrshort{al} do comportamento de voto, que foi extensivamente examinado por Hix(\citeyear{Hix2002}) e engloba um aspecto distinto do engajamento parlamentar. Isto é, não buscamos analisar os padrões de votos dos parlamentares, mas sim o seu grau de envolvimento em um domínio político específico.

Entendo por lobby como \textit{o exercício de influência sobre a atividade legistativa realizado por atores interessados em um ambiente marcado por diferenças de recursos}. Essa definição evidencia aspectos fundamentais do fenômeno do lobby. Lobby é caracterizado por uma ação ("exercício") que demanda dispêndio de recursos a fim de alterar o comportamento parlamentar com maior ou menor grau de efetividade. O quão efetivo são essas ações e quais fatores afetam essa efetividade que são o foco desta tese.

Opto pelo termo "influência" para designar o objetivo final do exercício do lobby. Por diferentes vias, os lobistas buscam angariar apoio, ou reduzir atritos, dos parlamentares a fim de alcançar seus objetivos em um determinado domínio de políticas públicas. A escolha por "influência" em detrimento do termo "poder" reflete a natureza menos rigorosa do primeiro conceito, amplamente utilizado pela literatura empírica sobre lobby \cite{lowery_lobbying_2013}. Embora poder e influência sejam frequentemente tratados como sinônimos nos trabalhos empíricos, influência é um termo mais abrangente que não exige necessariamente a identificação de conflito de interesses entre os atores envolvidos, como discutido na seção \ref{section:poder_influe}. Há, assim, influência quando A altera o comportamento de B, independentemente da existência de conflito ou do mecanismo pelo qual isso ocorreu. Importante destacar que a identificação de influência constitui uma relação causal, implicando que o comportamento do parlamentar teria sido diferente na ausência do exercício do lobby \cite{lowery_lobbying_2013}. Essa relação causal é fundamental para a análise dos efeitos do lobby sobre o comportamento parlamentar, ainda que sua identificação empírica apresente desafios metodológicos significativos, como detalhado na seção \ref{section:poder_influe}.

O termo "atores interessados" refere-se à ampla gama de organizações e indivíduos que buscam influenciar o processo legislativo. A literatura empírica identifica uma diversidade significativa de atores envolvidos em atividades de lobby, incluindo corporações, associações comerciais, grupos de cidadãos, sindicatos, fundações, \textit{think tanks}, governos e instituições diversas \cite{baumgartner2009lobbying, de_figueiredo_advancing_2014}. Embora os interesses empresariais representem a maior parte dos gastos em lobby - mais de 84\% do total nos Estados Unidos, por exemplo \cite{de_figueiredo_advancing_2014} -, eles constituem apenas uma parcela, ainda que significativa, do número total de organizações que realizam lobby. A quantidade e variedade de grupos de interesse variam ao longo do tempo, impulsionadas pela legitimação de causas e pela competição por recursos \cite{lowery2007organized}. Grandes grupos de interesse organizados e grupos apoiados por grandes corporações têm maior probabilidade de realizar lobby em comparação a grupos menores, que frequentemente buscam atuar em coalizações para compensar limitações de recursos \cite{de_figueiredo_advancing_2014, richter2011good}. Essa diversidade de atores reflete a natureza pluralista dos sistemas políticos modernos, nos quais diferentes interesses competem por influência no processo decisório \cite{schmitter1974still}.

A expressão "ambiente marcado por diferenças de recursos" evidencia uma característica fundamental do fenômeno do lobby: a desigualdade na distribuição de recursos entre os diversos atores interessados. Os recursos mobilizados no lobby podem assumir diferentes naturezas, incluindo recursos informacionais, relacionais (conexões e acessos), econômicos e reputacionais \cite{Pop2013Lobbying}. A literatura empírica demonstra que o acesso aos legisladores tende a ser uma função positiva dos recursos das organizações, criando um ambiente onde grupos com maior capacidade financeira e organizacional possuem vantagens significativas \cite{dur2007question, eising2007institutional}. No contexto da \acrshort{ue}, por exemplo, organizações empresariais representam cerca de 70\% das organizações com acesso às instituições europeias \cite{Coen2007}, evidenciando a concentração de recursos entre interesses comerciais. Contudo, é importante destacar que recursos não se traduzem automaticamente em poder e influência \cite{simon_notes_1953}. O sucesso do lobby depende da utilização eficaz dos recursos disponíveis e da capacidade de mobilização dos grupos de interesse \cite{Pop2013Lobbying}. Além disso, o contexto institucional e as características específicas dos temas em debate podem moderar o impacto das diferenças de recursos, como discutido na seção \ref{section:effects}. Essa desigualdade de recursos levanta questões importantes sobre a representação democrática e a legitimidade das decisões políticas, uma vez que pode favorecer determinados interesses em detrimento de outros \cite{kohler2007desmyth}.

Se, por um lado, os lobistas buscam influenciar os parlamentares; por outro, estes possuem diversos incentivos para além do lobby que afetam o seu comportamento. De acordo com Müller(\citeyear{muller1999}), políticos e partidos podem ser caracterizados como atores em busca de carreira, votos e políticas. Dada a natureza intrincada da formulação de políticas públicas, os legisladores enfrentam restrições em sua atenção e devem priorizar domínios políticos específicos \cite{Hall2006, jones2005politics}. Como resultado, os grupos de interesse, ao estabelecerem relacionamentos com legisladores, podem potencialmente moldar a agenda e o foco da atenção dos legisladores \cite{Huwyler2022}. Assim, espero que \emph{\acrshort{mpe}s que experimentam níveis mais altos de pressão de lobby em relação a um determinado domínio temático estão mais inclinados a exibir um maior grau de \acrshort{al} dentro dessa área política específica} (\ref{item:H1}).

Outro aspecto crucial abordado nos estudos de lobby, como mencionado anteriormente, é a questão da desigualdade de representação. O lobby na \acrshort{ue} é caro. Como resultado, é de se esperar que o acesso aos legisladores seja uma função positiva dos recursos das organizações. Vários estudos observaram um aumento nas atividades de lobby corporativo dentro da \acrshort{ue} \cite{Bouwen2002, Poletti2016, Berkhout2018, Hanegraaff2021}.

O aumento do lobby empresarial é acompanhado por uma maior probabilidade de sucesso para os interesses comerciais, o que pode ser atribuído a três fatores chave. Em primeiro lugar, os interesses comerciais tendem a exibir melhores estruturas organizacionais e coordenação \cite{olson1971logic}. Em segundo lugar, eles geralmente possuem informações técnicas que são altamente relevantes e valiosas para os formuladores de políticas públicas \cite{kerwin2018rulemaking}. Por fim, em terceiro lugar, os interesses empresariais possuem um papel importante na economia \cite{lindblom1980politics}, o que pode gerar benefícios eleitorais para os parlamentares. Portanto, espero que \emph{as pressões de lobby de organizações empresariais são mais propensas a aumentar a \acrshort{al} dos \acrshort{mpe}s}(\ref{item:H2}).

Em temas com maior saliência, isto é, maior relevância na opinião pública, a probabilidade de sucesso dessas organizações, contundo, tende a ser menor \cite{caldeira2000lobbying, baumgartner2010agendas, bonardi2005corporate, mahoney_lobbying_2007}. Temas com alta atenção pública podem ter consequências eleitorais para os parlamentares, afetando a capacidade de grupos de interesse empresariais de influenciar o processo decisório \cite{mahoney_lobbying_2007, mahoney2008brussels}. Nesse sentido esperamos encontrar, \emph{em temas com maior saliência, um maior efeito do lobby de organizações não-empresariais sobre a \acrshort{al} dos \acrshort{mpe}s em comparação com os das organizações empresariais} (\ref{item:H3}).

Em suma, com base na literatura até então apresentada, espero que:
\begin{hypotheses}
    \item \label{item:H1} \acrshort{mpe}s que experimentam níveis mais altos de pressão de lobby em relação a um determinado domínio de questão estão mais inclinados a exibir um maior grau de \acrshort{al} dentro dessa área política específica.
    
    \item \label{item:H2} As pressões de lobby de organizações empresariais são mais propensas a aumentar a \acrshort{al} dos \acrshort{mpe}s.
    
    \item \label{item:H3} Em temas mais salientes, as pressões de lobby de organizações não empresariais são mais propensas a aumentar a \acrshort{al} dos \acrshort{mpe}s em comparação com as de organizações empresariais.
\end{hypotheses}


% Trazer alguma discussão sobre poder e influencia