Este texto para qualificação foi dividido em mais quatro capítulos. No primeiro, \ref{chapter:poder}, trago uma discussão sobre os conceitos de poder e influência. Além disso, apresento algumas definições sobre lobby, resultados encontrados na literatura empírica sobre os efeitos do lobby e uma revisão da literatura sobre lobby na \acrshort{ue}. Por fim, encerro o capítulo com uma revisão da literatura sobre comportamento parlamentar, nossa variável dependente, propondo um \textit{framework} de análise do comportamento parlamentar com base nos textos apresentados.

No capítulo \ref{chapter:ue}, resgato o crescimento da importância do \acrshort{pe} na construção da \acrshort{ue}. Além disso, apresento o processo legislativo da \acrshort{ue}, bem como os poderes legislativos, orçamentários e fiscalizatórios do Parlamento.

Em seguida, no capítulo \ref{chapter:metodologia}, apresento a metodologia que será empregada na tese, \acrfull{ddd}. Além disso, busco demonstrar a estratégia de identificação dos efeitos do lobby no comportamento parlamentar.

Por fim, no capítulo \ref{chapter:considfinal}, trago uma pequena consideração final sobre o projeto, destacando o que espero contribuir para a literatura sobre lobby, comportamento parlamentar e \acrshort{ue}.