\section{Poder e Influência: um debate teórico}
\label{section:poder_influe}

A discussão sobre os efeitos do lobby no comportamento parlamentar passa pelo debate sobre poder e influência. Inúmeras vezes vê-se afirmações sobre a "influência" dos lobistas, ou o "poder" do lobby. Mas o que queremos dizer quando falamos de "influência" e "poder"? 

A literatura empírica, frequentemente, aloca para essa discussão não mais do que algumas linhas. Entendo necessário pelo menos uma breve discussão sobre os termos uma vez que o estudo dos efeitos do lobby passa não só por uma medição de influência ou poder, mas também por uma relação causal. Isto é, dizer que um lobista influenciou determinado parlamentar é uma afirmação causal, portanto exige uma definição do que significa influência e de como se chegou a essa relação causal. 

O termo "influência" é utilizado pela literatura de maneira mais ampla do que "poder" uma vez que este é mais exigente. Segundo Perissinotto (\citeyear{perissinoto2008}), é possível distinguir dois campos conceituais sobre poder: um que entende poder como relações hierárquicas e conflituosas; e outro que entende o conceito como a consecução de interesses coletivos. O primeiro campo, mais hegemônico, pode, ainda, ser subdivido em duas concepções: uma subjetivista, que entende relações de poderes como conflituosas entre agentes conscientes dos seus interesses; e outra que entende poder como uma relação social institucionalizada, que distribui recursos sociais de maneira desigual, porém com um funcionamento que ocorre à revelia da consciência dos atores \cite{perissinoto2008}. 

Na concepção subjetivista, tributária da tradição weberiana, poder só é exercido sobre alguém quando há conflito entre os interesses dos atores e é baseado em algum fundamento. Influente nessa corrente é a definição de Robert Dahl (\citeyear{dahl2005governs}), segundo a qual poder é a capacidade de A fazer com que B aja de forma que de outro modo não o faria. Nesse sentido, a análise de uma relação de poder passa por identificar os interesses e propostas, e, em seguida, contabilizar vencedores e perdedores. Além disso, muitas vezes poder e influência são tradados como sinônimos \cite{lowery_lobbying_2013, simon_notes_1953}. 

    Uma distinção importante é destacada por Hebert Simon (\citeyear{simon_notes_1953}): o exercício do poder é diferente de recursos de poder. No primeiro caso, trata-se da ação social de A induzir a B fazer algo que de outro modo não faria. No segundo, refere-se aos recursos mobilizados para o exercício de poder \cite{clarkterry1968community}
    
    Os trabalhos que seguem essa tradição encontraram algumas evidências importantes para o estudo do poder: influência é bastante específica para o tema em debate, contextual e possui fraca relação com os recursos de poder \cite{lowery_lobbying_2013}. Resultados que estão de acordo com a natureza contingencial da influência do lobby defendida por neopluralistas \cite{mcfarland2004neopluralism}.

A segunda concepção de poder, a objetivista, relativiza a necessidade de conflito aparente \cite{perissinoto2008}. É possível haver relações de poder mesmo onde há consenso. Para essa concepção, é necessário analisar a própria formação das preferências, que é bastante mobilizada pela escola do elitismo democrático \cite{bachrach1967theory} por meio do conceito de segunda face do poder \cite{bachrach1962two}.

    A influência de interesses organizados vai além de táticas e estratégias em propostas específicas, como é o foco dos subjetivistas. O poder também se manifesta na manipulação da agenda política, limitando o debate a questões inócuas para determinados grupos  \cite{bachrach1962two}. Essa mobilização de viés pode ser tão forte que pouco esforço é necessário para manter o \textit{status quo} e impedir a atenção a questões importantes \cite{schattschneider1975semisovereign}.
    
    A literatura existente sobre a influência de grupos de interesse não aborda de forma completa a questão da formação da agenda política. Embora reconheça o papel desses grupos em introduzir novas questões na agenda, ela não examina a influência mais profunda que esses grupos exercem na definição dos temas que serão discutidos. Essa "segunda face do poder", que molda a própria agenda, permanece invisível para os pesquisadores pluralistas. A abordagem pluralista, que se concentra no exercício de influência sobre questões já presentes na agenda, não é adequada para lidar com essa questão mais profunda da formação da agenda.

Além da discussão sobre o \textit{locus} do poder - se na formação da agenda, como defendida pelos objetivistas, se na tomada de decisão, como defendida pelos subjetivistas -, há discussões sobre a natureza dos recursos mobilizados.

    Os recursos de poder formam o que Simon (\citeyear{simon_notes_1953}) denominou de "bases de poder". A reputação dos agentes aparece frequentemente na literatura como uma das bases de poder \cite{Ibenskas2021, simon_notes_1953}. Uma abordagem comum é, por meio de questionários, identificar os atores com maior reputação e, portanto, maior influência política \cite{thomas2004interest}. Consequentemente, é comum encontrar evidências de que os interesses empresariais exercerem enorme influência nas decisões políticas \cite{schlozman1986organized, schlozman1984accent}.

    Recursos, contudo, nem sempre se traduzem em influência. Baldwin (\citeyear{baldwin1971money}) argumenta que a associação entre poder e dinheiro não é clara. Poder e dinheiro são recursos que não necessariamente podem ser transformados um no outro facilmente. Dinheiro, por um lado, uma vez gasto se extingue. Poder político, por outro lado, não é extinguido uma vez utilizado. São recursos, portanto, que operam de maneiras distintas.

Outra discussão que pode ser encontrada na literatura é sobre as vias em que a influência pode ocorrer. Baldwin (\citeyear{banfield1961}) identificou uma série de vias pelas quais a influência pode ser exercida:
\begin{itemize}
    \item Influência por obrigação: baseada na autoridade ou respeito, a pessoa influenciada sente que deve seguir a orientação do influenciador. Exemplos incluem relações hierárquicas no trabalho ou respeito a figuras de autoridade como pais ou professores;
    \item Influência por gratificação: o influenciador busca satisfazer os desejos da pessoa influenciada, seja por amizade, benevolência ou outros motivos. A pessoa influenciada age de determinada forma para agradar o influenciador;
    \item Influência por persuasão racional: o influenciador utiliza argumentos lógicos e informações para convencer a pessoa influenciada. A persuasão se baseia na razão e na evidência, buscando mudar a opinião da pessoa influenciada de forma racional;
    \item Influência por manipulação: o influenciador altera a percepção da pessoa influenciada sobre as alternativas de comportamento disponíveis ou sobre a avaliação dessas alternativas, sem utilizar a persuasão racional. Isso pode ser feito por meio de técnicas de venda, sugestão, fraude ou decepção, buscando influenciar a decisão da pessoa sem que ela perceba a manipulação; e
    \item Influência por coerção ou indução: o influenciador muda as alternativas de comportamento objetivamente disponíveis para a pessoa influenciada. A coerção impede a pessoa de escolher uma alternativa indesejada pelo influenciador, enquanto a indução a leva a escolher a alternativa preferida pelo influenciador, seja por meio de incentivos positivos (recompensas) ou negativos (punições).
\end{itemize}

A literatura sobre lobby foca extensivamente na última via (coerção ou indução) \cite{lowery_lobbying_2013}. Há, porém, trabalhos sobre lobby que também focalizam no terceiro tipo - persuasão racional \cite{bauer1963, milbrath1963, burnstein2002, kersh2002}. 

A multiplicidade de caminhos de influência pode dificultar a observação do fenômeno da influência do lobby. Por exemplo, a abordagem de Yackee (\citeyear{yackee2006sweet}) e Kluver (\citeyear{kluver2015legislative}) para medir a influência baseia-se na suposição de que as propostas iniciais representam as preferências sinceras das agências reguladoras \cite{yackee2006sweet} ou da Comissão Europeia \cite{kluver2015legislative}. No entanto, se o lobby prévio opera por meio de persuasão racional, as propostas iniciais podem estar mais alinhadas com as dos grupos de interesse, e não haveria nada a observar usando a metodologia deles \cite{lowery_lobbying_2013}.

Outro desafio na observação de influência consiste no problema do contrafatual. Realmente B não queria o que A desejava? Como muitos estudos focam em apenas uma decisão, pouco se pode afirmar o que teria acontecido caso não houvesse pressão por parte do lobby \cite{lowery_lobbying_2013}.
    
Mesmo superando-se as dificuldades acima, ainda restam pelo menos outras duas apontadas por Simon (\citeyear{simon_notes_1953}): o problema do feedback simétrico e o das reações antecipadas.

    O primeiro refere-se ao pressuposto de que a influência ocorre em apenas uma direção. Isto é, B não influencia A. Esse pressuposto, porém, parece improvável \cite{lowery_lobbying_2013}. Uma vez que o acesso é um recurso escasso \cite{hall1990buying, chin2000foot} e que a política não se limita a penas um tema, pode haver influência de B sobre A em diversos temas para além daquele sendo observado \cite{peterson1992presidency}. Dessa forma, torna-se difícil assumir que as preferências tanto de A quanto de B sejam genuínas \cite{lowery_lobbying_2013}.

    O segundo desafio, o das reações antecipadas, é destacado quando consideramos que nem as decisões oficiais nem os interesses organizados tomam posições no vácuo. Isto é, A pode assumir determinada posição na tentativa de antecipar a posição de B, o que enfatiza um problema de endogeneidade complexo de ser resolvido. Tais reações antecipadas podem, permitindo a influência recíproca, funcionar em ambos os sentidos com o ator B - o influenciado - assumindo uma postura inicial dura em antecipação ao lobby de A. Para Lowery (\citeyear{lowery_lobbying_2013}), muito do que legitimamente deveria ser rotulado como influência provavelmente ocorre por meio dessa moldagem das posições iniciais de barganha por reações antecipadas, que é essencialmente invisível para a maioria das pesquisas sobre lobby. Isso é mais facilmente visto no trabalho de Yackee (\citeyear{yackee2006sweet}) e Kluver (\citeyear{kluver2015legislative}). O problema das reações antecipadas é ainda mais grave na maioria das situações de lobby, dada a complexidade institucional do processo político.

Pensando nessas dificuldades metodológicas, Lowery (\citeyear{lowery_lobbying_2013}) sugeriu um catálogo de hipóteses nulas que podem ocorrer nos estudos sobre lobby no que tange a poder e influencia:
\begin{itemize}
    \item A hipótese "por trás do véu": a influência pode estar oculta devido à nossa atenção limitada a modos específicos de influência em decisões reais \cite{bachrach1962two, schattschneider1975semisovereign};
    \item A hipótese da "agenda lotada": uma proposta política pode falhar por não conseguir espaço na agenda política, não apenas por perder para a oposição \cite{baumgartner2010agendas, keim1988efficacy};
    \item A hipótese do "outro cara fez isso": resultados nulos em tentativas de exercer influência podem ser devido à falha em adotar estratégias e táticas adequadas, especialmente em situações com múltiplos atores envolvidos \cite{kollman1998outside, beyers2013policy, marshall2010lobby, binderkrantz2012customizing};
    \item A hipótese dos "dois lados": como as decisões políticas geralmente se resumem a dois lados, o sucesso de um lado implica o fracasso do outro. Resultados nulos de um lado explicam o sucesso do outro \cite{heinz1993hollow, baumgartner2009lobbying, mahoney2008brussels};
    \item A hipótese do "status quo": o \textit{status quo} é forte e difícil de mudar. O sucesso em manter o \textit{status quo} pode ser um "não evento", dificultando a avaliação da influência \cite{baumgartner2009lobbying};
    \item A hipótese do "buraco negro": o sucesso em influenciar em um momento pode dificultar a influência futura, pois altera o \textit{status quo}. Um pequeno desequilíbrio inicial pode levar à dominação completa de um conjunto de interesses \cite{anderson2004mayflies, berkhout2011short};
    \item A hipótese do "grau de dificuldade": nem todos os objetivos de lobby são iguais em termos de dificuldade. O grau de dificuldade de um objetivo pode influenciar o sucesso ou fracasso do lobby \cite{clarkterry1968community};
    \item A hipótese "nem sempre se trata de vencer": organizações de interesse podem ter outros objetivos além da influência política, como mobilização e sobrevivência \cite{grote2003europeanization, lowery2007organized}. Resultados nulos em uma questão podem representar sucesso em outras áreas \cite{pfeffer2015external};
    \item  A hipótese do "viés de seleção": a literatura sobre lobby pode ter resultados nulos por estar buscando evidências nos lugares errados, com designs de pesquisa que ignoram o contexto crucial para a influência \cite{baumgartner1998basic};
    \item A hipótese do "projeto de lei ordenhador": políticos podem extorquir fundos de campanha de organizações de interesse, introduzindo projetos de lei que os obrigam a pagamentos visando proteção contra o projeto. Invertendo, portanto, a lógica do lobby \cite{mueller1986interest, coughlin1990electoral, shughart1986growth};
    \item A hipótese do "agente sem princípios": lobistas podem agir de forma independente e não no melhor interesse de seus clientes, levando a resultados nulos em tentativas de influência \cite{kersh2002, stephenson2010lobbyists}; e
    \item A hipótese pluralista: a teoria pluralista sugere que o lobby é limitado e se concentra em fornecer informações técnicas aos funcionários eleitos, em vez de exercer táticas de pressão. Os resultados políticos refletem a vontade do público, possivelmente ponderada pela relevância da questão \cite{dahl2005governs}.
\end{itemize}

Como visto, poder e influência são conceitos diferentes, porém normalmente tratados como sinônimos - principalmente nos trabalhos empíricos. Influência é o termo mais utilizado pela literatura por ser menos rigoroso do que o de poder. Importante dizer que, ao longo deste trabalho, utilizo mais o conceito de influência.

Além disso, a literatura sobre lobby é bastante citada no contexto dessas discussões sobre poder e influência. Nesse sentido, a próxima seção, \ref{section:lobby}, detalha mais as definições de lobby, como são medidos seus efeitos e, por fim, focalizo no contexto da \acrshort{ue}.