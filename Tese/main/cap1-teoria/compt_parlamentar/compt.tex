O comportamento parlamentar envolve uma série de ações sociais. Podemos dividi-las em três tipos \cite{behm_sheep_2019}: (\textit{i}) legislativas, que focam na formulação de políticas e na busca de construção de maiorias favoráveis a determinada iniciativa política; (\textit{ii}) atividades de fiscalização, que estão ligadas à capacidade de os parlamentares obterem informações e fiscalizarem demais órgãos; e (\textit{iii}) atividades de publicidade, que objetivam comunicar aos seus eleitores seus posicionamentos. 

Uma ampla gama de atividades recebem atenção da literatura, tais como autoria de proposições legislativas \cite{crisp2007incentives, gagliarducci2011electoral,mancuso2023business}, emendas orçamentárias \cite{kerevel2015pork}, discursos em plenária  \cite{proksch2012institutional}, questionamentos parlamentares \cite{fernandes2019impact, russo2021mps}, participação em determinadas comissões  \cite{crisp2007incentives,stratmann2002plurality}, troca de partido político \cite{klein2018personal}, dissenso em votações  \cite{carey2007competing, sieberer2010behavioral}, entre outros.

A literatura sobre comportamento parlamentar é fortemente influenciada pela abordagem racionalista, que analisa as ações dos legisladores como um cálculo para maximizar interesses próprios, especialmente a reeleição \cite{mayhew2004congress}. Embora essa perspectiva seja central, o comportamento parlamentar é multifacetado, moldado também por outros incentivos, como a progressão na carreira política e o desejo de influenciar políticas públicas, como veremos a seguir.

É importante ressaltar que maximizar as chances de reeleição não é sinônimo de maximizar a quantidade de votos, mas sim de obter o número suficiente para garantir a reeleição \cite{mayhew2004congress}. Nesse sentido, os parlamentares tendem a ser ativos no parlamento, sinalizando aos eleitores seu trabalho em prol destes e, consequentemente, aumentando suas chances de reeleição.

Parte da literatura, porém, ressalta que tais incentivos somente existem em determinadas configurações institucionais, particularmente em sistemas eleitorais personalistas \cite{aleman2009comparing, kessler1996dynamics, krehbiel1995cosponsors} ou com forte voto de preferência \cite{brauninger2012personal}.

Mesmo na ausência desses mecanismos, a atividade parlamentar não cessa. Outros incentivos também são encontrados na literatura  \cite{strom1997rules}, tais como o de re-seleção e promoção \cite{martin2014electoral, rahat2001candidate,shomer2009candidate};  e instituições parlamentares, como partidos políticos e comitês legislativos, nas quais o parlamentar é socializado \cite{andeweg2011pathways, asher1973learning, louwerse_personalised_2016}.

Assim, os parlamentares podem objetivar tanto maximizar a chance de reeleição (comportamento do tipo \textit{vote-seeking}), quanto a re-seleção, ou renomeação, progressão na carreira política (\textit{carrer-seeking}) por meio de nomeação a cargos partidários - liderança partidária - ou cargos legislativos - lideranças interpartidárias - \cite{strom1997rules} e elaboração de políticas públicas (\textit{policy-seeking}) \cite{muller1999}

As atividades parlamentares voltadas para a busca por votos, ou \textit{vote-seeking}, são aquelas que visam aumentar a visibilidade e a popularidade do parlamentar junto ao eleitorado. Mayhew (\citeyear{mayhew2004congress}) propôs uma tipologia dessas atividades dos congressistas norte-americanos, classificando-as em três categorias: (\textit{i}) propaganda, (\textit{ii})posicionamento político e (\textit{iii}) reivindicação de crédito. 

As duas primeiras categorias, propaganda e posicionamento político, visam aumentar a visibilidade e a imagem positiva do congressista perante seu eleitorado, sem a necessidade de realizações políticas concretas. Em contrapartida, a reivindicação de crédito exige que o parlamentar participe ativamente da produção de resultados políticos desejáveis para seus eleitores e que consiga se apresentar como o responsável por tais resultados, o que é desafiador em um Congresso com muitos membros. Para superar esse desafio, a distribuição de benefícios particularizados permite que o parlamentar vincule seu nome a uma medida de forma crível, incentivando-o a buscar a aprovação de legislação ou outros resultados políticos que beneficiem seu distrito ou grupos específicos nele localizados.

Esses incentivos eleitorais são fortemente dependentes do sistema eleitoral. Diferentes sistemas eleitorais produzem diferentes incentivos e restrições para os parlamentares, moldando suas estratégias de busca por votos e suas prioridades políticas. 

Samuels (\citeyear{samuels1997determinantes}) argumenta que os sistemas de voto único não-transferível, de voto único transferível e de representação proporcional de lista aberta favorecem o voto pessoal. Nesses sistemas, a capacidade do parlamentar de construir uma base eleitoral pessoal é fundamental para sua reeleição. Em sistemas majoritários, por exemplo, parlamentares que estão mais vulneráveis eleitoralmente têm mais chance de apresentarem projetos \cite{bowler2010private}

Ibenskas (\citeyear{Ibenskas2021}) discute a reeleição dos \acrshort{mpe}s, enfatizando sua dependência dos eleitores de seus países de origem e dos partidos nacionais. O estudo sugere que os \acrshort{mpe}s, especialmente sob os sistemas de lista flexível, lista aberta ou voto único transferível, buscam se envolver com organizações que aumentam seu apelo eleitoral.

É importante diferenciar os efeitos do sistema eleitoral quanto à reeleição (explicado anteriormente) e quanto à seleção. Viganò (\citeyear{vigano2024electoral}) estudou os efeitos da reforma eleitoral italiana de 2005, que mudou de um sistema misto para um sistema proporcional, na atenção dos parlamentares reeleitos a questões locais. Os resultados mostram que os parlamentares anteriormente eleitos em distritos de membro único não diminuíram significativamente sua atenção a questões locais após a reforma, indicando que os incentivos eleitorais sozinhos não são suficientes para modificar o comportamento dos parlamentares, e que os efeitos de seleção devem ser também considerados.

Um outro aspecto que deve ser mencionado é que o tamanho e a diversidade socioeconômica do eleitorado de determinado parlamentar também impacta seu comportamento legislativo. Willumsen, Stecker e Goetz (\citeyear{willumsen_electoral_2019}) encontraram evidências de que há uma relação positiva entre o tamanho do distrito e a quantidade de questões e de proposições de senadores australianos. A diversidade do eleitorado, entretanto, está negativamente relacionada com a atividade legislativa \cite{willumsen_electoral_2019}.

Além dos incentivos eleitorais e institucionais, outros fatores também influenciam o comportamento parlamentar individual. A participação em comitês legislativos, a formação profissional, o partido político e o lobby são alguns dos fatores que podem moldar as ações dos parlamentares. Tais fatores estão relacionados com os objetivos de progressão na carreira (\textit{carreer-seeking}) e formulação de políticas (\textit{policy-seeking}).

O trabalho do parlamentar em comissões está correlacionado com esses dois objetivos \cite{daniel2015career}. A presença em comissões fomenta a possibilidade de formulação de políticas. O sucesso na realização de objetivos políticos leva a novas funções e cargos de poder, de modo que ambos os objetivos estão relacionados \cite{daniel2015career}. 

A expertise, experiência e lealdade partidária são fatores importantes que impactam na capacidade de o parlamentar não só participar em comissões, mas também concretizar seus objetivos de carreira e de políticas públicas \cite{chiru2020loyal}

A presença em comissões também é um indiciativo da agenda legislativa do parlamentar. Como as comissões normalmente se organizam em torno de temas, os parlamentares podem buscar participar daquelas cujos temas estão correlacionados aos seus objetivos políticos \cite{schiller1995senators}.

Um indicativo importante desses objetivos políticos do parlamentar está ligado ao seu \textit{background} ocupacional \cite{damgaard1980dilemma}. Não só sua ocupação prévia está relacionada com a suas áreas de interesses e expertise, mas também com a sua auto-seleção para comissões \cite{hamm2011committee, mcelroy2006committee, yordanova2009rationale} e com o foco da sua agenda legislativa \cite{burden2015personal}.

O partido político ao qual o parlamentar pertence também é um fator determinante de seu comportamento. A ideologia partidária, a disciplina partidária e a posição do partido no governo influenciam as posições dos parlamentares em relação a diferentes temas, suas alianças políticas e suas estratégias legislativas. Um comportamento que vá de encontro ao estabelecido pelo partido pode ser sancionado com diferentes mecanismos, tais como pelo controle de seleção de candidatos que vão concorrer a alguma eleição \cite{carey2007competing, hix2004electoral} e recompensas e sanções da liderança \cite{bressanelli2016impact}. A importância da filiação partidária também foi encontrada por Santos (\citeyear{santos2011parlamento}) no caso brasileiro.

No contexto do \acrshort{pe}, vale destacar, não há regras específicas que formalizem a obrigação de os parlamentares seguirem os seus respectivos partidos. Contudo, as lideranças partidárias mesmo assim tendem a sancionar comportamentos contrários à direção partidária por meio dos mecanismos mencionados acima \cite{carey2007competing, hix2004electoral, bressanelli2016impact}.

Outro aspecto relacionado aos partidos que afetam o comportamento parlamentar que deve ser mencionado é a ideologia partidária. Entendida como um conjunto de regras institucionais informais \cite{helmke2012informal}, a ideologia partidária pode ser compreendida como um conjunto de regras e valores que guiam o comportamento parlamentar \cite{hix2004electoral}. Parlamentares de partidos considerados como verdes, alternativos e libertários tendem a publicizar mais seus contatos com grupos de interesse, por exemplo, quando comparados a parlamentares de partidos tradicionalistas, autoritários e nacionalistas \cite{font_legislative_2023}.

O status do partido no governo também é um fator que deve ser levado em conta ao se estudar o comportamento parlamentar. Os partidos podem tornar-se \textit{veto players}, influenciando sua capacidade de negociar com o governo \cite{tsebelis2002veto}. Assim, o potencial de políticas que modifiquem o \textit{status quo} é menor quando partidos da oposição fazem parte do processo de tomada de decisão \cite{ganghof2006government}. 

\subsection{Um modelo gráfico causal para o comportamento parlamentar}

Com base na revisão da literatura apresentada, esta seção propõe um modelo teórico integrado para a análise do comportamento parlamentar e dos efeitos do \textit{lobby}. O modelo é formalizado por meio de um \acrfull{dag}, representado na Figura \ref{fig:framework}. O uso de DAGs como ferramenta analítica para explicitar pressupostos causais e guiar estratégias de identificação em pesquisas observacionais tornou-se prática consolidada na ciência política e na economia empírica. Conforme ressaltam Keele, Stevenson e Elwert (\citeyear{keele2020causal}), os DAGs permitem aos pesquisadores formalizar suas teorias causais, identificar potenciais vieses de variáveis omitidas e derivar as condições necessárias para a identificação de efeitos causais.

A Figura \ref{fig:framework} representa o sistema de relações causais que molda a atuação dos legisladores, sintetizando os principais achados da literatura discutida nas seções anteriores. Cada nó do diagrama representa uma variável ou um conjunto de variáveis teoricamente relevantes, e cada seta indica uma relação causal direta postulada pela teoria. A ausência de seta entre dois nós implica a ausência de efeito causal direto entre eles, condicional às demais variáveis do modelo. As setas pontilhadas indicam relações causais envolvendo variáveis latentes, isto é, construtos teoricamente importantes, mas não diretamente observáveis nos dados.

    \begin{figure}[htbp]
        \centering
        \caption{\textit{\acrshort{dag}} de análise do comportamento parlamentar e dos efeitos do lobby}
        \includegraphics[width=\textwidth]{imgs/DAG_v2.png}
        \label{fig:framework}
        
        \caption*{Fonte: o autor (2025)}
    \end{figure}

\subsubsection{A relação causal de interesse: lobby e comportamento parlamentar}

A relação central de interesse desta tese é o efeito causal do \textit{lobby} sobre o comportamento parlamentar, representada pela seta azul que conecta o nó ``Lobby'' ao nó ``Comportamento Parlamentar''. Esta relação está fundamentada na teoria de que grupos de interesse buscam influenciar a atuação dos legisladores por meio de contatos diretos, transferência de informações e subsídios legislativos \cite{Hall2006}. O mecanismo causal subjacente é o de troca: lobistas oferecem recursos valiosos — informações técnicas, expertise política, apoio eleitoral — em troca de atenção, acesso e, potencialmente, alinhamento comportamental \cite{huwyler_no_2023, de_figueiredo_advancing_2014}.

Entretanto, a identificação empírica dessa relação causal é desafiadora precisamente porque não ocorre em um vácuo. O diagrama explicita que múltiplos fatores influenciam simultaneamente tanto a atividade de \textit{lobby} quanto o comportamento parlamentar, configurando potenciais fontes de viés de variáveis omitidas.

Os nós ``Contexto'' e ``Parlamento'' representam fatores de confusão (\textit{confounders}) de nível macro que afetam tanto o lobby quanto o comportamento parlamentar, caracterizando o que a literatura de inferência causal denomina \textit{backdoor paths} — caminhos espúrios que podem gerar associação estatística entre tratamento e resultado na ausência de efeito causal genuíno.

O ``Contexto'' abrange o ambiente político, econômico e social mais amplo em que a atividade legislativa se insere. Crises econômicas, mudanças na opinião pública, eventos de grande repercussão midiática ou ciclos eleitorais podem simultaneamente intensificar a mobilização de grupos de interesse e alterar a agenda legislativa dos parlamentares, gerando correlação espúria entre lobby e comportamento. Por exemplo, uma crise ambiental pode levar ONGs a intensificarem seu lobby sobre parlamentares ao mesmo tempo em que estes, respondendo à pressão pública, aumentam sua atividade legislativa na área sem que uma relação causal direta exista necessariamente.

O ``Parlamento'' refere-se às regras, estruturas e dinâmicas institucionais da própria casa legislativa. A agenda legislativa definida pela liderança, os prazos de tramitação de proposições específicas e a estrutura das comissões podem afetar simultaneamente quais parlamentares são alvo de lobby e como estes se comportam. Um relatório legislativo importante em tramitação, por exemplo, atrai a atenção de lobistas para o relator designado e, ao mesmo tempo, induz este parlamentar a ser mais ativo no tema — novamente configurando uma fonte potencial de correlação espúria.

O comportamento parlamentar, conforme discutido ao longo deste capítulo, é determinado por um conjunto amplo de fatores que podem ser agrupados em três dimensões principais, representadas no diagrama pelos blocos ``País'', ``Partido'' e ``Aspectos Individuais''. Esta categorização integra contribuições de diferentes tradições teóricas no estudo do comportamento legislativo.

A dimensão do ``País'' abrange o sistema eleitoral e as características do eleitorado, fatores que criam incentivos estruturais para determinados padrões de comportamento parlamentar. Como amplamente documentado pela literatura institucional \cite{mayhew2004congress, samuels1997determinantes, carey2007competing}, sistemas eleitorais que favorecem o voto pessoal induzem parlamentares a buscar visibilidade individual, enquanto sistemas de lista fechada tendem a fortalecer a disciplina partidária. As características do eleitorado — tamanho, diversidade socioeconômica, perfil ideológico — também moldam os incentivos para busca por votos (\textit{vote-seeking}) \cite{willumsen_electoral_2019}. No contexto do \acrshort{pe}, esta dimensão é particularmente relevante porque os parlamentares pertencem a 27 países distintos, cada um com seu próprio sistema eleitoral e configuração de eleitorado, gerando variação substancial nos incentivos enfrentados.

O partido político influencia o comportamento parlamentar por meio de dois mecanismos principais: ideologia e disciplina. No diagrama, essas variáveis são representadas com setas pontilhadas, indicando sua natureza latente. A ideologia partidária funciona como um conjunto de regras informais que orientam o posicionamento dos parlamentares em diferentes temas \cite{hix2004electoral, helmke2012informal}. A disciplina partidária opera por meio de mecanismos de sanção e recompensa — controle sobre a seleção de candidatos, distribuição de cargos legislativos, acesso a recursos de campanha — que induzem os membros a seguirem as orientações da liderança \cite{carey2007competing, bressanelli2016impact}. O fato de ideologia e disciplina serem construtos latentes implica que sua mensuração é indireta e sujeita a erro, uma limitação que deve ser considerada na estratégia empírica.

Os aspectos individuais do parlamentar constituem a terceira dimensão determinante do comportamento. A expertise, frequentemente derivada da ocupação prévia do legislador, influencia tanto suas áreas de atuação prioritárias quanto sua capacidade de formular políticas efetivas \cite{burden2015personal, damgaard1980dilemma}. A experiência política — tempo de mandato, cargos ocupados, trajetória partidária — afeta a progressão na carreira (\textit{career-seeking}) e a habilidade de navegar o processo legislativo \cite{daniel2015career, chiru2020loyal}. Outras características individuais, como gênero, idade e formação educacional, são representadas como variáveis latentes por serem potencialmente relevantes, mas nem sempre diretamente observáveis ou mensuráveis com precisão nos dados disponíveis.

Esta estrutura tridimensional é consistente com achados empíricos da literatura. Mancuso (\citeyear{mancuso2023business}), estudando o comportamento parlamentar brasileiro, encontrou que a presença em comitês (aspecto individual), a ideologia (aspecto partidário) e o tempo no cargo (experiência individual) são fatores estatisticamente significativos para explicar a produção de projetos de lei favoráveis ao empresariado. No caso do \acrshort{pe}, a dimensão do país adquire relevância especial por constituir fonte de variação substantiva dentro do próprio parlamento — diferentemente de legislaturas nacionais onde todos os membros compartilham o mesmo sistema eleitoral.

O DAG apresentado tem implicações diretas para a estratégia de identificação empírica dos efeitos do \textit{lobby}. Ao explicitar as relações causais postuladas pela teoria, o diagrama permite identificar os caminhos de confusão (\textit{backdoor paths}) que precisam ser bloqueados para obter estimativas não enviesadas do efeito causal de interesse.

A principal fonte de viés potencial reside nos fatores de confusão que afetam simultaneamente o lobby e o comportamento parlamentar. Os nós ``Contexto'' e ``Parlamento'' representam choques agregados que podem gerar correlação espúria. As três dimensões do comportamento parlamentar — país, partido e aspectos individuais — também configuram potenciais fontes de viés na medida em que características do parlamentar que determinam seu comportamento podem também afetar a probabilidade de ser alvo de lobby. Por exemplo, um parlamentar com expertise em tecnologia pode ser mais propenso a apresentar projetos na área \textit{e}, simultaneamente, mais procurado por lobistas do setor — sem que exista necessariamente relação causal direta entre o lobby recebido e seu comportamento.

A estratégia de identificação adotada nesta tese, detalhada no Capítulo \ref{chapter:metodologia}, busca bloquear esses caminhos de confusão por meio de uma estrutura de efeitos fixos de alta dimensão. Efeitos fixos de país-tempo absorvem choques comuns a parlamentares do mesmo país em cada período, controlando por fatores relacionados ao sistema eleitoral e ao eleitorado. Efeitos fixos de partido-tempo capturam variações na disciplina e nas estratégias partidárias ao longo do tempo. Efeitos fixos de domínio temático-tempo absorvem flutuações na saliência de temas específicos. Por fim, a exploração de variação \textit{within-individual} ao longo do tempo permite controlar características estáveis dos parlamentares, observáveis ou não.

É importante ressaltar as limitações deste \textit{framework}. Primeiro, o DAG assume que não há causalidade reversa direta — isto é, que o comportamento parlamentar não causa lobby contemporaneamente. Embora o comportamento possa sinalizar receptividade e atrair lobby futuro \cite{wright1996contributions}, assumimos que, dentro de um mesmo período, a direção causal predominante vai do lobby para o comportamento. Esta premissa será testada empiricamente por meio de análises de leads e lags. Segundo, o modelo não representa explicitamente todos os mecanismos pelos quais o lobby pode afetar o comportamento — transferência de informações, subsídios legislativos, construção de relações de longo prazo —, tratando-os como componentes de um efeito agregado. Terceiro, variáveis latentes como ideologia e disciplina partidária são mensuradas indiretamente, introduzindo potencial erro de mensuração.

Apesar dessas limitações, o \textit{framework} proposto oferece uma fundamentação teórica explícita para a análise empírica, integrando contribuições dispersas na literatura e formalizando as premissas necessárias para a identificação causal. Ao explicitar as relações postuladas e os potenciais vieses, o DAG permite uma avaliação mais transparente da validade das inferências causais realizadas.

É nesse cenário complexo de interações simultâneas que os grupos de interesse buscam exercer influência. Conforme ilustrado no DAG, os lobistas operam em um ambiente onde múltiplos fatores determinam o comportamento dos parlamentares que pretendem influenciar. Para serem eficazes, devem desenvolver estratégias que considerem não apenas as preferências e posições individuais dos legisladores, mas também os incentivos estruturais derivados do país de origem, a dinâmica partidária e o contexto institucional mais amplo.

A literatura sugere que lobistas sofisticados selecionam seus alvos estrategicamente, priorizando parlamentares que já possuem alguma predisposição favorável — seja por expertise no tema, posição em comitês relevantes ou alinhamento ideológico \cite{Hall2006}. Essa seleção não aleatória, embora racional do ponto de vista dos grupos de interesse, constitui precisamente o desafio metodológico central para a identificação de efeitos causais: parlamentares que recebem mais lobby diferem sistematicamente daqueles que recebem menos, e essas diferenças podem estar correlacionadas com seu comportamento legislativo por outras razões que não o próprio lobby.

Nesse sentido, os lobistas buscam estabelecer relações de troca \cite{huwyler_no_2023} a fim de impactarem o comportamento do legislador. As evidências dessa relação causal, porém, são de difícil comprovação, envolvendo uma série de dificuldades metodológicas discutidas na seção \ref{section:effects}. O capítulo \ref{chapter:ue}, a seguir, resgata o contexto institucional da \acrshort{ue} com foco especial no \acrshort{pe}.
