\subsection{Os efeitos do lobby na literatura empírica}
\label{section:effects}
A mensuração dos efeitos do lobby é desafiadora. A identificação econométrica é de difícil operacionalização e os mecanismos causais são difíceis de serem isolados.

Segundo de Figueiredo e Richter (\citeyear{de_figueiredo_advancing_2014}), a literatura sobre os efeitos do lobby tende a focalizar em alguns desses temas: comércio exterior; finanças e regulação; orçamento governamental e contratos; taxação; nomeações judiciais; políticas de migração; difusão tecnológica; tramitação de projetos de lei em geral; entre outros. Como se vê, a gama de temas é ampla, porém, normalmente, há a escolha de apenas um tema para a análise, como visto na seção \ref{section:poder_influe}. 

O estudo de vários temas é mais raro. Podemos citar, por exemplo, o trabalho de Baumgartner \textit{et al.} (\citeyear{baumgartner2009lobbying}), que estudou 98 temas diferentes ao longo de oito anos e 2.200 lobistas. Os achados do trabalho demonstraram que ambos os lados de uma disputa conseguiram mobilizar recursos semelhantes, já que há uma mistura heterogênea de empresas, cidadãos e governos em cada lado. Outro achado interessante é que o \textit{status quo} é resiliente. Ou seja, as políticas geralmente são estáveis e resilientes a mudança, porém quando ocorrem são, geralmente, substanciais.

Como mencionado anteriormente, de acordo com Mahoney (\citeyear{mahoney_lobbying_2007}), os efeitos do lobby dependem essencialmente de três dimensões: (\textit{i}) as características intrínsecas do próprio grupo de interesse; (\textit{ii}) as características específicas do tema em debate; e (\textit{iii}) o contexto institucional onde o lobby ocorre. A seguir fazemos um breve apanhado de como cada uma dessas dimensões afeta os resultados do lobby.

Com relação às características intrínsecas do grupo de interesse, a literatura destaca dois fatores: recursos e capacidade de utilização de seus recursos.

    Recursos, como visto na seção \ref{section:poder_influe}, podem ser de diferentes naturezas: informacionais, relacionais (conexões e acessos), econômicos e reputacionais. O sucesso do lobby, porém, depende da utilização eficaz dos recursos a disposição \cite{Pop2013Lobbying}. Ou seja, recursos não se traduzem automaticamente em poder e influência \cite{simon_notes_1953}.

    Na literatura empírica, há diversas fontes de recursos analisadas, tais como: recursos financeiros \cite{dur2007question, eising2007institutional}; vínculos formais com parlamentares \cite{huwyler_no_2023}, legitimidade \cite{bunea2018legitimacy}, informação fornecida aos parlamentares \cite{kluver_informational_2012}; entre outros.

    Uma outra discussão sobre a capacidade de mobilização de recursos se refere ao clássico problema da ação coletiva \cite{olson1971logic}. Quanto maior a coalização, mais difícil se torna gerenciá-la e direcioná-la. Mark Smith (\citeyear{smith2000merican}) encontrou evidências de que quando os interesses empresariais estão unidos em uma coalização há uma menor chance de seus interesses serem acatados pelo Congresso dos EUA. 

    Essa ideia, porém, não é consensual. A partir da teoria dos jogos, Ward (\citeyear{ward_pressure_2004}) defende que coalizações maiores podem angariar mais recursos de forma a aumentar suas chances de sucesso. Vale destacar, contudo, que o custo da coalização não foi captado pelo modelo proposto, o que pode ter importantes consequências para seus corolários.

    Mahoney (\citeyear{mahoney_lobbying_2007}), por outro lado, encontrou que coalizações \textit{ad hoc} não produzem efeitos sobre o sucesso do lobby. A questão das coalizações em relação ao sucesso do lobby permanece, portanto, em debate na literatura.

Com relação às características dos temas, podemos identificar pelo menos duas dimensões, a saber: saliência e diversidade de interesses.

    A saliência do tema tem relação negativa com a chance de sucesso do lobby. Os parlamentares, quando deparados com um tema de grande repercussão pública, tendem a preferir estes ao lobby direto \cite{kollman1998outside}. Assim, temas com maior saliência reduzem a chance de sucesso do lobby \cite{mahoney_lobbying_2007}. Page \textit{et al.} (\citeyear{page1987moves}) ainda encontram evidências de que não somente o lobby reduz suas chances de sucesso, mas também produz efeitos contraproducentes nesse contexto.

    A diversidade de interesses também é outro aspecto destacado pela literatura sobre os efeitos do lobby. Dado que os parlamentares possuem o tempo como recurso escasso, quanto maior a quantidade de atores interessados, maior a competição pelo acesso. De tal modo que quanto maior a diversidade de interesses, maior a dificuldade que determinados atores têm de exercer influência. Os efeitos da competição porém, são marginais no que tange ao sucesso do lobby \cite{lowery_lobbying_2013}.

O contexto institucional não apenas molda as estratégias dos atores envolvidos no lobby \cite{Pop2013Lobbying}, mas também influencia diretamente seus resultados. A estrutura e as regras do sistema político podem favorecer certos grupos ou indivíduos, criando desigualdades no acesso e na capacidade de influenciar as decisões \cite{mahoney_lobbying_2007}.
    Em sistemas eleitorais com financiamento privado de campanha, a desigualdade de recursos financeiros entre os atores pode ser um fator determinante para o sucesso do lobby. Grupos e indivíduos com maior capacidade financeira podem financiar campanhas eleitorais, contratar lobistas profissionais e realizar ações de comunicação e mobilização mais impactantes, o que aumenta suas chances de influenciar as decisões políticas\cite{mahoney_lobbying_2007}.

    Um dos principais desafios na mensuração da influência do lobby é a diversidade de canais pelos quais ela pode ser exercida. Além das estratégias tradicionais, como o contato direto com legisladores e a participação em audiências públicas, o lobby também pode ocorrer por meio de campanhas de mídia, mobilização de bases, litígios estratégicos e até mesmo a produção de conhecimento especializado para influenciar o debate público \cite{dur_measuring_2008}. Essa multiplicidade de canais torna difícil identificar e quantificar a influência de cada ator e estratégia específica.

A literatura empírica sobre a influência do lobby em políticas públicas apresenta resultados mistos e inconclusivos. Quatro grandes revisões da literatura chegaram a conclusões semelhantes \cite{smith1995interest, baumgartner1998basic,burnstein2002, de_figueiredo_advancing_2014}, indicando que o lobby pode ter forte influência, influência marginal ou nenhuma influência, dependendo do contexto.

Essa ambiguidade nos resultados pode ser explicada, em parte, pela dificuldade em definir e medir a influência do lobby. A pesquisa de Baumgartner \textit{et al.} (\citeyear{baumgartner2009lobbying}) sugere que ambos os lados de uma questão geralmente mobilizam recursos semelhantes, devido à heterogeneidade dos grupos envolvidos. Além disso, as políticas tendem a ser estáveis e resistentes à mudança, mas quando ocorrem, as mudanças são substanciais.

Esses resultados indicam que a influência do lobby não é um fenômeno simples e unidirecional. O sucesso ou fracasso do lobby depende de uma complexa interação de fatores, incluindo o contexto institucional, a natureza da questão em debate, os recursos mobilizados e as estratégias adotadas pelos atores envolvidos. A literatura ainda precisa avançar na compreensão desses mecanismos para oferecer uma resposta mais precisa sobre os efeitos do lobby nas políticas públicas. Na próxima seção, \ref{section:lobbyUE}, abordaremos os principais trabalhos sobre lobby na \acrshort{ue}.
