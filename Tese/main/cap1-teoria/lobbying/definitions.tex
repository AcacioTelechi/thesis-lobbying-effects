\subsection{Lobby: definições e alguns consensos da literatura}
\label{subsection:lobbydef}
O termo "lobby" surgiu como uma forma de designar a atividade dos porta-vozes de interesses afetados que ficavam nos saguões, ou "lobbies", de hotéis onde ficavam hospedados figuras políticas \cite{mancuso2018}. Para além do resgate etimológico, há pelo menos duas manerias de definir lobby.

A primeira, mais operacional, é aquela utilizada pelos legisladores quando da regulamentação da atividade. Segundo o \acrfull{hloga}, é considerado como lobista o indivíduo que 

\begin{quote}
"seja empregado ou contratado por um cliente em troca de compensações financeiras ou de outro tipo, por serviços que incluam mais que um contato de lobby, e cujas atividades de lobby tomem 20\% ou mais do tempo de serviço prestado por aquele indivíduo àquele cliente por um período de 3 meses" (\acrshort{hloga}, 2007, título II, seção 201, b, 1).
\end{quote}

Já na \acrshort{ue}, adota-se uma definição mais ampla. Segundo o \acrfull{airt} definiu-se como "representante de interesse" aquela pessoa singular ou coletiva, ou grupo formal ou informal, associação ou rede que "exerça atividades abrangidas", quais sejam: 
\begin{quote}
"atividades exercidas por representantes de interesses com o objetivo de influenciar a formulação ou execução de políticas ou de legislação, ou os procedimentos de tomada de decisões das instituições signatárias ou de outras instituições, órgãos e organismos da União (conjuntamente designadas 'instituições da União')" (\acrshort{airt}, 2021, art. 3º, 1).
\end{quote}

Tanto no \acrshort{hloga} quanto no \acrshort{airt}, atividades exercidas por funcionários públicos no exercício de suas funções, ou atividades distribuídas ao público ou divulgadas por meios de comunicação em massa não são consideradas como atividades de lobby, bem como apresentação de observações em processo judicial ou administrativo não são consideradas como atividades lobby. 

Uma segunda forma de definir lobby é a partir da discussão acadêmica do fenômeno. Segundo Schmitter (\citeyear{schmitter1974still}), o lobby é atividade típica de sistemas pluralistas. Uma vez que, nestes sistemas, os processos decisórios são marcados pela discussão livre de interesses divergentes. 

Lobby também pode ser definido como a transferência de informação em meios e reuniões privados entre grupos de interesses e políticos, seus assessores e agentes \cite{de_figueiredo_advancing_2014}. Informação é entendida como a representação teórica de uma mensagem, a qual pode assumir diferentes formas, tais como fatos estatísticos, argumentos, mensagens, previsões, ameaças, compromissos, sinais ou alguma combinação destes \cite{de_figueiredo_advancing_2014}.

É comum, na literatura, encontrar uma diferenciação entre lobby e \textit{advocacy}. Enquanto aquele está ligado a ação direta de grupos de interesse, este aparece como um estilo de defesa de interesses voltado à promoção de bens públicos e grandes causas sociais (como direitos humanos, meio ambiente, etc.) \cite{mancuso2018}. Essa diferença é comumente mobilizada pela literatura de movimentos sociais e em como eles mobilizam a esfera pública para promover suas causas \cite{van2024rise}.

Grupos de interesse, contudo, empregam estratégias complexas  por meio de campanhas de pressão pública \cite{hall2012targeted}. Assim, essa diferença entre lobby e \textit{advocacy} será relativizada ao longo deste trabalho, de modo que utilizaremos uma definição mais ampla de lobby como \textit{o exercício de influência sobre a atividade legistativa realizado por atores interessados em um ambiente marcado por diferenças de recursos}.

Entendo, portanto, que o lobby demanda um exercício, isto é, uma ação. Por diferentes meios, um lobista, representante de interesses de um ente privado ou um grupo de entes privados, tomam decisões, alocam e utilizam recursos ativamente a fim de  influenciar o comportamente de um parlamentar. Destaco que, para fins de operacionalização desta tese, a ação considerada são reuniões privadas entre o lobista, ou grupo de lobistas, e os parlamentares.

Para entender o fenômeno do lobby, bem como a sua influência, Mahoney (\citeyear{mahoney_lobbying_2007}) argumenta que é necessário olhar para três esferas: (\textit{i}) as características intrínsecas do próprio grupo de interesse; (\textit{ii}) as características específicas do tema em debate; e (\textit{iii}) o contexto institucional onde o lobby ocorre.

Definido o que entendo por lobby, devemos compreender que tipo de atores o fazem e como suas características intrínsecas afetam a maneria como agem. Inicialmente, as corporações e associações comerciais compreendem o maior volume de gastos em lobby. Tais gastos, nos Estados Unidos, representam mais de 84\% do total gasto por grupos de interesse em lobby no nível federal e 86\% dos gastos no nível estadual \cite{de_figueiredo_advancing_2014}.

Além disso, grandes grupos de interesse organizados e grupos que são apoiados por grandes corporações têm maior chance de fazer lobby em comparação a grupos menores \cite{de_figueiredo_advancing_2014}. Essas evidências foram encontradas em diversos setores para o caso norte-americano \cite{ansolabehere2003why, hansen2004collective, guo2009lobby, hochberg2009lobbying, richter2009lobbying, hill2013determinants}. Evidências similares também são encontradas para temas tarifários e disputas comerciais \cite{bombardini2012competition}.

Embora os interesses empresariais representam a maior parte do que é gasto em lobby, eles representam uma parte, ainda que significativa, do número de grupos de interesse que realizam lobby. Baumgartner \textit{et al.} (\citeyear{baumgartner2009lobbying}) encontraram evidências de que grupos de cidadãos, sindicatos, fundações, \textit{think tanks}, governos, instituições, dentre outros grupos representam 46\% das organizações que realizaram lobby no Congresso dos EUA entre 1999 e 2001.

Outra evidência que os estudos sobre lobby revela é que grandes corporações e grupos com muitos recursos têm maior chance de exercer pressão de maneira independente em comparação a grupos menores. Estes, por sua vez, buscam atuar em coalizações. Richter \textit{et al.} (\citeyear{richter2011good}) e Kerr \textit{et al.} (\citeyear{kerr2014dynamics}) encontraram que apenas 10\% de companhias abertas realizam lobby em seu próprio nome. A heterogneidade desses atores, contudo, não foi considerada nesses estudos. Esta tese contribui com o debate ao levar em conta as diferenças de recursos entre atores do mesmo tipo.

Empresas podem, ainda, apresentar resistências a utilizar lobistas externos quando segredos industriais e inovação estão em jogo \cite{de2001structure}. Além disso, em contextos onde há eleições diretas e financiamento privado de campanha, firmas possuem um incentivo para agirem sozinhas \cite{mahoney_lobbying_2007}. Outro fator que pode incentivar esse tipo de atuação é quando há o problema do caroneiro (\textit{free-rider}), ou seja, um agente que se beneficiaria de determinada decisão sem ter de arcar com os custos de pressionar para a sua efetivação \cite{bombardini2012competition}.

A diversidade de atores e estratégias no lobby é um tema complexo e multifacetado. A quantidade e variedade de grupos de interesse variam ao longo do tempo, impulsionadas pela legitimação de causas e pela competição por recursos \cite{lowery2007organized} Além disso, fatores como o tamanho da economia, a quantidade de questões legislativas \cite{baumgartner2009lobbying, lowery_lobbying_2013} e a diversidade de interesses dos eleitores \cite{berkman2001legislative} também influenciam essa dinâmica.

Uma outra regularidade encontrada pelos estudos empíricos sobre lobby é a de que há uma relação positiva entre a quantidade de lobby e a saliência e relevância de temas. Aqueles temas, portanto, que ocupam maior espaço no debate público tendem a receber maior atenção de lobistas \cite{caldeira2000lobbying,  baumgartner2010agendas, bonardi2005corporate}.

O maior interesse dos lobistas em temas mais salientes, entretanto, não se traduz em maiores chances de sucesso. Pelo contrário, como argumenta Mahoney (\citeyear{mahoney_lobbying_2007}), quanto maior a saliência do tema, menores tendem a ser as chances de sucesso de lobistas individualmente. Uma das interpretações da autora é que quanto maior a saliência temática, maior é a chance de haver oposição e interesses divergentes, de tal maneira a reduzir a probabilidade de sucesso. Tal proposição ainda carece de maior estudo pela literatura.

O contexto institucional também é um fator que deve ser levado em conta. As regras institucionais moldam as estratégias dos atores. Há certo consenso de que os legisladores com maior poder de agenda são os alvos preferenciais dos lobistas \cite{de_figueiredo_advancing_2014}. Ora, esse poder de agenda está relacionado com as regras de cada parlamento, por exemplo. Além disso, o \textit{locus} de atuação normalmente se dá em comissões influentes em determinados temas como orçamento e finanças \cite{hojnacki2001pac, drope2004purchasing}. Lideranças parlamentares, como presidentes de comissões, líderes da maioria ou da minoria, também são alvos prioritários \cite{evans1996before}.

Ainda sobre a questão dos alvos prioritários do lobby, há um consenso crescente de que parlamentares tanto aliados \cite{kollman1998outside, caldeira2000lobbying, hojnacki2001pac} quanto marginais \cite{holyoke2003choosing, kelleher2009political, bertrand2014whom, gawande2012lobbying} são procurados por lobistas. Inimigos ferrenhos, porém, tendem a ser ignorados  \cite{de_figueiredo_advancing_2014}

No entanto, a pesquisa de Austen-Smith e Wright (\citeyear{austen1996theory}) sobre indicações para a Suprema Corte dos EUA oferece uma visão diferente. Eles argumentam que lobistas visam legisladores marginais para influenciá-los a mudar de posição e aliados para contra-atacar o lobby da oposição. Essa abordagem, chamada de "lobby contra-ativo", sugere que a influência se concentra em mudar a posição dos indecisos e fortalecer o apoio dos aliados.

Em contraste, Hall e Miler (\citeyear{Hall2006}) e Hojnacki e Kimball (\citeyear{hojnacki2001pac}) criticam essa visão e defendem que os aliados são os principais alvos, seguidos pelos marginais. Eles argumentam que os lobistas buscam influenciar a legislação por meio de seus aliados e incentivá-los a mobilizar outros legisladores. Essa abordagem sugere que o lobby se concentra em construir coalizões e fortalecer o apoio em torno de uma determinada posição.

Outro debate presente na literatura é sobre o que seria mais importante: quem você conhece ou o que você sabe. Isto é, um debate sobre qual recurso é mais valioso: conexão ou expertise. Empiricamente, é possível encontrar lobistas que buscam se especializar em certos temas \cite{de_figueiredo_advancing_2014}, sugerindo que a expertise seria mais valiosa do que as conexões \cite{bertrand2014whom}.

Blanes i Vidal \textit{et al.} (\citeyear{blanes2012}), porém, encontraram que a receita de lobistas que eram assessores de parlamentares que perderam o cargo reduzia, em média, 23\% após a perda do cargo. Bertrand \textit{et al.} (\citeyear{bertrand2014whom}) também encontraram evidências de que os lobistas tendem a seguir os políticos com os quais possuem conexão (mesmo que troquem de comissão ou atuem em temas divergentes do seu interesse). Essas evidências, portanto, sugerem que a conexão, ou "quem você conhece" é tão relevante, ou mais, do que "o que você sabe", ou a expertise dos lobistas.

Para o modelo utilizado nesta tese, consideramos que ambos os fatores são importante. Devemos, porém, separar o fenômeno em dois efeitos. O primeiro é o \textit{acesso}. O lobista deve conseguir, em primeiro lugar, acesso ao parlamentar. O acesso reflete a capacidade de troca de informação. O acesso, contudo, não garante a influência. É condição necessária, mas não suficiente. A influência surge da \textit{persuasão}, o segundo elemento para a influência ocorrer. A persuasão reflete a capacidade de o lobista alterar o comportamento parlametar. A combinação de acesso e persuasão geram o nosso efeito de interesse - o efeito total do lobby no comportamento parlamentar.

Em suma, a questão de quem são os alvos prioritários do lobby e qual recurso é mais valioso - conexões ou expertise - ainda é objeto de debate na literatura. As evidências apontam para a importância de ambos os fatores, mas a dinâmica do lobby pode variar dependendo do contexto institucional, dos objetivos dos lobistas e das características dos legisladores. Dado esse cenário complexo de interações, a literatura empírica sobre os efeitos do lobby encontra dificuldades de comprová-los. A seguir, seção \ref{section:effects}, faremos uma discussão sobre a literatura que objetiva medir os efeitos do lobby.
