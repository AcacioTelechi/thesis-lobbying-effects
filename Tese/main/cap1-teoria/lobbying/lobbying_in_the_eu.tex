\subsection{Lobby na UE: os atores, as estratégias e as instituições}
\label{section:lobbyUE}

A literatura sobre lobby na \acrshort{ue} tem explorado diversos aspectos desse fenômeno. Contribuições recentes examinaram o regime de regulação de lobby da UE \cite{bunea2019regulating, bunea2018legitimacy}, a densidade de interesses organizados \cite{carroll2020cultural}, o acesso de grupos ao processo deliberativo do \acrshort{pe} \cite{coen2019legislative}, a influência em resultados políticos \cite{kluver2015legislative, marshall2010lobby}, o reconhecimento de grupos de interesse por \acrshort{mpe}s \cite{Ibenskas2021}, a transparência de lobistas empresariais na \acrshort{ue} \cite{naurin2007deliberation}, os mecanismos de trílogos e os processos informais de formulação de políticas \cite{brandsma2019transparency, leino2017search}, entre outros temas.

No que tange às dimensões que influenciam os efeitos do lobby, discutidas na seção \ref{section:effects}, o contexto internacional da \acrshort{ue} e a sua complexidade institucional geram efeitos particulares \cite{kluver2015legislative}. Em primeiro lugar, há diferentes sistemas políticos e eleitorais em cada membro da \acrshort{ue} que, como visto, influenciam os incentivos não só dos parlamentares no que se refere à responsividade com o eleitorado \cite{mahoney_lobbying_2007}, mas também da mobilização dos grupos de interesse \cite{kluver2015legislative}. Em segundo lugar, a estrutura em múltiplos níveis da \acrshort{ue} apresenta-nos um ambiente institucional complexo. Há diferentes instituições envolvidas no processo de tomada decisão, o que aumenta a quantidade de canais de lobby e complexidade das estratégias dos grupos de interesse \cite{richardson2000government,eising2007institutional, coen2019legislative}. 

Tradicionalmente, os estudos focam nas características individuais dos grupos de interesse, tais como seus recursos financeiros, características organizacionais, ideologia e expertise \cite{Bouwen2002, dur2007question, eising2007institutional}. Nesse sentido, os trabalhos empíricos encontram alguns padrões. Um deles é que grupos de interesse com maiores recursos financeiros, como grupos empresariais e associações nacionais, predominam no contexto da \acrshort{ue} \cite{dur20212wholobbies, eising2007institutional}.

    A força de interesses empresariais geram críticas a respeito da desigualdade de representação e, consequentemente, da legitimidade das decisões. Kohler-Koch e Quittkat (\citeyear{kohler2007desmyth}) chegam a afirmar que o ideal democrático participativo é um mito inalcançável. Hanegraaf (\citeyear{Hanegraaff2024firms}), entretanto, ao analisar a política comercial da \acrshort{ue}, defende que o lobby mais centrado em firmas (ao invés de associações nacionais) é uma função das características de cada indústria e de cada país.

    Além disso, as associações empresariais possuem maior acesso à \acrshort{ce} e aos governos nacionais. Ao passo que grupos de cidadãos possuem maior acesso aos parlamentares \cite{dur20212wholobbies}. No computo geral, as organizações empresariais representam cerca de 70\% das organizações com acesso às instituições europeias \cite{Coen2007}, o que leva os autores a classificar como um regime pluralista de elite \cite{coen1997evolution, coen1998european, Bouwen2002, schmidt2006procedural}.

    Apesar da importância dos recursos financeiros, o tipo de organização também importa. A legitimidade e representatividade, o conhecimento e a expertise e as informações dos atores são relevantes \cite{Coen2019, dur_measuring_2008}. Diferentes tipos de organizações possuem diferentes recursos e em diferentes proporções. Por um lado, associações empresariais tendem a possuir maiores informações e expertise \cite{dur20212wholobbies}. Por outro lado, interesses difusos podem favorecer o acesso a outros tipos de organizações que apresentem legitimidade nos temas debatidos, geralmente no \acrshort{pe} \cite{kluver2015legislative}. 

    Vale ressaltar, contudo, que não há consenso sobre esse ponto. Pakull \textit{et al.} (\citeyear{Pakull2020}), por exemplo, argumenta que organizações da sociedade civil não estão interessadas em acesso ao \acrshort{pe}. Já Hanegraaff e Berkhout (\citeyear{Hanegraaff2019}) encontraram evidências no sentido contrário. No estudo, houve pouca diferença nas vias de acesso à \acrshort{ue} em termos dos interesses representados. Para eles, a saliência e o escopo do tema são determinantes mais fortes para a variação na atividade do lobby empresarial.

    Apesar dessa discussão, os grupos de interesse devem frequentemente demonstrar legitimidade técnica para estar presentes nos debates, sobretudo na \acrshort{ce} \cite{Bouwen2002}. No \acrshort{pe}, contudo, diferentes formas de justificativa para essa legitimidade aparecem. Empresas farmacêuticas, por exemplo, quando atuam na \acrshort{ce}, apresentam argumentos relacionados à competitividade global, buscando patentes mais longas. Quando, porém, atuam no \acrshort{pe}, as argumentações vão no sentido do emprego e educação regionais \cite{earnshaw2002no, coen2000corporate}. Demonstrando que, no caso do \acrshort{pe}, os atores buscam apresentar aos \acrshort{mpe} argumentos que demonstrem sua capacidade de apoiar os parlamentares nos objetivos políticos: reeleição, progressão de carreira e influência na formulação de políticas \cite{Ibenskas2021}.
    
    Dessa forma, não apenas os grupos de interesse devem demonstrar sua legitimidade, mas esta deve ser reconhecida pelos parlamentares. O fato de o reconhecimento da legitimidade ser algo relevante aponta para mais um possível viés ao se estudar os efeitos do lobby.

    Essa discussão sobre as diferentes formas de os grupos de interesse se apresentarem perante a \acrshort{ue} remete-nos ao debate sobre a estratégia dos atores. Podemos investigar pelo menos dois aspectos sobre isso: em relação aos alvos escolhidos para o lobby; e aos canais e às formas de persuasão.

    Como a \acrshort{ce} possui a capacidade de formação de agenda, ela se torna o \textit{locus} principal para a atividade de lobby \cite{cram2001whither, pollack_delegation_2002}. O Tratado de Lisboa, contudo, imbuiu o parlamento de maiores poderes legislativos, como será visto no capítulo \ref{chapter:ue}, uma vez que a co-decisão virou o processo legislativo ordinário. Por conta disso, há um interesse crescente no \acrshort{pe} \cite{hix2013empowerment, Dionigi2017}.
    
    Na literatura sobre o parlamento, é possível encontrar algumas discussões sobre estratégias de lobby sobre legisladores individuais \cite{marshall2015explaining}, acesso de determinados grupos às comissões do \acrshort{pe} \cite{marshall2010lobby}, o relacionamento e o alinhamento político com grupos partidários europeus \cite{beyers2015alignment}, os níveis de sucesso do lobby \cite{Dionigi2017}, entre outros

    Coen e Katsaitis (\citeyear{coen2019legislative}) encontraram evidências de que as comissões com uma maior proporção de \acrfull{plo} em relação aos Relatórios de Iniciativa Própria observam uma maior mobilização de interesses privados. De maneira inversa, comissões das quais a razão de procedimentos é menor observam maior mobilização do público em geral.

    Os grupos de interesse buscam atuar com parlamentares com maior influência legislativa \cite{marshall2010lobby}. Essa proposição encontra diversas evidências empiricamente no caso dos \acrshort{mpe}s \cite{kluver2015legislative, marshall2010lobby}. Especificamente, os relatores e relatores-sombra são alvos prioritários dos grupos de interesse \cite{kluver2015legislative}.

Além disso, outra discussão relevante na literatura é o chamado \textit{venue-shopping}, isto é, a escolha estratégica dos canais de pressão pelos grupos de interesse. A \acrshort{ue} oferece uma estrutura de oportunidades com múltiplos pontos de acesso, como a \acrshort{ce}, o \acrshort{pe} e o \acrfull{coue} \cite{richardson2000government}.

    A natureza contextual de cada debate político é crucial para a escolha desses canais, e as características individuais dos grupos de interesse, como tipo, recursos e nível geográfico de organização, não são suficientes para explicar sozinhas os processos de lobby. A complexidade do processo decisório europeu, com múltiplas instituições e níveis de governança, leva os atores a realizarem cálculos estratégicos na escolha dos canais de pressão \cite{kluver2015legislative}.

    Grupos de interesse tendem a ser mais ativos em áreas onde o Estado também é mais ativo \cite{mahoney2008brussels}, buscando influenciar o processo decisório em diferentes níveis. A escolha do canal de pressão também pode ser influenciada pelo tipo de grupo de interesse, com grupos difusos tendendo a pressionar o \acrshort{pe} e grupos econômicos, o \acrshort{coue} ou a \acrshort{ce} \cite{coen2019legislative}.

    As formas de comunicação e táticas utilizadas no lobby também são relevantes, com predominância de formas escritas, mas com reuniões presenciais exercendo influência importante \cite{Huwyler2022}. Táticas que envolvem o engajamento face a face entre parlamentares e representantes de grupos de interesse são consideradas mais eficazes.

Os fatores relacionados à política desempenham um papel crucial no lobby na \acrshort{ue}. Propostas legislativas e questões políticas variam em diversas características, como complexidade, tipo de política, \textit{status quo}, saliência e o grau de conflito, e essas características influenciam as estratégias e o sucesso do lobby.

    A complexidade das propostas legislativas e questões políticas é um fator importante, já que a formulação de políticas é desafiadora devido à natureza técnica da sociedade e das tecnologias modernas \cite{kluver2015legislative}. A legislação varia em termos de complexidade, o que afeta a análise, compreensão e solução dos problemas políticos \cite{kluver_informational_2012}. Nesse sentido, a demanda por informações por grupos de interesse é proporcional à complexidade do tema \cite{kluver_informational_2012}.

    O tipo de política também é relevante, sendo distinguidas políticas regulatórias, distributivas e redistributivas \cite{lowi1964american}. Políticas distributivas distribuem recursos do governo para grupos sociais, políticas redistributivas transferem recursos entre grupos e políticas regulatórias moldam práticas comportamentais.

    O \textit{status quo }de uma política, ou seja, a legislação existente sobre o tema, também é importante. Iniciativas legislativas geralmente se baseiam em legislação anterior, e o lobby muitas vezes envolve grupos que defendem o \textit{status quo} e outros que lutam por mudanças. Notadamente, políticas que alteram o \textit{status quo} têm maior dificuldade de serem aprovadas \cite{baumgartner2009lobbying}.

    A saliência de uma proposta legislativa, ou seja, a atenção que ela recebe do público e dos \textit{stakeholders}, também tem implicações significativas para o lobby \cite{mahoney_lobbying_2007, mahoney2008brussels}. Propostas que geram grande atenção pública podem ter consequências eleitorais para os políticos e afetar a capacidade dos grupos de interesse de influenciar o processo decisório. 

    O grau de conflito em um debate político também é relevante. Algumas propostas são consensuais, enquanto outras geram grande oposição, afetando as estratégias de lobby e a formação de coalizões \cite{kluver_informational_2012, mahoney_lobbying_2007, mahoney2008brussels}.

Os fatores institucionais desempenham um papel crucial no lobby na \acrshort{ue}, influenciando as estratégias e o sucesso dos grupos de interesse. A estrutura institucional e as diferenças entre as instituições da \acrshort{ue}, como a \acrshort{ce}, o \acrshort{pe} e o \acrshort{coue}, moldam o ambiente em que o lobby ocorre.

    A \acrshort{ce}, responsável por propor legislação, apresenta uma configuração interna que influencia o lobby. O Colégio de Comissários, com seus gabinetes pessoais, e as Direções-Gerais (\acrshort{dg}s) desempenham papéis importantes. Comissários com experiência política em seus países de origem e laços com seus países influenciam as posições em relação a propostas legislativas. Além disso, as \acrshort{dg}s, com suas competências setoriais e culturas administrativas, diferem em suas visões políticas e exercem influência no conteúdo das propostas legislativas. A \acrshort{dg} responsável por uma proposta pode afetar a mobilização, a escolha de estratégias e o sucesso do lobby.

    O \acrshort{pe} também possui uma configuração interna relevante para o lobby. Os eurodeputados são agrupados em grupos partidários que coordenam suas atividades legislativas e influenciam seus votos \cite{Hix2002}. A escolha dos grupos partidários a serem abordados pelos lobistas é estratégica, considerando o papel dos grupos partidários e as diferenças ideológicas entre eles \cite{marshall2015explaining}.

    O sistema de comitês do \acrshort{pe} também é importante. A maior parte do trabalho parlamentar é conduzida em comitês organizados por temas, que diferem em composição e preferências \cite{yordanova2009rationale}. Os presidentes dos comitês, com sua influência na agenda legislativa e na organização do trabalho, são interlocutores atrativos para lobistas \cite{marshall2015explaining}. O relator e o relator sombra, responsáveis por elaborar relatórios e supervisionar o trabalho legislativo, também são alvos importantes para o lobby \cite{yordanova2009rationale}.

    Além disso, a transferência gradual de funções regulatórias dos Estados-membros para as instituições da \acrshort{ue} contribuiu para a europeização dos grupos de interesse. A crescente sofisticação política de interesses privados e públicos em um ambiente complexo de múltiplos níveis também é notável \cite{richardson2000government, Coen2007}.

Em suma, o lobby na \acrshort{ue} é um fenômeno complexo e multifacetado, moldado por uma série de fatores inter-relacionados. As características individuais dos grupos de interesse, como seus recursos e expertise, desempenham um papel importante, mas são insuficientes para explicar completamente as dinâmicas do lobby. Fatores contextuais, como a natureza das políticas em debate e o ambiente institucional, também são cruciais. A complexidade do sistema político da \acrshort{ue}, com múltiplas instituições e níveis de governança, oferece uma variedade de canais de acesso e oportunidades para os grupos de interesse, mas também exige estratégias sofisticadas e adaptação a diferentes contextos. A literatura sobre lobby na \acrshort{ue} continua a evoluir, buscando compreender as nuances desse processo e suas implicações para a formulação de políticas e a democracia na \acrshort{ue}. Na seção \ref{section:comportamento}, abaixo, descrevo melhor a literatura sobre comportamento parlamentar, buscando identificar seus determinantes.