Este capítulo dedica-se a entender o processo decisório da \acrshort{ue}. A forma de produção de normas comunitárias foi  sendo modificada ao longo da história da organização. Olhemos, então, para essas principais mudanças sob uma perspectiva histórica com foco na atuação do \acrshort{pe}.

A \acrshort{ue} é uma entidade política supranacional com raízes na \acrfull{ceca}, fundada em 1951, e na \acrfull{cee}, criada em 1971, sendo alargada em 1972-73 com a entrada de Reino Unido, Dinamarca e Irlanda. A historiografia inicial da \acrshort{ue}, liderada por Walter Lipgens, focou no idealismo federalista e na superação dos estados-nação. Alan S. Milward, por outro lado, adotou uma perspectiva revisionista, enfatizando os interesses nacionais e o fortalecimento dos estados-nação através da integração \cite{kaiser_history_2009}, debate que é presente até os dias atuais na integração europeia \cite{engelbrekt_eu_2023}. 

Sucessor da Assembleia Comum da \acrshort{ceca}, o \acrshort{pe} realizou sua primeira eleição direta em 1979, como demandado pelo Tratado de Roma. O poder do Parlamento, contudo, era limitado a um caráter consultivo. Desde então, \acrshort{pe} tem visto um aumento significativo de influência e poder dentro do quadro institucional da \acrshort{ue}. Em 1984, o Parlamento elaborou o "Projeto de Tratado que institui a \acrshort{ue}", demonstrando seu papel proativo na definição do futuro da integração europeia. Embora não tenha sido adotada, essa proposta lançou as bases para tratados subsequentes.

Na década de 1980, o Parlamento também ganhou influência informal sobre a seleção dos Presidentes da \acrshort{ce}, uma prática que mais tarde evoluiu para um direito formal de veto. A composição do \acrshort{pe} se expandiu juntamente com o crescimento da \acrshort{ue}, atingindo um limite de 751 membros com o Tratado de Lisboa.

A sede do Parlamento tornou-se um ponto de controvérsia, estabelecendo-se um arranjo dual em Estrasburgo e Bruxelas. Apesar da preferência do Parlamento por Bruxelas, foi consagrado Estrasburgo como a sede oficial no Tratado de Amsterdã.

O poder do Parlamento continuou a crescer através de revisões dos tratados, ganhando mais influência legislativa e o direito de aprovar acordos internacionais. Em 1999, o \acrshort{pe} demonstrou seu poder ao forçar a renúncia da Comissão Santer devido a alegações de fraude e má gestão, estabelecendo um precedente para responsabilizar a Comissão.

Outro marco fundamental para a configuração do \acrshort{pe} foi a criação do procedimento de codecisão. O Tratado de Maastricht (1992), que estabeleceu a \acrshort{ue}, instituiu o procedimento em que, sobre determinadas matérias, tanto o \acrshort{pe}, quanto o \acrshort{coue} deveriam aprovar um texto legislativo. O tratado dotou, assim, o \acrshort{pe} de maiores poderes legislativos. Porém, além do escopo limitado de domínios legislativos em que se aplicaria a codecisão (essencialmente o mercado interno), houve críticas no sentido de que o \acrshort{coue} poderia agir unilateralmente nos casos em que o Comitê de Conciliação\footnote{Em casos nos quais o \acrshort{pe} e o \acrshort{coue} não concordam com um texto único, é chamado o Comitê de Conciliação, formado por representantes de ambos os órgãos, a fim de se obter um acordo sobre um texto conjunto. Na seção \ref{section:podereslegislativos}, esse processo será apresentado com maiores detalhes.} não chegasse a um acordo de texto conjunto \cite{crombez_co-decision_1997}.

Desde então, o procedimento de codecisão foi sendo alargado. O Tratado de Amsterdã (1999) simplificou o processo de codecisão - ao permitir a conclusão de acordos em primeira leitura. Além disso, alargou as bases jurídicas em que deveria ser aplicado para mais de 40, incluindo transportes, meio ambiente, justiça e assuntos internos, emprego e temas sociais. Esse tratado, portanto, reforçou a interdependência entre o Parlamento e o Conselho \cite{shackleton2003codecision}.

Na literatura acadêmica, uma série de estudos começaram a aparecer com o objetivo de avaliar a influência relativa dos \acrshort{mpe}s na formação das leis da \acrshort{ue} \cite{tsebelis1994power, scully2000democracy, earnshaw1999european, crombez2000understanding, tsebelis2001legislative}.

Em 2003, pelo Tratado de Nice, cinco novos domínios legislativos foram incluídos no escopo do processo de codecisão. Em 2009, a partir do Tratado de Lisboa, esse número foi elevado para 85 domínios, tais como agricultura, pesca e política comercial comum. Não só os domínios foram expandidos, mas o processo de codecisão tornou-se o \acrfull{plo}.

O art. 289 (1), do \acrfull{tfue}, definiu a codecisão como o processo de: 
\begin{quote}
    "adoção de um regulamento, de uma diretiva ou de uma decisão conjuntamente pelo Parlamento Europeu e pelo Conselho, sob proposta da Comissão"  (art. 289, 1, \acrshort{tfue})
\end{quote}

Esse mecanismo constituiu uma espécie de bicameralismo da \acrshort{ue}. O \acrshort{coue} representa os interesses dos Estados, já o \acrshort{pe} representa os cidadãos. De acordo com Lijphart (\citeyear{lijphart2019modelos}), é possível distinguir entre bicameralismos fortes e fracos. Nestes, há divergências nos métodos de representação e uma divisão razoavelmente justa de poderes. Nesse sentido, o sistema instituído, e reforçado, da codecisão enquadra-se nesse conceito de bicameralismo forte \cite{burns2014},

Deve-se, contudo, relativizar o segundo aspecto de bicameralismos fortes - divisão razoavelmente justa de poderes. O Conselho possui mais prerrogativas no que diz respeito à tomada de decisão, já que está relativamente mais próximo do \textit{status quo}, além de possuir uma vantagem informacional \cite{shackleton2003codecision}.

Como forma de acelerar o processo de tomada de decisão, foram criados os chamados "trílogos". Contendo cerca de 10 participantes de cada instituição (\acrshort{pe}, \acrshort{ce} e \acrshort{coue}), os trílogos são reuniões informais realizadas antes da segunda leitura a fim de se facilitar o diálogo inter-instrucional. Apesar de acelerar a tramitação, os trílogos constituem uma barreira para um diálogo construtivo entre um número amplo de membros das instituições \cite{griller_lisbon_2008}. Nesse sentido, há um \textit{trade-off} entre eficiência do processo legislativo e sua legitimidade democrática \cite{shackleton2003codecision}.

A necessidade de buscar um equilíbrio é objeto de debate e acordos interinstitucionais. A Declaração Comum sobre as Modalidades do Processo de Codecisão (2007) estabeleceu as práticas desse processo e reconhece a importância do sistema de trílogos. O Acordo-Quadro sobre as Relações entre o Parlamento Europeu e a Comissão Europeia (2010) visou fortalecer o diálogo e a cooperação entre essas instituições, melhorando o fluxo de informações e a colaboração em procedimentos e programação. Abrangeu, ainda, aspectos como reuniões com especialistas, informações confidenciais, acordos internacionais e o calendário de trabalho da Comissão. O Acordo Interinstitucional "Legislar Melhor" (2016) substituiu o acordo anterior de 2003 e estabeleceu um conjunto de iniciativas e procedimentos para aprimorar a qualidade da legislação da \acrshort{ue}. Abordou, também, questões como programação legislativa, avaliações de impacto, consultas públicas, escolha da base jurídica, atos delegados e de execução, transparência, aplicação e simplificação.

Por fim, vale destacar algumas características do \acrshort{pe} relevantes para este trabalho. Em primeiro lugar, as eleições ocorrem por sufrágio universal a cada cinco anos. As regras eleitorais que são aplicadas são responsabilidades de cada Estado. Alguns princípios comuns, contudo, devem ser observados:
\begin{itemize}
    \item As eleições devem ocorrer durante um período de quatro dias, entre quinta-feira e domingo;
    \item O numero de eurodeputados eleitos por um partido é proporcional ao número de votos que recebe; e
    \item Cada cidadão pode votar somente uma vez.
\end{itemize}

O total de membros eleitos é variável de eleição para eleição, porém não pode exceder o número de 750 eurodeputados, sem contar o ocupante do cargo de Presidente do Parlamento. Além disso, a distribuição por país é baseada no princípio da proporcionalidade degressiva, isto é, parlamentares de países menores representam proporcionalmente mais, em número de eleitores, do que países maiores populacionalmente. Nas eleições de 2024, a Alemanha, país mais populoso da Europa, elegeu 96 eurodeputados; enquanto que Malta elegeu 6.

Em segundo lugar, as eleições são disputadas pelos partidos nacionais. Quando eleito, o eurodeputado normalmente opta por ingressar em grupos políticos transnacionais, aos quais grande parte dos partidos nacionais estão filiados. Atualmente há sete grupos parlamentares: o Grupo do Partido Popular Europeu (Democratas-Cristãos); o Grupo da Aliança Progressista dos Socialistas e Democratas no Parlamento Europeu; o Grupo "Patriotas pela Europa"; o Grupo dos Conservadores e Reformistas Europeus; o Grupo \textit{Renew Europe}; o Grupo dos Verdes/Aliança Livre Europeia; Grupo da Esquerda no Parlamento Europeu - GUE/NGL; e o Grupo Europa das Nações Soberanas.
 
O \acrshort{pe} conta com poderes legislativos (para adoção de legislação em conjunto com o \acrshort{coue}), poderes orçamentários (atuando em conjunto com \acrshort{coue} para definir o orçamento anual da \acrshort{ue}), e poderes fiscalizatórios (com os quais os \acrshort{mpe}s fiscalizam o trabalho de outras instituições do bloco, especialmente a \acrshort{ce}). Como esses poderes destacam a importância do \acrshort{pe} para a organização da \acrshort{ue}, analisemos a seguir cada um desses poderes, com um foco maior no primeiro.