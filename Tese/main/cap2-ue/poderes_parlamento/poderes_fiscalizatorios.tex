\section{Poderes Fiscalizatórios}

O \acrshort{pe} possui um amplo espectro de poderes fiscalizatórios sobre as principais instituições da \acrshort{ue}, buscando assegurar a \textit{accountability} e a transparência no funcionamento do bloco.

No âmbito do \acrshort{coue}, o \acrshort{pe} assegura sua influência através da participação do seu presidente nas reuniões do Conselho, onde este apresenta a posição do Parlamento sobre os temas em discussão. Além disso, o presidente do \acrshort{coue} reporta os resultados das reuniões ao \acrshort{pe}, permitindo um acompanhamento contínuo das decisões tomadas pelos chefes de Estado e de governo.

A interação com o Conselho da \acrshort{ue} também é fundamental para o exercício do controle parlamentar. O Parlamento debate o programa de cada presidência semestral do Conselho, apresenta perguntas e solicita informações sobre políticas específicas. O Alto Representante para os Negócios Estrangeiros e a Política de Segurança também presta contas ao Parlamento, apresentando relatórios periódicos sobre as ações da UE em matéria de política externa e de segurança.
 
O \acrshort{pe} exerce um controle crucial sobre a \acrshort{ce}, com a prerrogativa de aprovar ou destituir seus membros. Os comissários designados passam por audiências no Parlamento, onde são questionados sobre suas qualificações e planos de ação. A Comissão também apresenta relatórios regulares ao Parlamento, incluindo um relatório anual sobre as atividades da \acrshort{ue} e a execução do orçamento, garantindo a transparência e a prestação de contas. Além disso, o \acrshort{pe} pode apresentar monções de censura à Comissão, com a possibilidade de destituí-la em última instância. 

O Parlamento também exerce controle sobre outras instituições, como o \acrfull{tj}, o \acrfull{bce} e o \acrfull{tce}. O Parlamento pode solicitar ao \acrshort{tj} que tome medidas contra a Comissão ou o Conselho em caso de violação da legislação da \acrshort{ue}. No caso do \acrshort{bce}, o Parlamento aprova a nomeação dos seus principais dirigentes e recebe relatórios periódicos sobre a política monetária da zona euro. O Parlamento também utiliza os relatórios do \acrshort{tce} para avaliar a execução do orçamento da \acrshort{ue} e decidir sobre a concessão da quitação à Comissão.

Além disso, o \acrshort{pe} recebe petições de cidadãos da \acrshort{ue} e pode estabelecer comissões de inquérito para investigar alegações de violações ou má administração na aplicação do direito da \acrshort{ue}. Essas ferramentas permitem que o Parlamento atue como um canal de comunicação entre os cidadãos e as instituições europeias, garantindo que suas preocupações sejam ouvidas e consideradas.

Os poderes fiscalizatórios do \acrshort{pe}, portanto,  são essenciais para o funcionamento democrático e transparente da \acrshort{ue}. Através do exercício desses poderes, o Parlamento busca assegurar que as instituições da \acrshort{ue} atuem em conformidade com os princípios do Estado de Direito e que os interesses dos cidadãos europeus sejam protegidos e promovidos.

