\section{Poderes Orçamentários}

Com a entrada em vigor do Tratado de Lisboa, o \acrshort{pe} adquiriu um papel crucial na elaboração e aprovação do orçamento anual global a \acrshort{ue}, não só em colaboração com o Conselho, mas também com capacidade decisória na matéria.


O processo orçamentário anual inicia-se com a elaboração das previsões de receitas e despesas por todas as instituições da \acrshort{ue}. A \acrshort{ce}, com base nessas previsões, consolida e apresenta um projeto de orçamento ao Parlamento e ao Conselho. O Conselho, por sua vez, adota uma posição sobre o projeto e a encaminha ao Parlamento, que dispõe de um prazo para aprová-la ou emendá-la.

Durante as discussões no \acrshort{pe}, as comissões parlamentares debatem o projeto de orçamento e apresentam os seus pareces à Comissão dos Orçamentos, responsável pela preparação da posição do Parlamento. A decisão do parlamento é tomada por maioria absoluta dos membros, podendo aceitar o parecer da \acrshort{ce}, propor alterações, ou aceitar tacitamente - caso não decida dentro de 42 dias.


Caso o Parlamento opte por alterar o projeto, um Comitê de Conciliação, composto por representantes de ambas as instituições, é convocado para buscar um consenso. Se um acordo for alcançado, o projeto comum é submetido à aprovação do Parlamento e do Conselho. Na ausência de consenso, a Comissão apresenta um novo projeto de orçamento, reiniciando o ciclo.

Vale mencionar que as decisões do Parlamento e do Conselho no que diz respeito às receitas e às despesas devem respeitar os limites das despesas da \acrshort{ue}, definidas no Quadro Financeiro Plurianual, negociado uma vez a cada sete anos.

% despesas anuais fixadas na programação financeira da \acrshort{ue}

Além da co-decisão na elaboração do orçamento anual, o \acrshort{pe} exerce um papel importante no controle da execução orçamentária. A \acrshort{ce}, responsável pela gestão do orçamento, está sujeita ao escrutínio do Parlamento, que avalia a conformidade das ações com as diretrizes estabelecidas e a eficácia na utilização dos recursos.

O Parlamento detém a prerrogativa de conceder ou recusar a quitação à Comissão, ou seja, a aprovação final das contas do exercício. Essa decisão é tomada após uma análise minuciosa dos relatórios financeiros e das atividades da Comissão, incluindo a avaliação do Tribunal de Contas Europeu.

Em suma, o \acrshort{pe} desempenha um papel ativo e influente no processo orçamentário da \acrshort{ue}, compartilhando com o Conselho a responsabilidade pela definição das prioridades de gastos e exercendo um controle rigoroso sobre a execução do orçamento, garantindo a transparência e a \textit{accountability} na gestão dos recursos públicos.
