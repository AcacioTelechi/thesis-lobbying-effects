\section{Poderes Legislativos}
\label{section:podereslegislativos}

O \acrshort{pe} possui um papel crucial no processo legislativo da \acrshort{ue}, atuando em conjunto com o \acrshort{coue} na aprovação da maioria das leis europeias, de acordo com o processo legislativo ordinário (artigo 289 do \acrshort{tfue}). O trabalho legislativo se inicia com a apresentação de um "texto legislativo" pela \acrshort{ce}, que detém o monopólio da iniciativa normativa. Em seguida, a proposta é analisada por um parlamentar no âmbito de uma comissão parlamentar, que elabora um relatório, sob a liderança de um relator designado. A escolha da comissão, normalmente, ocorre por decisão dos coordenadores de comissão. 

Cada proposta legislativa é confiada a um grupo político, o qual designa um "relator" para elaborar o relatório em nome da comissão. Demais grupos políticos podem designar "relatores-sombra" para coordenar as respectivas posições sobre o assunto em debate.

Após votação e possíveis alterações no âmbito da comissão, o relatório é submetido ao plenário do Parlamento para aprovação. O presidente da comissão preside as suas reuniões e dos seus coordenadores. Além disso, o presidente possui ingerência no processo de votação, bem como nas regras relativas à admissibilidade das alterações.

Os coordenadores da comissão são designados pelos grupos políticos. Os coordenadores muitas vezes se reúnem a portas fechadas (\textit{in camera}) às margens das reuniões das comissões. A comissão pode delegar aos coordenadores o poder de alocação dos relatórios e opiniões aos grupos, de decisão sobre audiências públicas na comissão, de pedidos de estudos, entre outras atividades relativas à organização dos trabalhos da comissão.

As comissões parlamentares e seus membros possuem o apoio administrativo pelos secretários de comissões. Esses funcionários organizam as reuniões da comissão, planejam e dão suporte e assessoria a respeito dos assuntos da comissão). Além dos secretários, há os assessores dos grupos políticos, os quais dão suporte e assessoria tanto para o coordenador do seu grupo quanto para os membros individualmente). Outros membros e órgãos de apoios são: os assessores dos membros da comissão, a \acrfull{ual}, o Serviço Jurídico, o Diretório para Atos Legislativos, os Departamentos de Políticas, o Serviço de Pesquisa do Parlamento Europeu, a assessoria de imprensa do \acrshort{pe} e os Diretórios-Gerais para Tradução e Interpretação. Vale mencionar que na terceira leitura, a \acrshort{ual} coordena a assistência administrativa para as delegações do Parlamento no comitê de conciliação.

Aprovado o texto na comissão, ele é encaminhado para o Plenário. Caso obtenha aprovação em sessão plenária, a posição do \acrshort{pe} terá sido adotada. Esse processo pode se repetir uma ou mais vezes a depender do tipo de procedimento e do acordo, ou não, alcançado com o Conselho.

No que diz respeito à adoção de atos legislativos, há dois procedimentos: o \acrfull{plo}, ou também chamado de codecisão; e os processos legislativos especiais, aplicados a casos específicos nos quais o Parlamento possui apenas papel consultivo.

Sobre os últimos, o Conselho atua como único legislador. Ele ocorre em questões definidas nos tratados da \acrshort{ue}, como por exemplo, em temas fiscais. Em tais casos, o \acrshort{pe} emite apenas um parecer consultivo. Em casos específicos, porém, quando assim determinado, o parecer consultivo é obrigatório. A matéria só poderá ter força de lei quando o \acrshort{pe} tiver emitido seu parecer (art. 311º, \acrshort{tfue}).

\begin{figure}
    \caption{Fluxograma resumido do Processo Legislativo Ordinário}
    \includegraphics[width=\textwidth]{imgs/Processo legislativo ordinário.drawio.png}
    \label{fig:plo}
    \centering
    \caption*{Fonte: o autor (2025)}
\end{figure}


No \acrshort{plo}, o Parlamento e o Conselho atuam como colegisladores. Há uma gama vasta de temas em que o processo de codecisão é demandado: governança econômica, imigração, energia, transportes, meio ambiente, proteção dos consumidores, entre outros. Atualmente, a maior parte das leis europeias são adotadas por esse processo. 

O Tratado de Maastricht (1992) introduziu o mecanismo de codecisão, porém ainda num escopo limitado de temas. O Tratado de Amsterdã (1999) ampliou e reforçou a eficácia do mecanismo. Em 2009, com a entrada em vigor do Tratado de Lisboa, a codecisão passou a ser denominada de \acrshort{plo}, tornando-se, assim, o principal processo legislativo adotado pela \acrshort{ue}. A figura \ref{fig:plo} resume o principal rito legislativo da \acrshort{ue}.

No \acrshort{plo}, a posição do Parlamento é encaminhada ao Conselho. Se o texto for aprovado sem alterações, a proposta é adotada. Caso, entretanto, haja alterações por parte do Conselho, o texto é reencaminhado para o \acrshort{pe} para segunda leitura. O Parlamento examina as alterações e pode aprová-las, rejeitá-las, ou propor novas alterações. No primeiro caso, a proposta legislativa é adotada após aprovação em sessão plenária. Se, contudo, as alterações do Conselho forem rejeitadas, a proposta é arquivada. No terceiro caso, o texto alterado é reencaminhado para o Conselho para segunda leitura.

Na segunda leitura do conselho, se o Conselho aprovar todas as alterações do \acrshort{pe}, o texto é, então aprovado. Se, todavia, não aprová-las, convoca-se o Comitê de Conciliação. Tal comitê é composto por igual número de deputados e representantes do Conselho. Seus membros buscam chegar a um texto aceito por ambos. Se a tentativa for fracassada, o texto é arquivado. Se houver sucesso, o texto é reencaminhado para o Parlamento e para o Conselho para terceira leitura. Em terceira leitura, nem o Conselho, nem o Parlamento podem propor alterações. Devem, portanto, ou rejeitar, ou aprovar o texto como foi enviado pelo Comitê de Conciliação.


O \acrshort{pe} também possui participação em atos não-legislativos, chamado de "aprovação", introduzido pelo Ato Único Europeu (1986). O mecanismo aplica-se em acordos de associação e de adesão à \acrshort{ue}, em alguns acordos comerciais, em casos de violações graves aos direitos fundamentais (art. 7º, \acrshort{tfue}). A aprovação também pode ser utilizada como ato legislativo em casos de legislação sobre combate à discriminação.

Há outros procedimentos cuja participação do \acrshort{pe} é destacada: parecer nos termos do artigo 140.º do \acrshort{tfue} (União Monetária), procedimentos relativos ao diálogo social (art. 154º, \acrshort{tfue}), à apreciação de acordos voluntários (art. 48º, \acrshort{tfue})), à codificação (art. 46º do Regimento), à medidas de execução e disposições delegadas.

No âmbito da União Monetária (art. 140.º do \acrshort{tfue}), o Parlamento emite um parecer sobre os progressos dos Estados-Membros em relação à adoção da moeda única, embora o Conselho tome a decisão final.

Nos procedimentos relativos ao diálogo social (artigos 154.º e 155.º do \acrshort{tfue}), o Parlamento atua recepcionando a consulta realizada pela Comissão aos parceiros sociais e avaliando acordos e convenções. 

Adicionalmente, o Parlamento é informado sobre acordos voluntários propostos pela Comissão como alternativa a medidas legislativas (artigo 48.º do Regimento) e pode apresentar uma resolução recomendando sua aprovação ou rejeição.

Quanto à codificação, que visa consolidar atos legislativos em um único texto para maior clareza (artigo 46.º do Regimento), a Comissão de Assuntos Jurídicos do Parlamento fica a cargo dessa consolidação.

O Parlamento também exerce controle sobre medidas de execução e disposições delegadas adotadas pela Comissão, podendo se opor a medidas que não estejam em conformidade com a legislação vigente ou que violem os princípios da subsidiariedade e proporcionalidade.

Embora a iniciativa legislativa na \acrshort{ue} seja de competência da \acrshort{ce}, o Parlamento também possui um direito de iniciativa, garantido pelo Tratado de Maastricht e consolidado no Tratado de Lisboa.

Esse direito permite ao Parlamento solicitar à Comissão a apresentação de propostas legislativas em áreas específicas, mediante aprovação da maioria de seus membros e com base em um relatório elaborado pela comissão parlamentar competente (artigo 225º do \acrshort{tfue}). O Parlamento pode, inclusive, estabelecer um prazo para a apresentação da proposta, mas a Comissão tem a prerrogativa de aceitar ou recusar o pedido.

Além disso, deputados individuais também podem apresentar propostas de atos da União com base nesse direito de iniciativa. A proposta é encaminhada ao Presidente do Parlamento, que a transmite à comissão competente para análise e possível apresentação à sessão plenária.

O Parlamento Europeu também exerce sua iniciativa através da elaboração de relatórios de iniciativa pelas comissões parlamentares, que podem apresentar propostas de resolução sobre assuntos de sua competência, desde que autorizados pela Conferência dos Presidentes (artigos 37, 46 e 52 do Regimento e artigo 17(1) do \acrshort{tfue}).

A participação do Parlamento na programação anual e plurianual da \acrshort{ue} também é relevante. O Parlamento coopera com a Comissão na elaboração do programa de trabalho, que define as prioridades da \acrshort{ue} para o período. Após a adoção do programa pela Comissão, um diálogo tripartido (trílogo) entre o Parlamento, o Conselho e a Comissão busca um acordo sobre a programação, conforme detalhado no anexo XIV do Regimento.

Em resumo, o direito de iniciativa do \acrshort{pe}, seja através de solicitações à Comissão, propostas de deputados individuais ou relatórios de iniciativa, fortalece seu papel no processo legislativo da \acrshort{ue}, complementando a iniciativa da Comissão.
