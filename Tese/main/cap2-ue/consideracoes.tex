\section{Considerações sobre o processo decisório na UE}
\label{sec:cap2_consideracoes}

O processo decisório da União Europeia é um complexo sistema que reflete a natureza \textit{sui generis} da organização, equilibrando interesses supranacionais e nacionais. No cerne deste sistema encontra-se o processo legislativo, o mecanismo através do qual as políticas da \acrshort{ue} são formuladas e adotadas. A evolução deste processo está intrinsecamente ligada à questão da legitimidade democrática do bloco. Em resposta a um crescente "déficit democrático", a \acrshort{ue} tem procurado fortalecer seus canais de participação popular, como a Iniciativa de Cidadania Europeia. 

A Iniciativa de Cidadania Europeia permite que cidadãos da \acrshort{ue} convidem a Comissão Europeia a apresentar uma proposta de ato jurídico para a aplicação dos Tratados. Este instrumento de democracia participativa visa reforçar a ligação entre os cidadãos e as instituições, permitindo-lhes participar ativamente na formulação de políticas. Para que uma iniciativa seja considerada, deve reunir o número necessário de assinaturas no prazo de um ano e ser validada pelas autoridades competentes de cada país. Uma vez validada, a Comissão analisa a proposta, podendo organizar audiências públicas, estudos, etc. Em seguida, apresenta a sua resposta oficial ao \acrshort{pe}.

Todo este movimento em direção a uma maior participação cidadã ilustra a contínua busca da \acrshort{ue} por maior legitimidade democrática. É precisamente neste contexto de resposta ao "déficit democrático" que se deve entender o fortalecimento progressivo do \acrshort{pe}. A expansão dos seus poderes legislativos e de fiscalização não é um mero ajuste institucional, mas sim uma consequência direta da necessidade de dotar a estrutura de governação da \acrshort{ue} de um pilar de representação popular direta, tornando-a mais responsiva e permeável às demandas dos cidadãos.

Contudo, é crucial ressalvar que uma maior permeabilidade institucional não garante, por si só, uma representação equitativa. A abertura de canais de participação, embora positiva, não anula as desigualdades de recursos, organização e acesso entre os diversos atores sociais. Organizações internacionais, pela sua própria natureza, tendem a apresentar um hiato entre representantes e representados, e a \acrshort{ue} não é exceção \cite{dahl1999can}. Grupos de interesse com maior poder econômico e estrutura profissional possuem uma vantagem desproporcional para navegar a complexidade do processo decisório e influenciar os seus resultados, em detrimento de cidadãos comuns ou de organizações da sociedade civil com menos recursos.

Essa assimetria na representação de interesses torna imperativo o estudo aprofundado dos efeitos do lobby sobre o comportamento parlamentar. Compreender como os eurodeputados respondem às diversas pressões externas é fundamental para avaliar a qualidade da democracia no nível supranacional. Por isso, esta tese foca não apenas em saber \textit{se} o lobby influencia as decisões, mas em analisar \textit{o quanto} a heterogeneidade dos atores - comparando, por exemplo, o lobby empresarial com o de organizações não-governamentais - produz efeitos distintos no processo legislativo e no comportamento dos representantes eleitos.
