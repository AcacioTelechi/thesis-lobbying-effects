\chapter{Resultados}
\label{chapter:resultados}
Este capítulo tem como objetivo apresentar e discutir os principais resultados empíricos da tese, respondendo às hipóteses formuladas e avaliando o impacto do lobby sobre a atividade legislativa no Parlamento Europeu. Inicialmente, são descritas as características dos lobistas e dos parlamentares presentes na amostra, bem como padrões descritivos das reuniões e perguntas parlamentares. Em seguida, são apresentados os resultados dos testes das hipóteses centrais, com análise dos efeitos estimados, heterogeneidades e mecanismos identificados. Por fim, o capítulo discute as implicações dos achados para o debate sobre influência política e transparência institucional.

% \input{main/cap4-resultados/resultados}
\subsection{Análise descritiva dos lobistas}
\label{sec:resultados_descritica_lobistas}



Nesta subseção apresentamos estatísticas descritivas calculadas diretamente do ficheiro \texttt{./data/silver/df\_lobbyists.csv}. Para assegurar reprodutibilidade, as tabelas e figuras abaixo são geradas pelo script \texttt{generate\_lobbyists\_descriptives.py} e incluídas automaticamente.

\paragraph{Distribuição por categoria}
\begin{table}[!htbp]
\centering
\caption{Distribuição por categoria}
\begin{tabular}{lrr}
\toprule
Categoria & Total & (\%) \\
\midrule
Empresas & 5{.}770 & 46{,}3 \\
ONGs & 3{.}480 & 27{,}9 \\
Outros & 3{.}218 & 25{,}8 \\
\bottomrule
\end{tabular}

\end{table}

A Tabela acima apresenta a composição do universo de lobistas por categoria organizacional. Observa-se o predomínio de entidades empresariais e organizações da sociedade civil, com participação ainda substantiva da categoria residual \textit{Other}. Esse arranjo é compatível com um campo plural, no qual interesses concentrados e difusos coabitam o espaço de representação junto às instituições europeias.

\begin{figure}[!htbp]
\centering
\includegraphics[width=0.9\textwidth]{figures/category_distribution.png}
\caption{Distribuição por categoria}
\end{figure}

O gráfico reforça assimetrias moderadas entre categorias e não sugere concentração extrema em um único tipo organizacional. Em termos substantivos, isso indica competição horizontal por acesso e agenda entre perfis empresariais e societais.

\paragraph{País-sede e ano de registo}
\begin{table}[!htbp]
\centering
\caption{Distribuição por país-sede}
\begin{tabular}{lrrr}
\toprule
País-sede & Total & (\%) & Membro da UE \\
\midrule
BELGIUM & 2270 & 18.200 & True \\
GERMANY & 1740 & 14.000 & True \\
FRANCE & 1165 & 9.300 & True \\
UNITED KINGDOM & 787 & 6.300 & False \\
NETHERLANDS & 786 & 6.300 & True \\
SPAIN & 780 & 6.300 & True \\
ITALY & 753 & 6.000 & True \\
UNITED STATES & 567 & 4.500 & False \\
AUSTRIA & 312 & 2.500 & True \\
FINLAND & 297 & 2.400 & True \\
SWEDEN & 293 & 2.400 & True \\
SWITZERLAND & 276 & 2.200 & False \\
POLAND & 251 & 2.000 & True \\
DENMARK & 224 & 1.800 & True \\
IRELAND & 209 & 1.700 & True \\
PORTUGAL & 177 & 1.400 & True \\
GREECE & 154 & 1.200 & True \\
CZECH REPUBLIC & 139 & 1.100 & False \\
ROMANIA & 120 & 1.000 & True \\
NORWAY & 105 & 0.800 & False \\
HUNGARY & 91 & 0.700 & True \\
LUXEMBOURG & 83 & 0.700 & True \\
ESTONIA & 67 & 0.500 & True \\
CROATIA & 66 & 0.500 & True \\
LITHUANIA & 62 & 0.500 & True \\
SLOVENIA & 57 & 0.500 & True \\
SLOVAKIA & 54 & 0.400 & True \\
BULGARIA & 52 & 0.400 & True \\
CANADA & 46 & 0.400 & False \\
MALTA & 42 & 0.300 & True \\
LATVIA & 31 & 0.200 & True \\
UKRAINE & 29 & 0.200 & False \\
JAPAN & 26 & 0.200 & False \\
CYPRUS & 24 & 0.200 & True \\
BRAZIL & 23 & 0.200 & False \\
AUSTRALIA & 20 & 0.200 & False \\
TURKEY & 20 & 0.200 & False \\
ISRAEL & 19 & 0.200 & False \\
SERBIA & 19 & 0.200 & False \\
ALBANIA & 11 & 0.100 & False \\
SINGAPORE & 10 & 0.100 & False \\
UNITED ARAB EMIRATES & 10 & 0.100 & False \\
NEW ZEALAND & 8 & 0.100 & False \\
CHINA & 8 & 0.100 & False \\
INDIA & 8 & 0.100 & False \\
MOROCCO & 7 & 0.100 & False \\
SOUTH AFRICA & 7 & 0.100 & False \\
SENEGAL & 6 & 0.000 & False \\
MALAYSIA & 6 & 0.000 & False \\
MEXICO & 6 & 0.000 & False \\
NIGERIA & 6 & 0.000 & False \\
BOSNIA-HERZEGOVINA & 6 & 0.000 & False \\
LIECHTENSTEIN & 5 & 0.000 & False \\
MOLDOVA, REPUBLIC OF & 5 & 0.000 & False \\
THAILAND & 5 & 0.000 & False \\
KOREA, REPUBLIC OF & 5 & 0.000 & False \\
TAIWAN & 4 & 0.000 & False \\
LEBANON & 4 & 0.000 & False \\
ECUADOR & 4 & 0.000 & False \\
COLOMBIA & 4 & 0.000 & False \\
ICELAND & 3 & 0.000 & False \\
PALESTINIAN OCCUPIED TERRITORY & 3 & 0.000 & False \\
INDONESIA & 3 & 0.000 & False \\
KENYA & 3 & 0.000 & False \\
BERMUDA & 3 & 0.000 & False \\
JORDAN & 3 & 0.000 & False \\
GEORGIA & 3 & 0.000 & False \\
ARGENTINA & 3 & 0.000 & False \\
HONG KONG & 3 & 0.000 & False \\
RUSSIA, FEDERATION OF & 3 & 0.000 & False \\
GHANA & 3 & 0.000 & False \\
MONACO & 3 & 0.000 & False \\
ZAMBIA & 2 & 0.000 & False \\
NIGER & 2 & 0.000 & False \\
MONTENEGRO & 2 & 0.000 & False \\
KAZAKHSTAN & 2 & 0.000 & False \\
PHILIPPINES & 2 & 0.000 & False \\
COSTA RICA & 2 & 0.000 & False \\
DOMINIQUE & 2 & 0.000 & False \\
URUGUAY & 2 & 0.000 & False \\
ARMENIA & 2 & 0.000 & False \\
TUNISIA & 2 & 0.000 & False \\
VENEZUELA & 2 & 0.000 & False \\
NORTH MACEDONIA & 2 & 0.000 & False \\
MAURITANIA & 1 & 0.000 & False \\
REUNION & 1 & 0.000 & False \\
GIBRALTAR & 1 & 0.000 & False \\
CONGO, DEMOCRATIC REPUBLIC OF & 1 & 0.000 & False \\
QATAR & 1 & 0.000 & False \\
DOMINICAN REPUBLIC & 1 & 0.000 & False \\
CHILE & 1 & 0.000 & False \\
SRI LANKA & 1 & 0.000 & False \\
BOLIVIA & 1 & 0.000 & False \\
VIETNAM & 1 & 0.000 & False \\
OMAN & 1 & 0.000 & False \\
MACAO & 1 & 0.000 & False \\
MONGOLIA & 1 & 0.000 & False \\
COTE D'IVOIRE & 1 & 0.000 & False \\
ZIMBABWE & 1 & 0.000 & False \\
NAMIBIA & 1 & 0.000 & False \\
NEPAL & 1 & 0.000 & False \\
LAOS, PEOPLE'S DEMOCRATIC REPUBLIC & 1 & 0.000 & False \\
BARBADOS & 1 & 0.000 & False \\
BURKINA FASO & 1 & 0.000 & False \\
KOSOVO & 1 & 0.000 & False \\
PAPUA NEW GUINEA & 1 & 0.000 & False \\
GABON & 1 & 0.000 & False \\
PAKISTAN & 1 & 0.000 & False \\
CAMEROON & 1 & 0.000 & False \\
FAROE ISLANDS & 1 & 0.000 & False \\
GUATEMALA & 1 & 0.000 & False \\
MYANMAR & 1 & 0.000 & False \\
CONGO & 1 & 0.000 & False \\
RUANDA & 1 & 0.000 & False \\
MALI & 1 & 0.000 & False \\
TRINIDAD AND TOBAGO & 1 & 0.000 & False \\
PARAGUAY & 1 & 0.000 & False \\
TOGO & 1 & 0.000 & False \\
CAMBODIA & 1 & 0.000 & False \\
TANZANIA, UNITED RE UBLIC OF & 1 & 0.000 & False \\
UGANDA & 1 & 0.000 & False \\
\bottomrule
\end{tabular}

\end{table}

A distribuição geográfica evidencia forte concentração em Estados-Membros centrais: Bélgica (\textasciitilde18,2\%), Alemanha (\textasciitilde14,0\%) e França (\textasciitilde9,3\%). Países como Países Baixos, Espanha, Reino Unido e Itália aparecem na sequência (\textasciitilde6\% cada). Nota-se presença extracomunitária não desprezível (Estados Unidos \textasciitilde4,5\%), sinalizando a atratividade regulatória do mercado europeu.

\begin{figure}[!htbp]
\centering
\includegraphics[width=0.9\textwidth]{figures/country_distribution_top20.png}
\caption{País-sede (top 20)}
\end{figure}

O recorte \textit{top 20} evidencia uma cauda longa: muitos países com baixas frequências, consistentes com a internacionalização seletiva do \textit{lobbying}. O padrão é coerente com hipóteses de \textit{venue shopping} e vantagens de proximidade institucional em Bruxelas/Strasburgo.

\begin{table}[!htbp]
\centering
\caption{Distribuição por ano de registo}
\begin{tabular}{rrr}
\toprule
Ano de registo & Total & (\%) \\
\midrule
2008 & 240 & 1.900 \\
2009 & 445 & 3.600 \\
2010 & 316 & 2.500 \\
2011 & 520 & 4.200 \\
2012 & 425 & 3.400 \\
2013 & 342 & 2.700 \\
2014 & 511 & 4.100 \\
2015 & 910 & 7.300 \\
2016 & 1004 & 8.100 \\
2017 & 773 & 6.200 \\
2018 & 723 & 5.800 \\
2019 & 675 & 5.400 \\
2020 & 923 & 7.400 \\
2021 & 1059 & 8.500 \\
2022 & 1366 & 11.000 \\
2023 & 2236 & 17.900 \\
\bottomrule
\end{tabular}

\end{table}

Temporalmente, observa-se aceleração do registro de entidades após meados da década de 2010, com 2023 concentrando \textasciitilde17,9\% do total. Picos intermediários (2015--2016; 2020--2022) são compatíveis com ciclos legislativos, janelas regulatórias e alterações incrementais nos mecanismos de transparência.

\begin{figure}[!htbp]
\centering
\includegraphics[width=0.75\textwidth]{figures/year_distribution.png}
\caption{Ano de registo}
\end{figure}

O padrão visual sugere crescimento estrutural recente do ecossistema de representação de interesses, possivelmente associado às agendas de transição digital e verde e à recomposição pós-pandemia.

\paragraph{Cobertura temática}
\begin{table}[!htbp]
\centering
\caption{Cobertura temática (proporções e contagens)}
\begin{tabular}{lrr}
\toprule
Tema & Total & Proporção \\
\midrule
Infraestrutura e Indústria & 8515 & 0.683 \\
Tecnologia & 8472 & 0.679 \\
Economia e Comércio & 8402 & 0.674 \\
Ambiente e Clima & 8067 & 0.647 \\
Política Externa e Segurança & 6393 & 0.513 \\
Saúde & 5447 & 0.437 \\
Educação & 5134 & 0.412 \\
Agricultura & 4405 & 0.353 \\
Direitos Humanos & 3156 & 0.253 \\
\bottomrule
\end{tabular}

\end{table}

As incidências por domínio destacam \textit{Infraestrutura e Indústria} (\textasciitilde68,3\%), \textit{Tecnologia} (\textasciitilde67,9\%) e \textit{Economia e Comércio} (\textasciitilde67,4\%), seguidas por \textit{Ambiente e Clima} (\textasciitilde64,7\%) e \textit{Assuntos Externos e Segurança} (\textasciitilde51,3\%). Temas como \textit{Saúde} (\textasciitilde43,7\%) e \textit{Educação} (\textasciitilde41,2\%) são intermediários; \textit{Agricultura} (\textasciitilde35,3\%) e \textit{Direitos Humanos} (\textasciitilde25,3\%) têm menor incidência relativa.

\begin{figure}[!htbp]
\centering
\includegraphics[width=0.9\textwidth]{figures/theme_coverage.png}
\caption{Cobertura temática (proporção de entidades)}
\end{figure}

O ordenamento por proporção sugere centralidade de agendas de competitividade industrial, digitalização e cadeias de valor, bem como a transversalidade da pauta ambiental.

\paragraph{Orçamento declarado}
\begin{table}[!htbp]
\centering
\caption{Medidas-resumo de l\_ln\_max\_budget}
\begin{tabular}{lr}
\toprule
metric & value \\
\midrule
count & 12,468.000 \\
mean & -inf \\
std & NaN \\
min & -inf \\
p25 & 10.127 \\
median & 11.513 \\
p75 & 12.928 \\
max & 24.658 \\
\bottomrule
\end{tabular}

\end{table}

As estatísticas de orçamento máximo declarado (log natural) indicam mediana em torno de 11,5 e quartis aproximadamente entre 10,1 e 12,9. A presença de valores inválidos/extremos na base administrativa (por exemplo, ocorrências infinitas) distorce a média e o desvio-padrão, recomendando foco em medidas robustas (mediana, intervalos interquartílicos) e rotinas de limpeza nos exercícios inferenciais.

\begin{figure}[!htbp]
\centering
\includegraphics[width=0.75\textwidth]{figures/budget_ln_hist.png}
\caption{Distribuição de l\_ln\_max\_budget}
\end{figure}

A distribuição apresenta assimetria e cauda à direita compatíveis com heterogeneidade de porte organizacional, sugerindo coexistência de grandes associações/empresas e organizações menores.

\paragraph{Número de temas por lobista}
\begin{figure}[!htbp]
\centering
\includegraphics[width=0.75\textwidth]{figures/themes_per_lobbyist_hist.png}
\caption{Número de temas por lobista}
\end{figure}

A distribuição do número de temas por entidade indica a coexistência de atores multi-temáticos e especializados. Esse traço é relevante para a modelagem, pois sugere que a intensidade de esforço (extensivo vs. intensivo) varia com o perfil organizacional e o ambiente regulatório dos domínios.

\paragraph{Síntese}
Em conjunto, os resultados descritivos apontam para um ecossistema plural, geograficamente ancorado em polos institucionais centrais, com dinamismo temporal recente e agendas orientadas por digitalização, competitividade industrial e sustentabilidade. Esses padrões informam as escolhas de especificação nos capítulos seguintes, notadamente a estratificação por perfis organizacionais, a construção de domínios temáticos e o controle para tendências temporais.

\paragraph{Discussão e interpretação}
Os resultados descritivos delineiam um panorama abrangente do universo de lobistas registados junto às instituições europeias. Em primeiro lugar, a distribuição por categoria revela a coexistência de diferentes perfis organizacionais (\textit{Business}, \textit{NGOs} e \textit{Other}), com magnitudes comparáveis entre atores empresariais e organizações da sociedade civil. Essa composição é compatível com a literatura sobre pluralismo organizacional e competição por acesso institucional no contexto da União Europeia, sugerindo um campo de ação onde interesses difusos e concentrados buscam simultaneamente agenda e influência.

No plano geográfico, observa-se forte concentração em Estados-Membros centrais e com infraestrutura institucional robusta. Destacam-se Bélgica (\textasciitilde18,2\%), Alemanha (\textasciitilde14,0\%) e França (\textasciitilde9,3\%), seguidas por Países Baixos, Espanha, Reino Unido e Itália (\textasciitilde6\% cada). Há ainda presença extracomunitária não desprezível (Estados Unidos \textasciitilde4,5\%), o que evidencia a atratividade regulatória do mercado europeu e a permeabilidade do \textit{lobbying} transnacional. Esses achados são consistentes com hipóteses de \textit{venue shopping} e vantagens de proximidade institucional (Bruxelas/Strasburgo) para atividades de representação de interesses.

Temporalmente, as frequências por ano indicam aceleração recente dos registos, com 2023 concentrando \textasciitilde17,9\% das entradas no período observado. Picos intermediários (2015--2016; 2020--2022) são compatíveis com ciclos legislativos, janelas regulatórias e mudanças incrementais no regime de transparência, fatores que tendem a alterar a propensão ao registro. A expansão no pós-2020 pode refletir a reconfiguração de estratégias após as restrições pandémicas, além da ênfase em agendas de transição digital e verde.

Quanto à cobertura temática, a incidência é mais elevada em \textit{Infraestrutura e Indústria} (\textasciitilde68,3\%), \textit{Tecnologia} (\textasciitilde67,9\%) e \textit{Economia e Comércio} (\textasciitilde67,4\%), seguidas por \textit{Ambiente e Clima} (\textasciitilde64,7\%) e \textit{Assuntos Externos e Segurança} (\textasciitilde51,3\%). Temas como \textit{Saúde} (\textasciitilde43,7\%) e \textit{Educação} (\textasciitilde41,2\%) ocupam posição intermediária, ao passo que \textit{Agricultura} (\textasciitilde35,3\%) e \textit{Direitos Humanos} (\textasciitilde25,3\%) apresentam menor incidência relativa. Em conjunto, esse perfil sugere: (i) centralidade de agendas de competitividade industrial, digitalização e cadeias de valor; (ii) transversalidade da pauta ambiental como condicionante regulatória; e (iii) segmentação de atores com missões setoriais mais estreitas ou normativas, potencialmente menos numerosos.

A distribuição do número de temas por lobista sugere coexistência de atores multi-temáticos, capazes de cobrir diversas frentes de política pública, e de atores especializados com foco estreito. Tal heterogeneidade é relevante para a modelagem empírica, pois a intensidade de esforço (extensivo vs. intensivo) pode variar sistematicamente com o tipo de organização e com o ambiente regulatório dos diferentes domínios.

No que se refere ao orçamento máximo declarado (em log natural), as medidas-resumo apontam mediana próxima de 11,5 e quartis aproximados entre 10,1 e 12,9. Identificam-se valores inválidos/extremos na base administrativa (por exemplo, ocorrências infinitas), que distorcem a média e o desvio-padrão; por isso, a interpretação deve privilegiar estatísticas robustas (mediana e intervalos interquartílicos) e, quando pertinente, rotinas de limpeza e estratégias robustas nos exercícios inferenciais. Substantivamente, a dispersão é compatível com a coexistência de grandes associações/empresas e organizações de menor porte, com implicações para capacidades de acesso e agenda-setting.

Em síntese, as evidências descritivas apontam para um ecossistema plural, geograficamente ancorado em polos institucionais centrais, com dinamismo temporal recente e agendas orientadas por digitalização, competitividade industrial e sustentabilidade. Esses padrões informam as escolhas de especificação nos capítulos seguintes, notadamente a estratificação por perfis organizacionais, a construção de domínios temáticos e o controle para tendências temporais.


\section{Lobby e os parlamentares: tendências, intensidade e distribuição}
\label{sec:resultados_descritica}

Esta seção apresenta uma análise descritiva sistemática dos dados utilizados para investigar os efeitos do lobbying na atividade parlamentar dos deputados do \acrshort{pe}. A abordagem adotada segue uma estratégia analítica multinível, iniciando com padrões agregados gerais e progredindo para análises desagregadas mais específicas. Esta progressão metodológica permite compreender tanto as tendências globais quanto os mecanismos específicos que operam no nível individual e temporal.

O conjunto de dados constitui um painel balanceado que combina informações sobre atividade parlamentar (perguntas) e intensidade de lobbying (reuniões) para 1.727 deputados ao longo de 63 meses, de agosto de 2019 a outubro de 2024 em 9 domínios de política pública. Esta estrutura temporal permite capturar variações tanto na dimensão \textit{cross-sectional} (entre deputados e domínios) quanto longitudinal (evolução temporal), fornecendo a base empírica necessária para estratégias de identificação causal robustas. A tabela \ref{tab:mep_treatment_stats} apresenta as estatísticas agregadas do painel de dados.

\begin{table}[htbp]
    \centering
    \caption{Estatísticas agregadas de tratamento por deputado}
    \label{tab:mep_treatment_stats}
    \begin{tabular}{lr}
    \toprule
    \textbf{Estatística} & \textbf{Valor} \\
    \midrule
    Total de reuniões & 235.638 \\
    Total de perguntas & 118.238 \\
    Total de deputados únicos & 1{,}727 \\
    Deputados que receberam tratamento & 804 \\
    Taxa de tratamento por deputado (\%) & 46{,}5\% \\
    \midrule
    \textbf{Entre deputados tratados:} & \\
    Reuniões médias por deputado & 293{,}1 \\
    Reuniões medianas por deputado & 109{,}0 \\
    Desvio padrão & 468{,}9 \\
    \midrule
    \textbf{Correlação agregada:} & \\
    Correlação reuniões-perguntas & 0{,}05 \\
    \bottomrule
    \end{tabular}
\end{table}

Considerando a unidade de análise a tríade MEP-domínio-mês, temos 979.209 observações com taxa de completude de 100\%. Esta estrutura balanceada é metodologicamente vantajosa, pois elimina preocupações com viés de seleção decorrente de atrito amostral e garante que as estimativas não sejam distorcidas por padrões de observações ausentes.

A cobertura temporal de agosto de 2019 a outubro de 2024 é particularmente relevante por abranger períodos de intensa atividade legislativa europeia, incluindo a transição entre legislaturas e eventos político-econômicos significativos. Destaca-se, nesse intervalo, o impacto da pandemia de COVID-19, que afetou profundamente tanto a dinâmica da atividade parlamentar quanto as estratégias de lobbying. A pandemia resultou em mudanças substanciais nos modos de trabalho do Parlamento Europeu, com a adoção de sessões remotas e restrições a reuniões presenciais, o que pode ter alterado padrões de interação entre deputados e grupos de interesse. Assim, a análise cobre não apenas períodos de normalidade institucional, mas também um contexto de crise sanitária global, permitindo investigar como choques exógenos desse tipo influenciam o comportamento político e o lobbying.

A \autoref{fig:time_series} apresenta a evolução temporal das variáveis principais no nível mais agregado, revelando padrões que são fundamentais para compreender a dinâmica do sistema político europeu ao longo do período estudado.

\begin{figure}[htbp]
\centering
\includegraphics[width=\textwidth]{figures/descriptive_plots/fig1_time_series_meetings_questions.pdf}
\caption{Evolução temporal da atividade parlamentar e de lobbying}
\label{fig:time_series}
% \note{O painel superior esquerdo mostra os totais mensais agregados de perguntas e reuniões. O painel superior direito apresenta as médias mensais por observação MEP-domínio. O painel inferior esquerdo mostra a evolução da proporção de observações com atividade de lobbying. O painel inferior direito apresenta a estabilidade da correlação contemporânea entre as variáveis ao longo do tempo.}
\end{figure}

A análise da evolução temporal (Figura \ref{fig:time_series}) revela uma dinâmica complexa na interação entre a atividade de lobbying (reuniões) e a atividade parlamentar (perguntas). O padrão mais saliente é a \textbf{divergência de tendências} a partir do início de 2022. Enquanto o volume de perguntas parlamentares (linha laranja) permanece relativamente estável ao longo de todo o período, oscilando dentro de uma faixa consistente - com leve tendência de queda -, o número de reuniões de lobby (linha azul) apresenta um crescimento acentuado e um aumento expressivo da volatilidade a partir de 2022.

A estabilidade no número de perguntas parlamentares sugere que este instrumento funciona mais como uma ferramenta de rotina da fiscalização e posicionamento político, menos elástica às flutuações de curto prazo da agenda legislativa mais contenciosa. Além disso, ambas as séries exibem uma clara \textbf{sazonalidade}, com quedas de atividade que correspondem aos períodos de recesso do calendário parlamentar, um fator institucional que molda o ritmo do trabalho político.

Do ponto de vista metodológico, estes padrões são cruciais. A tendência de crescimento no lobby e a sazonalidade em ambas as séries justificam a inclusão de efeitos fixos de tempo (ex: mês-ano) nos modelos econométricos, para controlar choques temporais comuns e variações sazonais que poderiam confundir a estimação dos efeitos. A divergência entre as séries a partir de 2022 também sugere a importância de se testar a estabilidade dos parâmetros do modelo ao longo do tempo.


% \subsection{Padrões de participação: análise agregada por deputado}

Complementando a análise temporal, é fundamental examinar os padrões de participação no nível individual dos deputados. Esta perspectiva agregada revela a distribuição da atividade de lobbying entre os parlamentares e fornece informações sobre a concentração e heterogeneidade dos fenômenos estudados.



As \autoref{fig:proportion_meetings}, \autoref{fig:meetings_hist} e \autoref{fig:correlation_meetings_questions} apresentam uma análise dos padrões de participação agregados por deputado, revelando aspectos da distribuição da atividade de lobbying no \acrshort{pe} que impactam a identificação causal.


\begin{figure}[htbp]
    \centering
    \includegraphics[width=\textwidth]{figures/descriptive_plots/fig2_proportion_meetings.pdf}
    \caption{Evolução temporal da proporção de \acrshort{mpe}s que participaram de reuniões de lobbying}
    \label{fig:proportion_meetings}
    % \note{O painel superior esquerdo mostra os totais mensais agregados de perguntas e reuniões. O painel superior direito apresenta as médias mensais por observação MEP-domínio. O painel inferior esquerdo mostra a evolução da proporção de observações com atividade de lobbying. O painel inferior direito apresenta a estabilidade da correlação contemporânea entre as variáveis ao longo do tempo.}
\end{figure}

\begin{figure}[htbp]
    \centering
    \includegraphics[width=\textwidth]{figures/descriptive_plots/fig3.1_meetings_hist.pdf}
    \caption{Distribuição de reuniões por \acrshort{mpe}}
    \label{fig:meetings_hist}
    % \note{O painel superior esquerdo mostra os totais mensais agregados de perguntas e reuniões. O painel superior direito apresenta as médias mensais por observação MEP-domínio. O painel inferior esquerdo mostra a evolução da proporção de observações com atividade de lobbying. O painel inferior direito apresenta a estabilidade da correlação contemporânea entre as variáveis ao longo do tempo.}
\end{figure}

\begin{figure}[htbp]
    \centering
    \includegraphics[width=\textwidth]{figures/descriptive_plots/fig3_correlation_meetings_questions.pdf}
    \caption{Evolução temporal da correlação entre reuniões e perguntas}
    \label{fig:correlation_meetings_questions}
    % \note{O painel superior esquerdo mostra os totais mensais agregados de perguntas e reuniões. O painel superior direito apresenta as médias mensais por observação MEP-domínio. O painel inferior esquerdo mostra a evolução da proporção de observações com atividade de lobbying. O painel inferior direito apresenta a estabilidade da correlação contemporânea entre as variáveis ao longo do tempo.}
\end{figure}


A análise revela três características fundamentais da distribuição de tratamento. Primeiro, existe \textbf{participação substancial mas não universal} (\autoref{fig:proportion_meetings}): 46,5\% dos deputados (804 de 1.727) receberam pelo menos uma reunião de lobbying durante o período estudado. Esta proporção indica que o lobbying é um fenômeno disseminado mas não ubíquo no sistema parlamentar europeu.

Segundo, observa-se \textbf{concentração extrema} na intensidade de tratamento (\autoref{fig:meetings_hist}). Entre os deputados que receberam lobbying, a distribuição é altamente assimétrica: enquanto a mediana é de 109 reuniões por deputado, a média é de 293,1 reuniões, indicando que uma minoria de parlamentares concentra uma quantidade desproporcional da atividade lobista. O caso extremo de um deputado com 4.274 reuniões ilustra esta concentração.

Terceiro, a \textbf{correlação agregada} entre reuniões e perguntas totais por deputado é surpreendentemente baixa (0,05), contrastando com correlações mais elevadas observadas no nível temporal (\autoref{fig:correlation_meetings_questions}). Este padrão sugere que os efeitos do lobbying podem ser mais evidentes em frequências temporais específicas do que em padrões de atividade agregados de longo prazo.


Estes padrões agregados têm implicações importantes para a identificação causal. A concentração do tratamento em uma minoria de deputados sugere que estratégias de identificação baseadas em variação \textit{cross-sectional} podem sofrer de poder estatístico limitado. Simultaneamente, a variação substancial na intensidade de tratamento entre deputados tratados fornece fonte valiosa de identificação para estimativas de dose-resposta.

A baixa correlação agregada, combinada com correlações temporais mais elevadas, indica que a identificação causal pode beneficiar-se de estratégias que explorem variação temporal \textit{within-individual} ao invés de \textit{cross-sectional between-individual}. Esta evidência preliminar orienta a especificação de modelos com efeitos fixos de deputado para controlar heterogeneidade não observada invariante no tempo.


% \subsection{Heterogeneidade entre domínios de política pública}

A terceira dimensão da análise agregada examina a variação entre domínios de política pública. Esta heterogeneidade setorial é teoricamente relevante porque diferentes áreas de política podem apresentar características distintas em termos de complexidade técnica, interesse econômico e organização de grupos de pressão, afetando tanto a demanda por lobbying quanto a responsividade parlamentar.

% \paragraph{Padrões de tratamento por domínio}

A análise da Tabela \ref{tab:domain_treatment_rates} revela uma heterogeneidade sistemática entre os domínios de política pública, que se manifesta de forma distinta na penetração e na intensidade do lobby.

\begin{table}[htbp]
    \centering
    \caption{Taxa de tratamento por domínio: deputados únicos que receberam lobbying}
    \label{tab:domain_treatment_rates}
    \begin{tabularx}{\textwidth}{>{\raggedright\arraybackslash}X c c c c c c}
        
        \hline
        \tiny{\textbf{Domínio}} & \tiny{\textbf{MEPs}} & \tiny{\textbf{MEPs}} & \tiny{\textbf{Reuniões}} & \tiny{\textbf{Perguntas}} & \tiny{\textbf{Penetração}} & \tiny{\textbf{Intensidade}} \\
        & \tiny{\textbf{(A)}} & \tiny{\textbf{Tratados (B)}} & \tiny{\textbf{(C)}} & \tiny{\textbf{}} & \tiny{\textbf{(B/A \%)}} & \tiny{\textbf{Média (C/B)}} \\
        \hline 
        \tiny{Tecnologia} & \tiny{1727} & \tiny{789} & \tiny{32.067} & \tiny{9.918} & \tiny{45,69} & \tiny{40,64} \\
        \tiny{Economia e Comércio} & \tiny{1727} & \tiny{788} & \tiny{34.762} & \tiny{13.884} & \tiny{45,63} & \tiny{44,11} \\
        \tiny{Assuntos Externos e Segurança} & \tiny{1727} & \tiny{783} & \tiny{28.041} & \tiny{13.462} & \tiny{45,34} & \tiny{35,81} \\
        \tiny{Infraestrutura e Indústria} & \tiny{1727} & \tiny{782} & \tiny{32.795} & \tiny{10.544} & \tiny{45,28} & \tiny{41,94} \\
        \tiny{Meio Ambiente e Clima} & \tiny{1727} & \tiny{777} & \tiny{32.038} & \tiny{14.665} & \tiny{44,99} & \tiny{41,23} \\
        \tiny{Saúde} & \tiny{1727} & \tiny{769} & \tiny{23.868} & \tiny{20.602} & \tiny{44,53} & \tiny{31,04} \\
        \tiny{Educação} & \tiny{1727} & \tiny{738} & \tiny{18.670} & \tiny{2.949} & \tiny{42,73} & \tiny{25,30} \\
        \tiny{Direitos Humanos} & \tiny{1727} & \tiny{727} & \tiny{17.194} & \tiny{26.482} & \tiny{42,10} & \tiny{23,65} \\
        \tiny{Agricultura} & \tiny{1727} & \tiny{711} & \tiny{16.203} & \tiny{5.732} & \tiny{41,17} & \tiny{22,79} \\
        \hline
        
    \end{tabularx}
\end{table}



Em termos de \textbf{penetração} (a proporção de deputados que recebem ao menos uma reunião), a variação entre os domínios é moderada, oscilando entre 41,2\% (Agricultura) e 45,7\% (Tecnologia). Essa pequena amplitude sugere que o lobby, como prática, é uma atividade transversal e disseminada por todo o Parlamento Europeu, não se restringindo a nichos específicos. Ainda assim, o padrão é teoricamente consistente: domínios ligados à regulação econômica e de alta complexidade (Tecnologia, Economia e Comércio, Infraestrutura) apresentam as maiores taxas, refletindo os elevados interesses em jogo \cite{mahoney_lobbying_2007}.

A heterogeneidade torna-se muito mais acentuada quando se analisa a \textbf{intensidade média do tratamento} (o número médio de reuniões por deputado "tratado"). Aqui, a variação é extrema: um deputado ativo em "Economia e Comércio" recebe, em média, 44 reuniões, quase o dobro das 23 reuniões de um deputado ativo em "Agricultura". Esta é a dimensão onde a alocação estratégica de recursos se torna mais evidente. Domínios com altas consequências econômicas e complexidade técnica atraem não apenas um pouco mais de lobistas, mas um esforço de influência massivamente mais concentrado, validando a tese de que os recursos de lobby fluem para as arenas de maior saliência e complexidade regulatória \cite{kluver_informational_2012}.

É notável também o contraste entre a intensidade do lobby e o volume de atividade parlamentar. O domínio de "Direitos Humanos", por exemplo, apresenta o maior número de perguntas parlamentares (26.482), mas uma das menores taxas de penetração e a segunda menor intensidade de lobby. Este padrão sugere que a dinâmica política varia conforme o tema. Enquanto domínios econômicos são caracterizados por um lobby intenso e técnico, temas de grande apelo público como "Direitos Humanos" podem ser mais influenciados por estratégias de \textit{advocacy} e pela sinalização política dos parlamentares através de instrumentos como as perguntas, em vez do lobby direto medido por reuniões.

% Inflação de zeros
A unidade de análise adotada — deputado-domínio-mês — revela uma alta proporção de observações com valor zero (>92\%) tanto para reuniões quanto para perguntas. Contudo, esta "inflação de zeros" não representa um problema estatístico, mas sim uma característica substantiva do comportamento parlamentar: a \textbf{especialização temática}. Como documentado na literatura \cite{schiller1995senators, burden2015personal}, parlamentares concentram sua atividade em um subconjunto de domínios, resultando em zero atividade na maioria das combinações deputado-domínio. A atividade de lobby, por sua vez, segue essa especialização, direcionando-se aos parlamentares já ativos em cada tema.

A natureza "artificial" dessa inflação de zeros é confirmada quando os dados são agregados. Ao considerar apenas os domínios em que um deputado demonstra alguma atividade, a proporção de zeros cai para 47,1\% para perguntas e 55,8\% para reuniões. Em um nível agregado por domínio-mês, a inflação de zeros torna-se negligível (3,2\% e 0\%, respectivamente), indicando que há atividade constante em todos os domínios quando se considera o conjunto de parlamentares relevantes. Portanto, a escolha do modelo econométrico \acrshort{ppml} justifica-se pela natureza da variável dependente (dados de contagem com excesso de zeros), evitando os vieses de modelos lineares com transformações logarítmicas.

A análise descritiva delineia um ecossistema de lobbying no \acrshort{pe} caracterizado por \textit{(i)} dinamismo temporal acentuado, \textit{(ii)} concentração da atividade em um subconjunto de deputados e \textit{(iii)} heterogeneidade entre os domínios de política pública. Longe de serem meramente contextuais, esses padrões empíricos são a principal justificativa para a estratégia econométrica adotada nesta tese, detalhada no Capítulo \ref{chapter:metodologia}.

Primeiramente, a combinação de uma correlação agregada próxima de zero entre reuniões e perguntas com uma alta concentração da atividade de lobby em poucos deputados demonstra a inadequação de abordagens puramente \textit{cross-sectional}. A fonte de identificação mais promissora reside na variação \textit{within-individual}, ou seja, em como a atividade de um mesmo deputado muda ao longo do tempo em resposta a variações no lobby que recebe.

Em segundo lugar, a presença de tendências temporais claras, sazonalidade e choques de atividade específicos por domínio (como o aumento expressivo do lobby a partir de 2022) torna imperativo o uso de uma estrutura robusta de efeitos fixos. A escolha por efeitos fixos de alta dimensão — \textbf{país×tempo}, \textbf{partido×tempo} e \textbf{domínio×tempo} — é uma resposta direta a esses padrões, permitindo controlar uma vasta gama de fatores de confusão não observados e isolar de forma mais crível o efeito causal de interesse.

Terceiro, a heterogeneidade da intensidade do lobby entre os domínios valida a decisão de não se limitar a um único efeito médio, motivando as análises de heterogeneidade que exploram como o impacto do lobby pode variar conforme o contexto temático.

Finalmente, a natureza da variável dependente — um dado de contagem com uma alta proporção de zeros explicada pela especialização temática — fundamenta a escolha do estimador \acrshort{ppml}. Esta abordagem modela adequadamente a estrutura dos dados, evitando os vieses de modelos lineares e reconhecendo a especialização como um comportamento substantivo, e não como um mero problema estatístico.

Em suma, a análise descritiva não apenas caracteriza o fenômeno em estudo, mas também estabelece as bases empíricas que guiam e validam a estratégia de identificação causal empregada nos capítulos seguintes para testar as hipóteses centrais desta pesquisa.


\subsection{Análise de efeitos do lobby}
\label{sec:resultados_efeitos}



\subsubsection{Teste da hipótese 2}

A primeira vista, as \acrshort{ong}s parecem exercer um maior efeito. A figura \ref{fig:effect_linear_ppml_treatments} ilustra os resultados do modelo para diferentes tipos de tratamento: reuniões com \acrshort{ong}s, com empresas e com outros tipos de atores. 


\begin{figure}[htbp]
    \centering
    \includegraphics[width=\textwidth]{figures/h2_test/fig_coeff_treatments_overall.pdf}
    \caption{Efeito esperado \textit{ceteris paribus}: especificação linear (\acrshort{ppml}) para cada tratamento}
    \label{fig:effect_linear_ppml_treatments}
    \note{Cada ponto azul representa a estimativa do coeficiente associado a \textit{meetings} para um tratamento específico, refletindo o efeito marginal esperado de reuniões sobre o número de perguntas parlamentares, mantidos constantes os efeitos fixos. As linhas horizontais correspondem aos intervalos de confiança de 95\% para cada estimativa, indicando a incerteza estatística. A linha tracejada vermelha indica o efeito médio estimado para todos os tratamentos, servindo como referência para comparação entre tratamentos.}
\end{figure}

É possível identificar que o efeito de reuniões com \acrshort{ong}s é consideravelmente maior do que com empresas. Contudo, esse é o efeito marginal esperado para cada tipo de organização. Não sendo suficiente para o teste da hipótese 2, uma vez que o teste de que as empresas possuam um efeito meior do que outros tipos de agentes demanda considerar o efeito total do lobbying de cada ator.

Para tanto, devemos olhar para o "estoque" de reuniões, não somente para o "fluxo". Como explicado no capítulo \ref{chapter:metodologia}, utilizamos um modelo de binomial negativo para estimar a quantidade de reuniões de um agente baseado em suas características. Os resultados desse modelo estão apresentados na tabela \ref{tab:meetings_nb_centered}.

% \begin{table}
\centering
\begin{talltblr}[         %% tabularray outer open
entry=none,label=none,
note{}={+ p \num{< 0.1}, * p \num{< 0.05}, ** p \num{< 0.01}, *** p \num{< 0.001}},
]                     %% tabularray outer close
{                     %% tabularray inner open
colspec={Q[]Q[]Q[]Q[]},
column{2-4}={}{halign=c,},
column{1}={}{halign=l,},
hline{38}={1-4}{solid, black, 0.05em},
}                     %% tabularray inner close
\toprule
& NB meetings model & NB meetings model interaction & NB meetings model (centered) \\ \midrule %% TinyTableHeader
(Intercept) & \num{-1.025}* & \num{-0.516} & \num{0.978}* \\
& (\num{0.455}) & (\num{0.455}) & (\num{0.432}) \\
l\_categoryBusiness & \num{0.105}* & \num{-3.626}*** & \num{0.188}*** \\
& (\num{0.044}) & (\num{0.253}) & (\num{0.044}) \\
l\_categoryOther & \num{-0.337}*** & \num{0.881}*** & \num{-0.262}*** \\
& (\num{0.047}) & (\num{0.203}) & (\num{0.047}) \\
l\_ln\_max\_budget & \num{0.141}*** & \num{0.118}*** &  \\
& (\num{0.007}) & (\num{0.011}) &  \\
l\_agriculture & \num{-0.030} & \num{0.008} & \num{0.008} \\
& (\num{0.036}) & (\num{0.035}) & (\num{0.035}) \\
l\_economics\_and\_trade & \num{0.371}*** & \num{0.290}*** & \num{0.290}*** \\
& (\num{0.043}) & (\num{0.042}) & (\num{0.042}) \\
l\_education & \num{-0.030} & \num{-0.017} & \num{-0.017} \\
& (\num{0.037}) & (\num{0.036}) & (\num{0.036}) \\
l\_environment\_and\_climate & \num{0.237}*** & \num{0.200}*** & \num{0.200}*** \\
& (\num{0.042}) & (\num{0.041}) & (\num{0.041}) \\
l\_foreign\_and\_security\_affairs & \num{0.227}*** & \num{0.160}*** & \num{0.160}*** \\
& (\num{0.036}) & (\num{0.035}) & (\num{0.035}) \\
l\_health & \num{0.066}+ & \num{0.025} & \num{0.025} \\
& (\num{0.035}) & (\num{0.034}) & (\num{0.034}) \\
l\_human\_rights & \num{0.253}*** & \num{0.198}*** & \num{0.198}*** \\
& (\num{0.039}) & (\num{0.037}) & (\num{0.037}) \\
l\_infrastructure\_and\_industry & \num{0.065} & \num{0.029} & \num{0.029} \\
& (\num{0.043}) & (\num{0.041}) & (\num{0.041}) \\
l\_technology & \num{0.026} & \num{0.009} & \num{0.009} \\
& (\num{0.040}) & (\num{0.039}) & (\num{0.039}) \\
l\_categoryBusiness × l\_ln\_max\_budget &  & \num{0.301}*** &  \\
&  & (\num{0.020}) &  \\
l\_categoryOther × l\_ln\_max\_budget &  & \num{-0.090}*** &  \\
&  & (\num{0.015}) &  \\
l\_ln\_max\_budget\_c &  &  & \num{0.118}*** \\
&  &  & (\num{0.011}) \\
l\_categoryBusiness × l\_ln\_max\_budget\_c &  &  & \num{0.301}*** \\
&  &  & (\num{0.020}) \\
l\_categoryOther × l\_ln\_max\_budget\_c &  &  & \num{-0.090}*** \\
&  &  & (\num{0.015}) \\
Num.Obs. & \num{4647} & \num{4647} & \num{4647} \\
RMSE & \num{14.82} & \num{16.58} & \num{16.58} \\
\bottomrule
\end{talltblr}
\end{table}


No modelo, utilizamos a categoria \acrshort{ong} como referência. Assim, podemos verificar que, de fato, as empresas tendem a fazer mais reuniões do que \acrshort{ong}s, já que o coeficiente  da dummy representando
%  se o agente é o ou não empresa (\textit{l_categoryBusines}) foi signatificato e maior do que zero \(0,188\) quando consideramos o modelo centralizado. 

Além do fato de ser ou não empresa, devemos considerar a eficiência de alocação de recursos. Isto é, estimar o efeito de o aumento marginal no orçamento dedicado ao lobby. Para isso, adicionamos o fator de interção entre categoria e o orçamento 
% (\textit{l_ln_max_budget}). Esse coeficiente nos traz o quanto o aumento percentual no orçamento de uma empresa aumenta a quantidade de reuniões relaizadas. Esse coeficiente (0,301) foi consideravelmente superior ao das ongs (0,118).

Esse resultado nos indica que as empresas possuem uma eficiência alocativa maior. Conseguem, portanto, mais reuniões para cada da aumento percentual em orçamento dedicado ao lobby. A figura \ref{fig:h2_pred_meetings} demonstra a quantidade de reuniões estimada por categoria em função do orçamento em escala logarítimica.

\begin{figure}[htbp]
    \centering
    \includegraphics[width=\textwidth]{figures/h2_test/fig_pred_meetings_vs_budget_centered_by_category.pdf}
    \caption{Quantidade de reuniões estimadas \textit{ceteris paribus}}
    \label{fig:h2_pred_meetings}
    \note{}
\end{figure}

Podemos verificar que, até um orçamento de aproximadamente 27.000 dólares ($\approx exp(12.5)$), esperamos um número maior de reuniões realizados por \acrshort{ong}s do que empresas. A partir desse volume, passamos a esperar mais reuniões realizadas por empresas. Essa diferença tende a aumentar de maneira exponencial, indicando que empresas com orçamentos maiores tendem a obter muito mais reuniões do que \acrshort{ong}s com o mesmo orçamento.

% falar mais

Dito isso, podemos estimar o efeito total esperado ao multiplicarmos a quantidade estimada de reuniões pelo coeficiente obtido de cada reunião por categoria. O resultado pode dessa análise está resumido na figura \ref{fig:h2_total_effects}.

\begin{figure}[htbp]
    \centering
    \includegraphics[width=\textwidth]{figures/h2_test/fig_total_effect_vs_budget_by_category.pdf}
    \caption{Efeito total estimado por categoria \textit{ceteris paribus}}
    \label{fig:h2_total_effects}
    \note{}
\end{figure}

Ainda que o efeito marginal médio de uma reunião realizada por uma \acrshort{ong} seja maior, ao incluirmos os efeitos da eficiência alocativa, verificamos que as empresas com grandes recursos tendem, de fato, a conseguirem maiores resultados - mensurados pelo aumento da atividade parlamentar no tema de interesse. Com um orçamento entre, aproximadamente, 8,8 milhões de dólares ($\approx exp(16)$) e 40 milhões de dólares ($\approx exp(17,5)$) esperamos um efeito total similiar entre \acrshort{ong}s e empresas. Acima de 40 milhões, porém, esperamos um efeito maior de empresas. Abaixo de 8,8 milhões, esperamos um efeito maior de \acrshort{ong}s.

Esses resultados vão ao encontro de hipóteses levantadas na literatura, porém traz nuances que podem ser exploradas. Como indicam ...... grandes empresas tendem a conseguir mais resultado com lobby em comparação a outros tipos de entidade. Mas esse efeito não é linear e extrapolável para qualquer empresa. As pequenas e médias tendem a ter muito mais dificuldades em comparação com \acrshort{ong}s. 

Uma das exlpicações que podemos encontrar está na questão do reconhecimento da legitimidade. \acrshort{ong}s tendem a ter maior legitimidade perante os \acrshort{mpe}s em comparação com empresas em geral (CITAR). 

Empresas maiores conseguem alocar mais recursos e obter mais informações de modo a subisidiar os parlamentares (CITAR), conseguindo maiores efeitos no comportamento parlamentar. Associando isso ao fato de os efeitos marginais decrecentes serem pequenos (ver ver tabela \ref{tab:ppml_h1_both}), podemos concluir que o aumento de recursos dedicados ao lobby por empresas aumenta consideravelmente os efeitos sobre o comportamento parlamentar.

Esses resultados correspondem ao que na literatura se aponta como desigualdade de representação. Quanto mais recursos são aplicados no lobby, os resultados tendem a aumentar mais do que prpoporcinalmente, favorecendo grandes players.
\subsection{O papel da saliência temática na influência do lobby}

A Hipótese 3 postula que, em temas de maior saliência, o lobby exercido por organizações não empresariais (como as \acrshort{ong}s) tem uma maior probabilidade de influenciar a atividade legislativa dos \acrshort{mpe}s em comparação com o lobby de organizações empresariais. A lógica subjacente é que, quando um tema está sob intenso escrutínio público, os parlamentares se tornam mais sensíveis a argumentos que ressoam com a opinião pública e a interesses difusos, frequentemente representados por \acrshort{ong}s.

Para testar esta hipótese, mantivemos a estrutura do modelo \acrshort{ppml} com efeitos fixos, garantindo a consistência com as análises anteriores. A principal diferença metodológica foi a introdução de uma variável para capturar a saliência de um tema e a sua interação com os diferentes tipos de lobistas.

A saliência foi operacionalizada como uma proxy baseada na intensidade da atividade de lobby, uma abordagem que encontra respaldo na literatura \cite{baumgartner2010agendas}. Especificamente, criamos uma variável (salience\_std) que mede o volume total de reuniões de lobby dentro de cada domínio temático para cada período mensal, padronizada para ter média zero e desvio padrão um. Um valor mais alto nesta variável indica que um tema atraiu mais atenção de todos os grupos de interesse, sendo, portanto, considerado mais saliente.

O modelo econométrico foi então especificado para incluir termos de interação entre cada categoria de lobista (Empresa, \acrshort{ong}, Outros) e a variável de saliência, apresentados na Equação \ref{eq:modelo_h3}. Esta especificação permite-nos estimar como o efeito marginal de uma reunião de cada tipo de ator varia em função do nível de saliência do tema. Os resultados da regressão estão sumarizados na Tabela \ref{tab:h3_interaction} e visualizados no gráfico de efeitos marginais na Figura \ref{fig:h3_marginal_effects}.

\begin{table}

\centering
\begin{talltblr}[         %% tabularray outer open
entry=none,label=none,
note{}={+ p \num{< 0.1}, * p \num{< 0.05}, ** p \num{< 0.01}, *** p \num{< 0.001}},
]                     %% tabularray outer close
{                     %% tabularray inner open
colspec={Q[]Q[]},
column{2}={}{halign=c,},
column{1}={}{halign=l,},
hline{14}={1-2}{solid, black, 0.05em},
}                     %% tabularray inner close
\hline
& PPML com Interação (H3) \\ \hline %% TinyTableHeader
Empresa (base) & \num{0.035}*** \\
& (\num{0.006}) \\
ONG (base) & \num{0.090}*** \\
& (\num{0.006}) \\
Outros (base) & \num{0.032}** \\
& (\num{0.010}) \\
Empresa x Saliência & \num{-0.022}*** \\
& (\num{0.005}) \\
ONG x Saliência & \num{-0.016}* \\
& (\num{0.007}) \\
Outros x Saliência & \num{-0.024}* \\
& (\num{0.011}) \\
Num.Obs. & \num{600237} \\
R2 & \num{0.253} \\
RMSE & \num{0.56} \\
Std.Errors & by: cl\_dt \\
FE: fe\_ct & X \\
FE: fe\_pt & X \\
FE: fe\_dt & X \\
\hline
\end{talltblr}
\label{tab:h3_interaction}
\end{table}



\begin{figure}[htbp]
    \centering
    \includegraphics[width=\textwidth]{figures/h3_test/fig_h3_marginal_effects_manual.png}
    \caption{Efeito do Lobby Condicional à Saliência do Tema}
    \label{fig:h3_marginal_effects}
    \note{O gráfico exibe o efeito marginal esperado de uma única reunião sobre o número de perguntas parlamentares (eixo Y) em diferentes níveis de saliência do tema (eixo X). As linhas representam a estimativa para cada categoria de ator, e as áreas sombreadas correspondem aos intervalos de confiança de 95\%, calculados via bootstrap.}
\end{figure}

A análise revela um padrão complexo que contradiz parcialmente, mas também enriquece, a Hipótese 3. Contrariamente à expectativa de que o efeito das \acrshort{ong}s aumentaria com a saliência, observamos que o efeito marginal de uma reunião \textbf{diminui} para todos os grupos à medida que um tema se torna mais saliente (coeficiente negativo para todos os grupos nas variáveis de interação). Este achado está em forte alinhamento com a literatura, que sugere que a influência do lobby direto decresce quando a opinião pública e a atenção da mídia se intensificam, forçando os parlamentares a se alinharem a considerações eleitorais mais amplas \cite{mahoney_lobbying_2007, kollman1998outside}.

No entanto, a análise revela uma heterogeneidade crucial na taxa dessa diminuição. Três pontos principais se destacam na Figura \ref{fig:h3_marginal_effects}. Em temas de baixa saliência (à esquerda do gráfico), o efeito das \acrshort{ong}s é similar estatisticamente ao de empresas e outros atores. Isso pode ser observado pela intersecção das áreas sombreadas das linhas, que indicam o intervalo de confiança de 95\% das estimativas.
    
À medida que a saliência aumenta (movendo-se para a direita no gráfico), a vantagem comparativa das \acrshort{ong}s se acentua significativamente. O efeito do lobby de empresas e de outros atores decai rapidamente, enquanto o efeito das \acrshort{ong}s se mostra muito mais resiliente, diminuindo a uma taxa consideravelmente menor.
    
Em temas de alta saliência, onde a influência de empresas e outros grupos se torna estatisticamente indistinguível de zero (seus intervalos de confiança cruzam a linha pontilhada), o efeito das \acrshort{ong}s permanece positivo, robusto e estatisticamente significativo. É precisamente neste contexto de maior escrutínio público que a sua influência relativa se torna mais pronunciada.

Os resultados validam a Hipótese 3. Em temas de maior saliência, o lobby de organizações não empresariais é, de fato, mais eficaz em aumentar a atividade parlamentar em comparação com o lobby empresarial. A nuance importante é que essa maior eficácia não se manifesta como um aumento absoluto do efeito, mas sim como uma resiliência superior à pressão do escrutínio público, o que amplia a sua vantagem comparativa.

Esta descoberta dialoga diretamente com a teoria sobre os recursos do lobby e os incentivos parlamentares. Em temas de baixa saliência, os parlamentares, focados em seus objetivos de formulação de políticas (\textit{policy-seeking}), podem valorizar o subsídio informacional técnico fornecido por empresas. Contudo, quando um tema ganha visibilidade, os incentivos de reeleição (\textit{vote-seeking}) tornam-se dominantes \cite{mayhew2004congress}. Nesse contexto, alinhar-se a interesses empresariais pode ter um custo político elevado, enquanto responder a \acrshort{ong}s, que detêm maior capital de legitimidade \cite{bunea2018legitimacy}, reforça a imagem pública do parlamentar.

Os achados confirmam que a influência do lobby é altamente contextual e que, sob o escrutínio público, a vantagem se desloca para os atores percebidos como representantes de interesses mais amplos e difusos.
