\subsubsection{Teste da hipótese 3: O Efeito do Lobby em Temas Salientes}

A Hipótese 3 postula que, em temas de maior saliência, o lobby exercido por organizações não empresariais (como as \acrshort{ong}s) tem uma maior probabilidade de influenciar a atividade legislativa dos \acrshort{mpe}s em comparação com o lobby de organizações empresariais. A lógica subjacente é que, quando um tema está sob intenso escrutínio público, os parlamentares se tornam mais sensíveis a argumentos que ressoam com a opinião pública e a interesses difusos, frequentemente representados por \acrshort{ong}s.

\paragraph{Método de Teste}
Para testar esta hipótese, mantivemos a estrutura do modelo \acrshort{ppml} com efeitos fixos, garantindo a consistência com as análises anteriores. A principal inovação metodológica foi a introdução de uma variável para capturar a \textbf{saliência} de um tema e a sua interação com os diferentes tipos de lobistas.

A saliência foi operacionalizada como uma proxy baseada na intensidade da atividade de lobby, uma abordagem que encontra respaldo na literatura \cite{baumgartner2010agendas}. Especificamente, criamos uma variável (salience\_std) que mede o volume total de reuniões de lobby dentro de cada domínio temático para cada período mensal, padronizada para ter média zero e desvio padrão um. Um valor mais alto nesta variável indica que um tema atraiu mais atenção de todos os grupos de interesse, sendo, portanto, considerado mais saliente.

O modelo econométrico foi então especificado para incluir termos de interação entre cada categoria de lobista (Empresa, \acrshort{ong}, Outros) e a variável de saliência. A fórmula central do modelo é:
\[ \text{perguntas} \sim \beta_1(\text{Empresa} \times \text{Saliência}) + \beta_2(\text{ONG} \times \text{Saliência}) + \dots + \text{Controles} + \text{Efeitos Fixos} \]
Esta especificação permite-nos estimar como o efeito marginal de uma reunião de cada tipo de ator varia em função do nível de saliência do tema.


\paragraph{Resultados}
Os resultados da regressão estão sumarizados na Tabela \ref{tab:h3_interaction} e visualizados no gráfico de efeitos marginais na Figura \ref{fig:h3_marginal_effects}.

% Inserir a tabela aqui
% \begin{table}

\centering
\begin{talltblr}[         %% tabularray outer open
entry=none,label=none,
note{}={+ p \num{< 0.1}, * p \num{< 0.05}, ** p \num{< 0.01}, *** p \num{< 0.001}},
]                     %% tabularray outer close
{                     %% tabularray inner open
colspec={Q[]Q[]},
column{2}={}{halign=c,},
column{1}={}{halign=l,},
hline{14}={1-2}{solid, black, 0.05em},
}                     %% tabularray inner close
\hline
& PPML com Interação (H3) \\ \hline %% TinyTableHeader
Empresa (base) & \num{0.035}*** \\
& (\num{0.006}) \\
ONG (base) & \num{0.090}*** \\
& (\num{0.006}) \\
Outros (base) & \num{0.032}** \\
& (\num{0.010}) \\
Empresa x Saliência & \num{-0.022}*** \\
& (\num{0.005}) \\
ONG x Saliência & \num{-0.016}* \\
& (\num{0.007}) \\
Outros x Saliência & \num{-0.024}* \\
& (\num{0.011}) \\
Num.Obs. & \num{600237} \\
R2 & \num{0.253} \\
RMSE & \num{0.56} \\
Std.Errors & by: cl\_dt \\
FE: fe\_ct & X \\
FE: fe\_pt & X \\
FE: fe\_dt & X \\
\hline
\end{talltblr}
\label{tab:h3_interaction}
\end{table}


\begin{figure}[htbp]
    \centering
    \includegraphics[width=\textwidth]{figures/h3_test/fig_h3_marginal_effects_manual.png}
    \caption{Efeito do Lobby Condicional à Saliência do Tema}
    \label{fig:h3_marginal_effects}
    \note{O gráfico exibe o efeito marginal esperado de uma única reunião sobre o número de perguntas parlamentares (eixo Y) em diferentes níveis de saliência do tema (eixo X). As linhas representam a estimativa para cada categoria de ator, e as áreas sombreadas correspondem aos intervalos de confiança de 95\%, calculados via bootstrap.}
\end{figure}

A análise revela um padrão complexo que contradiz parcialmente, mas também enriquece, a Hipótese 3. Contrariamente à expectativa de que o efeito das \acrshort{ong}s aumentaria com a saliência, observamos que o efeito marginal de uma reunião \textbf{diminui} para todos os grupos à medida que um tema se torna mais saliente. Este achado está em forte alinhamento com a literatura, que sugere que a influência do lobby direto decresce quando a opinião pública e a atenção da mídia se intensificam, forçando os parlamentares a se alinharem a considerações eleitorais mais amplas \cite{mahoney_lobbying_2007, kollman1998outside}.

No entanto, a análise revela uma heterogeneidade crucial na \textit{taxa} dessa diminuição. Três pontos principais se destacam na Figura \ref{fig:h3_marginal_effects}:
\begin{enumerate}
    \item \textbf{Superioridade das ONGs}: Em todos os níveis de saliência observados, o efeito marginal de uma reunião com uma \acrshort{ong} (linha verde) é consistentemente superior ao de uma reunião com uma empresa (linha vermelha/rosa) ou outros atores (linha azul).
    
    \item \textbf{Resiliência do Efeito das ONGs}: A inclinação da linha para as \acrshort{ong}s é visivelmente menos acentuada. Isso indica que, embora a sua eficácia também diminua com o aumento da saliência, ela o faz a uma taxa muito mais lenta. O efeito do lobby de empresas e de outros atores, por outro lado, deteriora-se rapidamente.
    
    \item \textbf{Perda de Significância Estatística}: Para temas de alta saliência (valores positivos no eixo X), os intervalos de confiança para empresas e outros atores cruzam a linha do zero, sugerindo que seu efeito se torna estatisticamente indistinguível de zero. O efeito das \acrshort{ong}s, contudo, permanece positivo e estatisticamente significativo em toda a faixa de saliência.
\end{enumerate}

\begin{enumerate}
    \item \textbf{Vantagem Comparativa das ONGs}: Em temas de baixa saliência (à esquerda do gráfico), o efeito das \acrshort{ong}s já é superior ao de empresas e outros atores. Contudo, a diferença é mais modesta.
    
    \item \textbf{Aumento da Vantagem com a Saliência}: À medida que a saliência aumenta (movendo-se para a direita no gráfico), a vantagem comparativa das \acrshort{ong}s se acentua significativamente. O efeito do lobby de empresas e de outros atores decai rapidamente, enquanto o efeito das \acrshort{ong}s se mostra muito mais resiliente, diminuindo a uma taxa consideravelmente menor.
    
    \item \textbf{Efeito Robusto das ONGs em Alta Saliência}: Em temas de alta saliência, onde a influência de empresas e outros grupos se torna estatisticamente indistinguível de zero (seus intervalos de confiança cruzam a linha pontilhada), o efeito das \acrshort{ong}s permanece positivo, robusto e estatisticamente significativo. É precisamente neste contexto de maior escrutínio público que a sua influência relativa se torna mais pronunciada.
\end{enumerate}



Os resultados validam a Hipótese 3. Em temas de maior saliência, o lobby de organizações não empresariais é, de fato, mais eficaz em aumentar a atividade parlamentar em comparação com o lobby empresarial. A nuance importante é que essa maior eficácia não se manifesta como um aumento absoluto do efeito, mas sim como uma \textbf{resiliência superior} à pressão do escrutínio público, o que amplia a sua vantagem comparativa.


Esta descoberta dialoga diretamente com a teoria sobre os recursos do lobby e os incentivos parlamentares. Em temas de baixa saliência, os parlamentares, focados em seus objetivos de formulação de políticas (\textit{policy-seeking}), podem valorizar o subsídio informacional técnico fornecido por empresas. Contudo, quando um tema ganha visibilidade, os incentivos de reeleição (\textit{vote-seeking}) tornam-se dominantes \cite{mayhew2004congress}. Nesse contexto, alinhar-se a interesses empresariais pode ter um custo político elevado, enquanto responder a \acrshort{ong}s, que detêm maior capital de legitimidade \cite{bunea2018legitimacy}, reforça a imagem pública do parlamentar.

Portanto, a saliência atua como um "filtro" que altera o valor dos diferentes recursos de lobby. Em ambientes de alta visibilidade, a "moeda" da legitimidade, mais abundante entre as \acrshort{ong}s, se valoriza, enquanto os recursos informacionais e financeiros das empresas perdem parte de sua eficácia. Os achados confirmam que a influência do lobby é altamente contextual e que, sob o escrutínio público, a vantagem se desloca para os atores percebidos como representantes de interesses mais amplos e difusos.
uível de zero (seus intervalos de confiança cruzam a linha pontilhada), o efeito das \acrshort{ong}s permanece positivo, robusto e estatisticamente significativo. É precisamente neste contexto de maior escrutínio público que a sua influência relativa se torna mais pronunciada.

Os resultados validam a Hipótese 3. Em temas de maior saliência, o lobby de organizações não empresariais é, de fato, mais eficaz em aumentar a atividade parlamentar em comparação com o lobby empresarial. A nuance importante é que essa maior eficácia não se manifesta como um aumento absoluto do efeito, mas sim como uma \textbf{resiliência superior} à pressão do escrutínio público, o que amplia a sua vantagem comparativa.

% Esta descoberta dialoga diretamente com a teoria sobre os recursos do lobby e os incentivos parlamentares. Em temas de baixa saliência, os parlamentares, focados em seus objetivos de formulação de políticas (\textit{policy-seeking}), podem valorizar o subsídio informacional técnico fornecido por empresas. Contudo, quando um tema ganha visibilidade, os incentivos de reeleição (\textit{vote-seeking}) tornam-se dominantes \cite{mayhew2004congress}. Nesse contexto, alinhar-se a interesses empresariais pode ter um custo político elevado, enquanto responder a \acrshort{ong}s, que detêm maior capital de legitimidade \cite{bunea2018legitimacy}, reforça a imagem pública do parlamentar.


% \paragraph{Conclusão e Diálogo com a Literatura}
% Os resultados oferecem um suporte nuançado e, em última análise, mais sofisticado à Hipótese 3. A hipótese inicial, de que o efeito das \acrshort{ong}s *aumentaria* com a saliência, é rejeitada. Em vez disso, descobrimos que o lobby de \acrshort{ong}s é mais \textbf{resiliente} ao escrutínio público.

% Esta resiliência pode ser explicada pela maior legitimidade percebida das \acrshort{ong}s \cite{bunea2018legitimacy}. Quando um tema é altamente visível, os \acrshort{mpe}s, motivados por seus objetivos de reeleição (\textit{vote-seeking}), podem considerar politicamente arriscado serem vistos alinhando-se a interesses empresariais específicos. Em contrapartida, ouvir e responder a \acrshort{ong}s, que são frequentemente vistas como representantes de interesses públicos ou difusos, pode reforçar a imagem positiva do parlamentar.

% Assim, embora a saliência reduza a eficácia do lobby como um todo -- possivelmente porque mais atores competem pela atenção e os parlamentares se voltam para outras fontes de informação --, ela o faz de forma assimétrica. A "moeda" da legitimidade, mais abundante entre as \acrshort{ong}s, parece tornar-se particularmente valiosa em ambientes politicamente contestados e de alta visibilidade, confirmando que a eficácia do lobby não depende apenas dos recursos mobilizados, mas do contexto político em que são empregues.
