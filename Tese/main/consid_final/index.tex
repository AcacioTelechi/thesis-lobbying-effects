\chapter{Considerações finais}
\label{chapter:considfinal}

Esta tese investigará os efeitos do lobby sobre o comportamento parlamentar no \acrshort{pe}, um tema de grande relevância em um contexto de crescente influência de grupos de interesse na política. Para superar os desafios metodológicos inerentes a esse tipo de pesquisa, proponho uma estratégia inovadora que combinou a análise da atividade legislativa dos parlamentares com dados sobre suas interações com lobistas.

A coleta e análise de dados abrangentes sobre a atividade legislativa e as reuniões com lobistas permitirão traçar um panorama detalhado da relação entre lobby e comportamento parlamentar no \acrshort{pe}. Espera-se que os resultados dessa pesquisa contribuam para o debate sobre a transparência e a regulação do lobby na \acrshort{ue}, fornecendo evidências empíricas sobre seus impactos concretos no processo legislativo.

Além disso, esta tese permitirá lançar luz sobre os mecanismos específicos pelos quais o lobby influencia o comportamento parlamentar, identificando os domínios temáticos mais suscetíveis à pressão de grupos de interesse e os fatores que podem moderar essa influência. A compreensão desses mecanismos é fundamental para o desenvolvimento de políticas públicas que promovam a integridade e a representatividade do \acrshort{pe}, garantindo que as decisões legislativas reflitam os interesses da sociedade como um todo, e não apenas de grupos privilegiados.

Outro ponto que espero poder contribuir com a literatura é com o debate sobre comportamento parlamentar. Usualmente, busca-se definir uma outra dimensão como fundamental para explicar o fenômeno. Ao propor o \textit{frameork} de análise, sugiro que as três dimensões são relevantes para entendermos algo tão complexo quanto o comportamento dos parlamentares.


% Efeito maior das ONGS: parlamentares utilizando as 'correias de transmissão'... Steffek e Nanz (2008; citados em Scherer e De Ville, 2022, p.  436)  consideram  que  as  OSCs  funcionariamcomo “correias  de  transmissão”entre  os  níveis  de governança, na medida que conectariamo nível nacional (por meio da agregação das preocupações das partes interessadas nesse nível) aos níveis de governança europeia. (artido fo prof na CG). Esses autores entendem que as  OSCs  promoveriama  formação  de  uma  “esfera  pública”  na  qual  as  escolhas  políticas  seriamsubmetidas  ao  escrutínio  público. 