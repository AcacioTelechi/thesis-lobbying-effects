\chapter{Considerações finais}
\label{chapter:considfinal}

Esta tese investigou os efeitos do lobby sobre o comportamento parlamentar no \acrshort{pe}, um tema de grande relevância num contexto de crescente influência de grupos de interesse na política. O objetivo central foi superar os desafios metodológicos inerentes a este tipo de pesquisa, que dificultam a identificação causal. Para tal, propôs-se uma abordagem inovadora que combinou a análise da "atenção legislativa" (\acrshort{al}) dos parlamentares — uma medida do seu envolvimento ativo num tema — com dados detalhados sobre as suas interações com lobistas. A análise empírica testou três hipóteses centrais, cujos resultados e implicações passamos a discutir.

A \ref{item:H1} postulava que quanto maior o lobby, maior a atividade parlamentar. Os resultados encontrados corroboraram essa hipótese, indicando que cada reunião com lobistas está associada a um aumento de 2,5\% na atividade parlamentar. Ao testarmos os efeitos marginais, na especificação quadrática, encontramos um efeito marginal decrescente - ainda que pequeno -, cujo ponto de inflexão foi em 11,8 reuniões, ou seja, a partir de 12 reuniões, o efeito marginal passa a ser decrescente. Nesse ponto máximo, o efeito esperado na atividade parlamentar é de 77\%. 

Esse achado sugere que atores com maior disponibilidade de recursos enfrentam pouca perda de eficácia ao intensificar o número 
de reuniões e, portanto, podem sustentar níveis muito mais altos de lobby. Tal padrão é consistente com a hipótese de que 
grandes players conseguem alavancar sua capacidade financeira para obter influência relativamente maior, mesmo diante de 
retornos marginais decrescentes.

A \ref{item:H2}, que investigava se as empresas exercem influência agregada superior sobre a atividade parlamentar em comparação 
com outros atores, também foi corroborada. O efeito marginal isolado, contudo, foi insuficiente para um teste robusto da 
hipótese. A influência total de um grupo de interesse não depende apenas da eficácia de cada reunião, mas também da sua 
capacidade de assegurar acesso - isto é, o volume de reuniões que consegue realizar. Argumentamos que o impacto total é uma 
função dessas duas componentes: a frequência do acesso e a eficácia da persuasão em cada encontro. Obtivemos resultados que 
sugerem que as empresas exercem uma influência agregada superior sobre a atividade parlamentar em comparação com outros atores, 
mas que essa influência só se torna dominante quando alavancada por vastos recursos financeiros. Isto é, há diferenças 
relevantes dentro de cada tipo de ator. 

Empresas menores, com menos recursos, exercem uma influência menor mesmo em comparação com as \acrshort{ong}s, mas que essa influência só se torna dominante quando alavancada por vastos recursos financeiros. Para orçamentos abaixo de \$8,8 milhões, o efeito total das \acrshort{ong}s permanece superior. No entanto, acima de \$40 milhões, o efeito agregado das empresas torna-se substancialmente maior. Esses resultados se alinham com a literatura sobre os mecanismos de influência do lobby e reforçam a importância de considerar a heterogeneidade dos atores na análise dos efeitos do lobby sobre o comportamento parlamentar. 

De forma complementar, os resultados mostram que o efeito de persuasão por reunião das \acrshort{ong}s é consistentemente maior do que o das empresas. Tal evidência reforça a hipótese de que estas organizações funcionam como "correias de transmissão" entre a sociedade civil e o nível europeu \cite{pereira2025participacao}. Ao canalizarem as preocupações da sociedade civil e as perspectivas dos cidadãos para os seus representantes no \acrshort{pe}, as \acrshort{ong}s fornecem aos eurodeputados informações valiosas e um tipo de capital político que o lobby empresarial nem sempre consegue oferecer, apesar das diferenças de recursos e do desenho institucional que busca manter os agentes públicos na tomada de decisões políticas na \acrshort{ue} \cite{pereira2025participacao}.


Finalmente, a \ref{item:H3} foi igualmente corroborada, novamente com uma ressalva importante. Em temas de maior saliência pública, o efeito do lobby diminui para todos os atores, mas a influência das \acrshort{ong}s demonstra ser mais resiliente. Isto significa que, sob o escrutínio público, a capacidade de influência das empresas se degrada mais rapidamente que a das \acrshort{ong}s, tornando estas últimas *relativamente* mais influentes. Contudo, o efeito absoluto do lobby, para todos os atores, é maior em temas menos visíveis, longe dos olhares da opinião pública.

Os achados desta tese têm implicações significativas tanto para a teoria sobre representação de interesses como para a prática da regulação democrática. Teoricamente, os resultados desafiam uma visão monolítica da influência política. Ao demonstrar que a influência não é apenas uma função do acesso, mas uma combinação complexa de recursos, tipo de ator e saliência do tema, este trabalho contribui para uma compreensão mais sofisticada do lobby. A descoberta de que as \acrshort{ong}s possuem um maior poder de persuasão por reunião reforça a importância do lobby informacional: a provisão de informação credível e representativa de interesses sociais alargados emerge como uma moeda de troca tão ou mais valiosa que o capital financeiro.

É importante, neste ponto, revisitar a distinção entre influência e poder. A metodologia aqui empregada foi desenhada para medir a influência, entendida como a capacidade de um ator alterar o comportamento de outro. Não se mediu, contudo, o poder no seu sentido mais estrito, que implicaria demonstrar que o lobby levou um parlamentar a agir contra as suas preferências iniciais. Sem dados sobre as posições \textit{a priori} dos eurodeputados e dos lobistas, esta é uma inferência que permanece para investigações futuras, constituindo uma fronteira importante no estudo da influência.

Do ponto de vista prático, os resultados oferecem subsídios para o debate sobre a transparência e a regulação do lobby na \acrshort{ue}. A confirmação de que a influência agregada das empresas está intrinsecamente ligada a orçamentos de grande magnitude fortalece os argumentos em favor de um escrutínio mais apertado sobre o financiamento das atividades de representação de interesses. Por outro lado, o fato de a influência diminuir em temas de alta saliência pública é uma evidência encorajadora: sugere que a atenção dos cidadãos e da mídia funciona como um contrapeso democrático, limitando a capacidade de interesses especiais de dominar o processo político.

Toda investigação científica possui fronteiras, e é fundamental reconhecer as limitações deste estudo para pavimentar o caminho para pesquisas futuras. Em primeiro lugar, embora a "atenção legislativa" seja uma \textit{proxy} robusta, ela não captura a totalidade do comportamento parlamentar. Atividades cruciais como as negociações informais em comitês ou a formação de coligações permanecem por explorar com este tipo de dados. Em segundo lugar, a análise depende da qualidade dos dados do Registo de Transparência da \acrshort{ue}, que, apesar de abrangentes, podem não captar todas as interações de natureza informal.

Estas limitações, contudo, abrem portas para uma rica agenda de pesquisa futura. A metodologia aqui desenvolvida poderia ser aplicada a outros parlamentos para testar a generalização dos resultados. A combinação desta análise quantitativa com métodos qualitativos, como entrevistas em profundidade, poderia validar os mecanismos de persuasão aqui identificados. Adicionalmente, futuras investigações poderiam focar em medir os efeitos do lobby em outros resultados legislativos, como o sucesso na aprovação de emendas.

Em suma, esta tese procurou iluminar a "caixa preta" da influência política no \acrshort{pe}. Os resultados revelam um ecossistema político complexo. Demonstramos que, no processo legislativo europeu, os recursos financeiros são um fator inegável de influência, mas não o único. A credibilidade da informação, a representatividade dos interesses e o escrutínio público emergem como forças igualmente decisivas que moldam o comportamento dos representantes eleitos. Ao oferecer uma medida quantitativa rigorosa destas dinâmicas, esta investigação contribui não só para o avanço acadêmico, mas também para um debate mais informado sobre a qualidade e a resiliência da democracia supranacional no século XXI. 
