\chapter{Considerações finais}
\label{chapter:considfinal}

Esta tese investigou os efeitos do lobby sobre o comportamento parlamentar no \acrshort{pe}, um tema de grande relevância em um contexto de crescente influência de grupos de interesse na política. Para superar os desafios metodológicos inerentes a esse tipo de pesquisa, propus uma estratégia que combinou a análise da atividade legislativa dos parlamentares com dados sobre suas interações com lobistas. Analisemos cada uma das hipóteses levantadas e as implicações dos resultados encontrados.

A \ref{item:H1} postulava que quanto maior o lobby, maior a atividade parlamentar. Os resultados encontrados corroboraram essa hipótese, indicando cada reunião com lobistas está associada a um aumento de 2,5\% na atividade parlamentar. Ao testarmos os efeitos marginais, na especificação quadrática, encontramos efeito marginal decrescente - ainda que pequeno. Esse achado sugere que atores com maior disponibilidade de recursos enfrentam pouca perda de eficácia ao intensificar o número de reuniões e, portanto, podem sustentar níveis muito mais altos de lobby. Tal padrão é consistente com a hipótese de que grandes players conseguem alavancar sua capacidade financeira para obter influência relativamente maior, mesmo diante de retornos marginais decrescentes.

A \ref{item:H2}, que investigava se as empresas exercem influência agregada superior sobre a atividade parlamentar em comparação com outros atores, também foi corroborada. O efeito marginal isolado, contudo, foi insuficiente para um teste robusto da hipótese. A influência total de um grupo de interesse não depende apenas da eficácia de cada reunião, mas também da sua capacidade de assegurar acesso - isto é, o volume de reuniões que consegue realizar. Argumentamos que o impacto total é uma função dessas duas componentes: a frequência do acesso e a eficácia da persuasão em cada encontro. Obtivemos resultados que sugerem que as empresas exercem uma influência agregada superior sobre a atividade parlamentar em comparação com outros atores, mas que essa influência só se torna dominante quando alavancada por vastos recursos financeiros. Isto é, há diferenças relevantes dentro de cada tipo de ator. Empresas menores, com menos recursos, exercem uma influência menor mesmo em comparação com as \acrshort{ong}s, mas que essa influência só se torna dominante quando alavancada por vastos recursos financeiros. Para orçamentos abaixo de \$8,8 milhões ($\approx e^{16}$), o efeito total das \acrshort{ong}s permanece superior. No entanto, acima de \$40 milhões ($\approx e^{17.5}$), o efeito agregado das empresas torna-se substancialmente maior. Esses resultados se alinham com a literatura sobre os mecanismos de influência do lobby e reforçam a importância de considerar a heterogeneidade dos atores na análise dos efeitos do lobby sobre o comportamento parlamentar.

A \ref{item:H3}, que postulava que quanto maior a saliência do tema, maior a influência do lobby de organizações não empresariais em comparação com o lobby de organizações empresariais, também foi corroborada, porém com ressalvas. O efeito encontrado em função da saliência do tema foi negativo para todos os atores, indicando que o efeito do lobby diminui à medida que o tema se torna mais saliente. No caso de \acrshort{ong}s, a diminuição é menos acentuada do que no caso das empresas e outros atores. Esses resultados sugerem que a influência do lobby de \acrshort{ong}s é mais resiliente a pressões de escrutínio público. De fato, então, quanto maior a saliência do tema, maior a chance de sucesso do lobby de \acrshort{ong}s relativa ao lobby de empresas, todavia o efeito do lobby em temas menos salientes é maior independente do tipo de ator.

% Efeito maior das ONGS: parlamentares utilizando as 'correias de transmissão'... Steffek e Nanz (2008; citados em Scherer e De Ville, 2022, p.  436)  consideram  que  as  OSCs  funcionariamcomo “correias  de  transmissão”entre  os  níveis  de governança, na medida que conectariamo nível nacional (por meio da agregação das preocupações das partes interessadas nesse nível) aos níveis de governança europeia. (artido fo prof na CG). Esses audtores entendem que as  OSCs  promoveriama  formação  de  uma  “esfera  pública”  na  qual  as  escolhas  políticas  seriamsubmetidas  ao  escrutínio  público. 



% Discussão sobre poder: não medi os interesses a priori, apenas os resultados. 

% Há uma relação entre recurso e poder: porém não é linea. Não há uma tradução direta (Baldwin)